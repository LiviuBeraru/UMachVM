\usepackage[nonumberlist,toc]{glossaries}
\renewcommand{\glossaryname}{Glossar}
\makeglossaries

% Glossary definitions
% Definition format: \newglossaryentry{label}{key=value...}
% Keys:
% name        - the name which will appear in text
% description - the description of the text, suround it with {}
% plural      - plural form
%
% Using a glossary term: 
% \gls{label}   for singular
% \glspl{label} for plural
% \Gls{label}   for first letter uppercase
% \Glspl{label} for first letter uppercase plural

\newglossaryentry{Byte}{
name=Byte,
description={eine Reihe oder Gruppe von 8 Bit},
plural={Bytes}
}

\newglossaryentry{Befehl}{
name=Befehl,
description={Die ersten 8 Bits in einer Instruktion. Operation code},
plural={Befehle}
}



\newglossaryentry{Instruktion}{
name=Instruktion,
description={Eine Anweisung an die UMach VM etwas zu tun. Besteht aus einem
Befehl (Operation Code) und eventuellen Argumenten},
plural={Instructionen}
}

\newglossaryentry{Instruktionssatz}{
name=Instruktionssatz,
description={Die Menge aller Instruktionen},
plural=Instruktionssätze
}
