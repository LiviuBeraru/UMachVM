\documentclass[fontsize=12pt,parskip=full,index=totoc]{scrreprt}

\addtokomafont{disposition}{\rmfamily}
\addtokomafont{descriptionlabel}{\rmfamily}


\usepackage[utf8]{inputenc}
\usepackage[T1]{fontenc}
\usepackage[ngerman]{babel}

\usepackage{lmodern}
\usepackage{ellipsis}
\usepackage{microtype}
\usepackage{xcolor}
\usepackage{amsmath,amsfonts}

\usepackage{scrhack}%entfernt warnmeldungen von listings
\usepackage{listings}
\lstset{basicstyle=\small\ttfamily}
\lstset{commentstyle=\footnotesize\slshape}
\lstset{showstringspaces=false}
\lstset{frame=leftline}
\lstset{morecomment=[l][\color{blue!20!black}]{\#}}

\usepackage{makeidx}
\makeindex

\title{UMach Spezifikation}
%\author{ Werner Linne \and Willi Fink \and Simon Beer \and Liviu Beraru }
\author{}

\usepackage[
    pdftitle={UMach VM Spezifikation},
    colorlinks=true,
    linkcolor=blue!20!black]
{hyperref} 

\usepackage[nonumberlist,toc]{glossaries}
\renewcommand*{\glspostdescription}{}% kein automatischer Punkt am Ende
\makeglossaries

% Glossary definitions
% Definition format: \newglossaryentry{label}{key=value...}
% Keys:
% name        - the name which will appear in text
% description - the description of the text, suround it with {}
% plural      - plural form
%
% Using a glossary term: 
% \gls{label}   for singular
% \glspl{label} for plural
% \Gls{label}   for first letter uppercase
% \Glspl{label} for first letter uppercase plural


\newglossaryentry{Adressierungsart}{
name={Adressierungsart},
description={Die Art, wie eine Instruktion die Umach Maschine dazu veranlasst,
einen Speicherbereich zu adressieren. Siehe auch Abschnitt
\ref{subsec:Adressierungsarten}.},
plural={Adressierungsarten}
}


\newglossaryentry{Byte}{
name=Byte,
description={Eine Reihe oder Gruppe von 8 Bit.},
plural={Bytes}
}

\newglossaryentry{Befehl}{
name=Befehl,
description={Die ersten 8 Bits in einer Instruktion. Operation code.},
plural={Befehle}
}

\newglossaryentry{Befehlsbreite}{
name=Befehlsbreite,
description={Die Länge eines Befehls in Bits. Ein UMach-Befehl ist 8 Bit lang.},
plural={Befehlsbreiten}
}

\newglossaryentry{Befehlsraum}{
name=Befehlsraum,
description={Die Anzahl der möglichen Befehle, abhängig von der
Befehlsbreite. Beträgt die Befehlsbreite $8$ Bit, so ist der Befehlsraum
$2^{8} = 256$.},
plural=Befehlsräume
}

\newglossaryentry{Instruktion}{
name=Instruktion,
description={Eine Anweisung an die UMach VM etwas zu tun. Eine Instruktion
besteht aus einem Befehl (Operation Code) und eventuellen Argumenten.},
plural={Instructionen}
}

\newglossaryentry{Instruktionsbreite}{
name=Instruktionsbreite,
description={Die Länge einer Instruktion in Bits. Eine UMach-Instruktion ist 32 Bit lang.},
plural={Instruktionsbreiten}
}

\newglossaryentry{Instruktionsformat}{
name={Instruktionsformat},
description={Beschreibt die Struktur einer Instruktion auf Byte-Ebene und zwar
es gibt an, ob ein Byte als eine Registerangabe oder als reine
numerische Angabe zu interpretieren ist. Siehe \ref{sec:Instruktionsformate}.},
plural={Instruktionsformate}
}

\newglossaryentry{Instruktionssatz}{
name=Instruktionssatz,
description={Die Menge aller Instruktionen, die von der UMach Maschine
ausgeführt werden können.},
plural=Instruktionssätze
}

\newglossaryentry{Register}{
name={Register},
description={Eine sich im Prozessor befindende Speichereinheit. Das Register
ist dem Programmierer sichbar und kann mit Werten geladen werden. Siehe
Abschnitt \ref{sec:Register}, Seite \pageref{sec:Register}.},
plural={Register}
}


\usepackage{blindtext}

\begin{document}
\maketitle
\tableofcontents

%--- Jedes Kapitel in eigener Datei und eigenem Verzeichnis 
%--- Jede Kapitel-Datei fängt mit \chapter an (markiert Kapitel)
%--- und kann nach belieben andere \section enthalten
%--- Empfehlung: jeder Abschnitt \section in eigener Datei

\chapter{Einführung}

\section{Beispiel}
Gleich am Anfang soll ein Beispiel für die Verwendung der UMach
virtuellen Maschine und des Assemblers gegeben werden.

Wir haben ein UMach-Programm in eine normale Textdatei geschrieben. Das Programm
kann sich über mehrere Dateien erstrecken, aber hier verwenden wir nur eine
Datei. Das Programm sieht wie folgt aus:

\begin{lstlisting}
    set r1 3
loop:
    dec r1
    cmp r1 zero
    be  finish
    jmp loop
finish: 
    EOP
\end{lstlisting}

Dieses Programm wurde in der Datei \texttt{text.um} gespeichert (die Endung ist
egal). Wir gehen davon aus, dass der Assembler \texttt{uasm}, die Programmdatei
\texttt{test.um} und die virtuelle Maschine \texttt{umach} sich in dem selben
Verzeichniss befinden. Sonst muss man die Befehle entsprechend anpassen.

Das Programm kann wie folgt assembliert werden:
\begin{lstlisting}
 ./uasm -o prog.ux test.um
\end{lstlisting}
Die Option \texttt{-o} gibt die Ausgabedatei an. Wird diese Option nicht
angegeben, so wird das assemblierte Programm in die Datei \texttt{u.out}
geschrieben. Ergebnis des Assemblers is also die Datei \texttt{prog.ux}. Jetzt
kann man diese Datei \glqq ausführen\grqq:
\begin{lstlisting}
 ./umach prog.ux
\end{lstlisting}
Das Programm beendet sich ohne Ausgabe. Starten wird also das Programm im
Debug-Modus (Option \texttt{-d}):
\begin{lstlisting}
 ./umach -d prog.ux
\end{lstlisting}
Es wird die erste Anweisung angezeigt, der aktuelle Programm-Counter (Inhalt
des Registers \texttt{PC}) und ein Prompt, der auf eine Eingabe von uns wartet.
So könntes es weiter gehen:
\begin{lstlisting}
[256]   SET   R1    3
umdb > show reg r1
R1 = 0x00000000 = 0
umdb > s
[260]   DEC   R1
umdb > show reg r1
R1 = 0x00000003 = 3
umdb > s
[264]   CMP   R1    ZERO  
umdb > s
[268]   BE    2
umdb > s
[272]   JMP   -3
umdb > s
[260]   DEC   R1
umdb > s
[264]   CMP   R1    ZERO  
umdb > show reg r1
R1 = 0x00000001 = 1
umdb > s
[268]   BE    2
umdb > s
[272]   JMP   -3
umdb > s
[260]   DEC   R1
umdb > s
[264]   CMP   R1    ZERO  
umdb > s
[268]   BE    2
umdb > s
[276]   EOP   
umdb > s
umdb > s
The maschine is not running.
umdb > show reg r1 cmpr
R1 = 0x00000000 = 0
CMPR = 0x00000000 = 0
umdb > q
\end{lstlisting}


\chapter{Struktur der UMach VM}
\blindtext

\subsection{Betriebsmodi}
\label{subsec:Betriebsmodi}
\index{Betriebsmodus}

Die Umach VM kann in zwei verschiedenen Betriebsmodi oder auch
Betriebsarten betrieben werden:

\begin{enumerate}
  \item Normalmodus
  \item Debugmodus
\end{enumerate}

\paragraph{Normalmodus} Die virtuelle Maschine führt alle Instruktionen
ohne Unterbrechung aus. Nach der Ausführung schaltet sich die Maschine aus.

\paragraph{Debugmodus} Die virtuelle Maschine führt immer nur eine
einzige Instruktion aus und bleibt dannach stehen. Um mit der folgenden
Instruktion fortzufahren wird immer ein externen Signal benötigt.
Dieser Modus soll dem Entwickler erlauben, ein Programm schrittweise zu
debuggen.

Der Modus kann vor dem Hochfahren der Maschine festgelegt, jedoch im
Betrieb nicht mehr geändert werden. Wird kein Modus angegeben, startet die
Maschine im Normalmodus.




\section{Aufbau}
\index{UMach!Aufbau}
\label{sec:Aufbau}
Die virtuelle Maschine besteht aus einem Kern (der eigentlichen UMach Maschine)
und aus einem Bussystem. Das Bussystem wird im Abschnitt \ref{sec:Peripherie}
ab der Seite \pageref{sec:Peripherie} beschrieben.

Der Kern der Maschine besteht aus den folgenden Komponenten:

\begin{enumerate}
  \item Befehlsabruf Einheit (Instruction Fetch Unit)
  \item Decodierungseinheit
  \item Recheneinheit
  \item Register
\end{enumerate}

Die Decodierungseinheit ist dafür zuständig, eine abgerufene Instruktion zu
decodieren, bzw. in ihren Komponenten zu zerteilen.

Die Recheneinheit ist für die tatsächliche Ausführung der Instruktionen
zuständig. 

Die Register sind die Speichereinheiten, die sich in der Maschine befinden.






\chapter{Instruktionssatz}\index{Instruktionssatz}

In diesem Abschnitt werden alle Instruktionen (Befehle) der UMach VM
vorgestellt.

\section{Instruktionsformate}
\blindtext

\section{Klassifizierung der Instruktionen}
Zur besseren Übersicht der verschiedenen UMach-Befehlen, unterteilen wir die
Menge aller Befehlen in den folgenden Kategorien
(Abschnitt \ref{sec:Instruktionen} ab der Seite
\pageref{sec:Instruktionen} beinhaltet eine komplette Liste aller
Instruktionen):

\begin{enumerate}
  \item Kontrolle-Befehle: das sind Befehlen, die die Maschine selbst
    kontrollieren, wie z.B. den Betriebsmodus umschalten oder Ausschalten.
  \item Arithmetische Befehle: Addieren, Subtrahieren etc. Alle arithmetische
    Befehle operieren auf Register Inhalte.
  \item Logische Befehle.
  \item Programflußkontrolle.
  \item Sprünge.
  \item Load- und Store-Befehle: die einzigen Befehle, die einen Zugriff auf den
    Speicher ausführen.
  \item Input-Output-Befehle: operieren auf die I/O-Einheit der UMach VM.
\end{enumerate}


\section{Instruktionen}\index{Instruktionen}
\label{sec:Instruktionen}

\blindtext





\appendix

\printglossary[title=Glossar,toctitle=Glossar]
\printindex

\end{document}
