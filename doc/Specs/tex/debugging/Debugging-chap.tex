\chapter{Debugging}
\label{chap:Debugging}

Die virtuelle UMach Maschine stellt eine Debugging-Schnittstelle zur Verfügung.
Um diese Schnittstelle zu aktivieren, muss man die Maschine im Debugmodus
starten.

Die Schnittstelle besteht aus einer Reihe von Funktionen und aus einer Sammlung
von Informationen betreffend den inneren Zustand der Maschine zum Zeitpunkt der
Abfrage.

Folgende Funktionen dienen zur Steuerung der Maschine wärend sie die Ausführung
des Programms durchführt:

\begin{enumerate}
 \item \texttt{step}\\
       Führt die aktuelle Instruktion aus, lädt die nächste Instruktion aus
       dem Speicher und dann wartet.
\end{enumerate}


Die Informationen, die während einer Debug-Session zur Verfügung gestellt
werden sind die folgenden:
\begin{enumerate}
 \item PC, der aktuelle Program-Counter.
 \item Aktuelle Instruktion. Diese Instruktion wurde noch nicht ausgeführt.
 \item Werte aller Register.
 \item Informationen über den Speicher: Größe etc.
 \item Ausschnitte aus dem Speicherinhalt.
 \item Informationen über die I/O Einheit.
\end{enumerate}

