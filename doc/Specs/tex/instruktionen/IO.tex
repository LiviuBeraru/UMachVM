\section{IO Instruktionen}
\label{sec:IO-Instruktionen}



\opdef{IN}{$R, P\in \Reg$}{0xB0}{RR0}

Liest den Inhalt des Eingabeports mit Nummer $P$ in das Register $R$.
Diese Operation blockiert solange die Programausführung, bis Daten eingelesen
wurden.






\opdef{OUT}{$R, P, K \in \Reg$}{0xB8}{RRR}

Sendet den Inhalt des Registers $R$ zum Ausgangsport mit Nummer $P$. Dabei wird
die Kennung der Daten aus $R$ auf den Inhalt des Registers $K$ gesetzt.
Siehe auch den Abschnitt \ref{subsec:Ausgabeports}, auf der Seite
\pageref{subsec:Ausgabeports}.
\[
    R \to P
\]
$R$, $P$ und $K$ sind alle Register.

\paragraph{Beispiel}
Im folgenden Beispiel wird den Inhalt des Registers $R1$ an verschiedenen Ports
ausgegeben. Die Register \texttt{R2} und \texttt{ZERO} werden dabei für die
Kennung und Portnummer verwendet.

\begin{lstlisting}
  OUT R1 ZERO ZERO # R1 auf Port 0 mit Kennung 0 ausgeben
  SET R2 5         # port 5

  SET R1 10        # Inhalt in R1 ist newline
  OUT R1 R2   ZERO # newline zum Port 5, Kennung 0
  OUT R1 R2   R2   # nochmal, mit Kennung 5 auf Port 5
\end{lstlisting}

