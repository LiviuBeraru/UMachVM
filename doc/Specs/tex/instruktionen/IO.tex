\section{IO Instruktionen}
\label{sec:IO-Instruktionen}



\opdef{IN}{$A, N, P \in \Reg$}{0xB0}{RRR}

Veranlasst die I/O-Einheit, $\$N$ Bytes Daten aus dem Gerät am Port $\$P$, zum
Speicher ab der Adresse $\$A$ zu übertragen. $A$, $N$ und $P$ sind Register.

\[
  read(\$P) \to mem_{\$N}(\$A)
\]

Dieser Befehl blockiert den Kern solange, bis entweder $\$N$ Bytes gelesen
wurden, oder bis das Gerät das Ende der Übertragung signalisiert. Die
tatsächliche Anzahl der übetragenen Bytes kann kleiner als $\$N$ sein und wird
in das Register $N$ gespeichert.


\paragraph{Fehler}
Falls $\$P$ keine gültige Portnummer ist, wird der Interrupt mit Nummer 32
generiert und das Bit mit Nummer 16 im Register \texttt{ERR} gesetzt.
Falls die Speicheradressen $\$A$ bis $\$A + \$N$ keine gültigen Speicheradressen
sind, d.h. falls es sie gar nicht gibt, wird der Interrupt mit Nummer 16
(ungültige Speicheradresse) generiert und das Bit mit Nummer 8 im Register
\texttt{ERR} gesetzt. Falls der Speicherbereich $\$A$ bis $\$A + \$N$
schreibgeschützt ist, wird der Interrupt mit Nummer 17 (segfault) erzeugt.


\opdef{OUT}{$A, N, P \in \Reg$}{0xB8}{RRR}
Schreibt $\$N$ Bytes Daten aus dem Speicher ab der Adresse $\$A$ an den Port mit
Nummer $\$P$. Die Anzahl der tatsächlich geschriebenen Bytes wird zurück in das
Register $N$ geschrieben und beträgt $0$ bis $\$N$. Eine niedrigere Zahl
as $\$N$ wird z.B. dann erreicht, wenn das Ausgabegerät die Ausgabe aus
technischen Gründen verweigert (ausgeschaltet) oder nicht in der gewünschten
Anzahl ausführen kann.

Dieser Befehl blockiert den Kern solange, bis die I/O-Einheit fertig mit dem
Schreiben ist.

\paragraph{Fehler}
Wie bei dem Befehl \opref{IN}.


\paragraph{Beispiel}
Der folgende Code demonstriert die Benutzung der \texttt{IN} und \texttt{OUT}
Befehle in einem \glqq Hello User\grqq\ Programm. Der Benutzer wird nach seinem
Namen gefragt und dann wird er namentlich begrüßt.

\begin{lstlisting}
SET R1 prompt   # Adresse der ersten Ausgabe
SET R2 10       # Länge der Ausgabe
SET R3 0        # Portnummer 0
OUT R1 R2 R3    # Ausgeben "Your name: "

SET R1 name     # Speicheradresse
SET R2 16       # wieviel input maximal
SET R3 0        # Port 0 (Tastatur)
                # Port 0 für IN und OUT sind getrennt
IN  R1 R2 R3    # oder auch IN R1 R2 ZERO
### blockiert bis input fertig ###
### in R2 steht wieviel tatsächlich gelesen ###

SET R4 hello
SET R5 7        # strlen(hello)
OUT R4 R5 R3    # output R5 bytes auf Port 0
OUT R1 R2 ZERO  # "Hello, User"

SET R4 nl
SET R5 1
OUT R4 R5 ZERO  # '\n'

.data
array 16 name
string promtp "Your name: "
string hello  "Hello, "
string nl     "0x10"
\end{lstlisting}

