\section{Instruktionsformate}
\label{sec:Instruktionsformate}
\index{Instruktionsformat}

Eine Instruktion besteht aus einer Folge von 4 Bytes.
Das \gls{Instruktionsformat} beschreibt die Struktur einer Instruktion auf
Byte-Ebene. Das Format gibt an, ob ein Byte als eine Registerangabe oder als reine
numerische Angabe zu interpretieren ist.

\paragraph{Instruktionsbreite}
\index{Instruktionsbreite}
Jede UMach-Instruktion hat eine feste Bitlänge von 32 Bit (4 mal 8 Bit).
Instruktionen, die für ihren
Informationsgehalt weniger als 32 Bit brauchen, wie z.B. \texttt{NOP},
werden mit Nullbits gefüllt. Alle Daten und Informationen, die mit einer
Instruktion übergeben werden, müssen in diesen 32 Bit untergebracht werden.

\paragraph{Byte Order}
\index{Byte Order}
Die Byte Order (Endianness) der gelesenen \glspl{Byte} ist big-endian.
Die zuerst gelesenen 8 Bits sind die 8 höchstwertigen (Wertigkeiten $2^{31}$ bis
$2^{24}$) und die zuletzt gelesenen Bits sind die niedrigstwertigen
(Wertigkeiten $2^{7}$ bis $2^{0}$).
Bits werden in Stücken von $n$ Bits gelesen, wobei $n = k \cdot 8$ mit
$k \in \{1, 4\}$ (byteweise oder wortweise).


\paragraph{Allgemeines Format}
Jede \gls{Instruktion} besteht aus zwei Teilen: der erste Teil ist
8 Bit lang und entspricht dem tatsächlichen \gls{Befehl}, bzw. der Operation,
die von der UMach virtuellen Maschine ausgeführt werden soll.
Dieser 8-Bit-Befehl belegt also die 8 höchstwertigen Bits einer
32-Bit-Instruktion.  Die übrigen 24 Bits, wenn sie verwendet werden, werden
für Operanden oder Daten benutzt. Beispiel einer Instruktionszerlegung:

\begin{center}
  \begin{tabular}{|l|*{4}{c|}}
    \hline
    Instruktion (32 Bit) &
    \texttt{00000001} & \texttt{00000010} & \texttt{00000011} & \texttt{00000100}
    \\\hline
    Hexa  &
    \texttt{01}   & \texttt{02}   & \texttt{03}   & \texttt{04}
    \\\hline
    Byte Order &
    erstes Byte   & zweites Byte  & drittes Byte  & viertes Byte
    \\\hline
    Interpretation &
    Befehl (8 Bit) &  \multicolumn{3}{c|}{Operanden, Daten oder Füllbits}
    \\\hline
  \end{tabular}
\end{center}

Die Instruktionsformate unterscheiden sich lediglich darin, wie sie die 24 Bits
nach dem 8-Bit \gls{Befehl} verwenden. Das wird auch in der 3-buchstabigen
Benennung deren Formate wiedergeben.

In den folgenden Abschnitten werden die UMach-Instruktionsformate vorgestellt.
Jede Angegebene Tabelle gibt in der ersten Zeile die Reihenfolge der Bytes an. 
Die nächste Zeile gibt die spezielle Belegung der einzelnen Bytes an.



\subsection{000}
\label{subsec:000}
\index{Instruktionsformat!000}
\index{000}

\begin{center}
  \begin{tabular}{|*{4}{c|}} \hline
    erstes Byte & zweites Byte  & drittes Byte  & viertes Byte \\\hline\hline
    Befehl      & \multicolumn{3}{c|}{nicht verwendet}         \\\hline
  \end{tabular}
\end{center}

Eine Instruktion, die das Format 000 hat, besteht lediglich aus einem Befehl
ohne Argumenten. Die letzen drei Bytes werden von der Maschine nicht
ausgewertet und sind somit Füllbytes. Es wird empfohlen, die letzten 3 Bytes mit
Nullen zu füllen.




\subsection{NNN}
\label{subsec:NNN}
\index{Instruktionsformat!NNN}
\index{NNN}

\begin{center}
  \begin{tabular}{|*{4}{c|}}
    \hline
    erstes Byte  & zweites Byte  & drittes Byte  & viertes Byte \\\hline\hline
    Befehl       & \multicolumn{3}{c|}{numerische Angabe $N$}   \\\hline
  \end{tabular}
\end{center}

Die Instruktion im Format NNN besteht aus einem Befehl im ersten Byte und aus
einer numerischen Angabe  $N$ (einer Zahl), die die letzten 3 Bytes belegt.
Die Interpretation der numerischen Angabe wird dem jeweiligen Befehl überlassen.


\subsection{RNN}
\label{subsec:RNN}
\index{Instruktionsformat!RNN}
\index{RNN}

\begin{center}
  \begin{tabular}{|*{4}{c|}} \hline
    erstes Byte & zweites Byte  & drittes Byte  & viertes Byte   \\\hline\hline
    Befehl      & $R_{1}$ & \multicolumn{2}{c|}{numerische Angabe $N$} \\\hline
  \end{tabular}
\end{center}

Eine Instruktion im Format RNN besteht aus einem Befehl, gefolgt von einer
Register Nummer $R_{1}$, gefolgt von einer festen Zahl $N$, die die letzten
2 Bytes der Instruktion belegt.
Die genaue Interpretation der Zahl $N$ wird dem jeweiligen Befehl überlassen.
Zum Beispiel, die Instruktion
\begin{center}
  \begin{tabular}{|*{4}{c|}} \hline
    erstes Byte & zweites Byte  & drittes Byte  & viertes Byte \\\hline\hline
    \texttt{0x20} & \texttt{0x01} & \texttt{0x02} & \texttt{0x03} \\\hline
  \end{tabular}
\end{center}
wird folgenderweise von der UMach Maschine interpretiert: die Operation mit
Nummer \texttt{0x20} soll ausgeführt werden, wobei die Argumenten dieser
Operation sind das Register mit Nummer \texttt{0x01} und die numerische
Angabe \texttt{0x0203}.



\subsection{RRN}
\label{subsec:RRN}
\index{Instruktionsformat!RRN}
\index{RRN}

\begin{center}
  \begin{tabular}{|*{4}{c|}} \hline
    erstes Byte  & zweites Byte  & drittes Byte  & viertes Byte \\\hline\hline
    Befehl       & $R_{1}$       & $R_{2}$ & numerische Angabe $N$  \\\hline
  \end{tabular}
\end{center}
Eine Instruktion im Format RRN besteht aus einem Befehl, gefolgt von der
Angabe zweier Registers $R_{1}$ und $R_{2}$, jeweils in einem Byte, gefolgt von
einer numerischen Angabe $N$ (festen Zahl) im letzten Byte. Zum Beispiel, die
Instruktion
\begin{center}
  \begin{tabular}{|*{4}{c|}} \hline
    erstes Byte & zweites Byte  & drittes Byte  & viertes Byte \\\hline\hline
    \texttt{0x41} & \texttt{0x01} & \texttt{0x02} & \texttt{0x03} \\\hline
  \end{tabular}
\end{center}
soll wie folgt interpretiert werden: 
die Operation mit Nummer \texttt{0x41} soll ausgeführt werden, wobei die
Argumenten dieser Operation sind Register mit Nummer \texttt{0x01}, Register mit
Nummer \texttt{0x02} und die Zahl \texttt{0x03}.



\subsection{RRR}
\label{subsec:RRR}
\index{Instruktionsformat!RRR}
\index{RRR}

\begin{center}
  \begin{tabular}{|*{4}{c|}} \hline
    erstes Byte & zweites Byte  & drittes Byte  & viertes Byte \\\hline\hline
    Befehl      & $R_{1}$       & $R_{2}$       & $R_{3}$      \\\hline
  \end{tabular}
\end{center}

Eine Instruktion im Format RRR besteht aus der Angabe eines Befehls im ersten
Byte, gefolgt von der Angabe dreier Register $R_{1}$, $R_{2}$ und $R_{3}$ in den
jeweiligen folgenden drei Bytes.
Die Register werden als Zahlen angegeben und deren Bedeutung hängt vom
jeweiligen Befehl ab.


\subsection{Zusammenfassung}
\label{subsec:Instr-Formate-Zusammenfassung}
Im folgenden werden die Instruktionsformate tabellarisch zusammengefasst.

\index{Instruktionsformat!Liste}
\begin{center}
  \begin{tabular}{|l||*{4}{c|}}
    \hline
    Format & erstes Byte & zweites Byte  & drittes Byte  & viertes Byte
    \\\hline\hline
    000 & Befehl & \multicolumn{3}{c|}{nicht verwendet}                 \\\hline
    NNN & Befehl & \multicolumn{3}{c|}{numerische Angabe $N$}           \\\hline
    RNN & Befehl & $R_{1}$ & \multicolumn{2}{c|}{numerische Angabe $N$} \\\hline
    RRN & Befehl & $R_{1}$ & $R_{2}$ & numerische Angabe $N$            \\\hline
    RRR & Befehl & $R_{1}$ & $R_{2}$ & $R_{3}$                          \\\hline
  \end{tabular}
\end{center}



