\section{Instruktionsformat}
\index{Instruktionsformat}

\paragraph{Befehlsbreite}
\index{Befehlsbreite}
Jeder UMach-Befehl hat eine feste Bitlänge von 32 Bit (4 mal 8 Bit).
Das gilt unabhängig davon, was der Befehl tut. Befehle die weniger als 32 Bit
brauchen, wie z.B. \texttt{NOP}, werden mit Nullbits gefüllt.
Alle Daten und Informationen, die mit einem Befehl übergeben werden, müssen in
diesen 32 Bit untergebracht werden. Die übrigen 24 Bit werden für Operanden oder
Daten benutzt.

\paragraph{Byte Order}
\index{Byte Order}
Die Byte Order (Endianness) der gelesenen Bytes (Gruppen von 8 Bit jeweils) ist
big-endian, die Byte-Reihenfolge, die für den Mensch selbstverständlich wäre.


\paragraph{Allgemeines Format}
Jeder Befehl hat zwei Teile: der erste Teil ist 8 Bit lang und beinhaltet den
tatsächlichen Befehl, bzw. die Operation, die ausgeführt werden sollte. Es gibt
$2^{8} = 256$ mögliche Befehle.

\begin{center}
  \begin{tabular}{l|*{4}{c|}}
    Binär &
    \texttt{0000} & \texttt{0001} & \texttt{0010} & \texttt{0011} \\\hline
    Hexa  &
    \texttt{00} & \texttt{01} & \texttt{02} & \texttt{03}         \\\hline
    Byte Order &
    erstes Byte & zweites Byte & drittes Byte & viertes Byte      \\\hline
    Interpretation &
    Befehl &
    \multicolumn{3}{c|}{Operanden, Daten} \\\hline
  \end{tabular}
\end{center}


