\section{Instruktionsformate}
\index{Instruktionsformat}

\paragraph{Instruktionsbreite}
\index{Instruktionsbreite}
Jede UMach-Instruktion hat eine feste Bitlänge von 32 Bit (4 mal 8 Bit).
Das gilt unabhängig davon, was die Instruktion tut. Instruktionen, die für ihren
Informationsgehalt weniger als 32 Bit brauchen, wie z.B. \texttt{NOP},
werden mit Nullbits gefüllt. Alle Daten und Informationen, die mit einer
Instruktion übergeben werden, müssen in diesen 32 Bit untergebracht werden.

\paragraph{Byte Order}
\index{Byte Order}
Die Byte Order (Endianness) der gelesenen \glspl{Byte} ist
big-endian, die Byte-Reihenfolge, die für den Mensch selbstverständlich wäre
(von links nach rechts lesen):
die zuerst gelesenen 8 Bits sind die 8 höchstwertigen (Wertigkeiten $2^{31}$ bis
$2^{24}$) und die zuletzt gelesenen Bits sind die niedrigstwertigen
(Wertigkeiten $2^{7}$ bis $2^{0}$).
Bits werden in Stücken von $n$ Bits gelesen, wobei $n = k \cdot 8$ und
$k \in \mathbb{N} \setminus\{0\}$ (Byte für Byte).


\paragraph{Allgemeines Format}
Jede \gls{Instruktion} besteht aus zwei Teilen: der erste Teil ist
8 Bit lang und entspricht dem tatsächlichen \gls{Befehl}, bzw. der Operation,
die von der UMach virtuellen Maschine ausgeführt werden soll.
Dieser 8-Bit-Befehl belegt also die 8 höchstwertigen Bits einer Instruktion.
Die übrigen 24 Bit werden für Operanden oder Daten benutzt. Beispiel einer
Instruktionszerlegung:

\begin{center}
  \begin{tabular}{|l|*{4}{c|}}
    \hline
    Instruktion &
    \texttt{00000001} & \texttt{00000010} & \texttt{00000011} & \texttt{00000100}
    \\\hline
    Hexa  &
    \texttt{01}   & \texttt{02}   & \texttt{03}   & \texttt{04}
    \\\hline
    Byte Order &
    erstes Byte   & zweites Byte  & drittes Byte  & viertes Byte
    \\\hline
    Interpretation &
    Befehl    &  \multicolumn{3}{c|}{Operanden, Daten oder Füllbits}
    \\\hline
  \end{tabular}
\end{center}

Die Instruktionsformate unterscheiden sich lediglich darin, dass sie die 24 Bits
nach dem 8-Bitigen \gls{Befehl} unterschiedlich verwenden.

