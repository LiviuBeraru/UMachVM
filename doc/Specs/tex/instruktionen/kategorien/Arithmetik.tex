\subsection{Arithmetische Instruktionen} 

\opdef{ADD}{$X,Y,Z \in \Reg$}{0x40}{RRR}
Vorzeichen behaftete Addition der Registerinhalte $Y$ und $Z$ und
speichern des Ergebnisses in das Register $X$. Entspricht dem algebraischen
Ausdruck
\[
    X \gets Y + Z
\]
Beispiel:
\begin{lstlisting}
  SET   R1 1     # $R1 \gets 1$
  SET   R2 2     # $R2 \gets 2$
  ADD   R3 R1 R2 # $R3 \gets R1 + R2 = 1 + 2 = 3$
  #     X  Y  Z
  SET   R2 -2    # $R2 \gets -2$
  ADD   R3 R3 R2 # $R3 \gets R3 + R2 = 3 +(-2) = 1$
  ADD   R3 R4  5 # Fehler! 5 kein Register
\end{lstlisting}
Vorzeichenlose Addition wird durch den Befehl \texttt{ADDU} ausgeführt.



\opdef{ADDI}{$X,Y\in\Reg$, $N\in\mathds{Z}$}{0x41}{RRN}
Hinzuaddieren eines festen vorzeichenbehafteten ganzzahligen Wert $N$ zum Inhalt
des Registers $Y$ und speichern des Ergebnisses in das Register $X$.
Entspricht dem algebraischen Ausdruck
\[
  X \gets Y + N
\]
$N$ wird als vorzeichenbehaftete 8-Bit Zahl in Zweierkomplement-Darstellung
interpretiert und kann entsprechend Werte von $-128$ bis $127$ aufnehmen.

Beispiel:
\begin{lstlisting}
  SET   R1 1     # $R1 \gets 1$
  ADDI  R2 R1 2  # $R2 \gets R1 + 2 = 1 + 2 = 3$
  #     X  Y  N
  ADDI  R2 R2 -3 # $R2 \gets R2 + (-3) = 3 +(-2) = 1$
  ADDI  R2 R3 R4 # Fehler! R4 kein $n \in \mathds{Z}$
\end{lstlisting}



\opdef{ADDU}{$X,Y,Z \in \Reg$}{0x42}{RRR}
Vorzeichenlose Addition der Register $Y$ und $Z$. Das Ergebnis wird in das
Register $X$ gespeichert. Enthält $Y$ oder $Z$ ein Vorzeichen (höchstwertiges
Bit auf 1 gesetzt), so wird es nicht als solches interpretiert, sondern als
Wertigkeit, die zum Betrag des Wertes hinzuaddiert wird ($+2^{31}$).

\begin{lstlisting}
  SET   R1 1     # $R1 \gets 1$
  SET   R2 -2    # $R2 \gets -2$
  ADDU  R3 R1 R2 # $R3 \gets (1 + 2 + 2^{31}) = 2147483651$
\end{lstlisting}



\opdef{ADDUI}{$X,Y \in \Reg$, $N\in\mathds{N}$}{0x43}{RRN}
Vorzeichenlose Addition des ganzzahligen Wertes $N$ zum Inhalt des Registers $Y$
und speichern des Ergebnisses in das Register $X$.
Der Inhalt des Registers $Y$, die Zahl $N$ und das Ergebnis $Y + N$ werden als
vorzeichenlose Werte interpretiert.


