\section{Instruktionen}\index{Instruktionen}

Zur besseren Übersicht der verschiedenen UMach-Befehlen, unterteilen wir den
\gls{Instruktionssatz} der UMach virtuellen Maschine in den folgenden Kategorien:
\index{Instruktionen!Kategorien}

\begin{enumerate}
  \item Kontrollbefehle: das sind Befehle, die die Maschine selbst
    kontrollieren, wie z.B. den Betriebsmodus umschalten oder Ausschalten.
  \item Arithmetische Befehle: Addieren, Subtrahieren etc. Alle arithmetische
    Befehle operieren auf Register Inhalte.
  \item Logische Befehle, die eine logische Verknüpfung zwischen den Inhalten
    von Registern veranlassen.
  \item Vergleichsbefehle: Vergleichen von Registerinhalten.
  \item Sprünge.
  \item Load- und Store-Befehle: die einzigen Befehle, die einen Zugriff auf den
    Speicher ausführen.
  \item Input-Output-Befehle: operieren auf die I/O-Einheit der UMach VM.
\end{enumerate}





