\section{Instruktionen}\index{Instruktionen}

Zur besseren Übersicht der verschiedenen UMach-\glspl{Instruktion}, unterteilen
wir den \gls{Instruktionssatz} der UMach virtuellen Maschine in den folgenden
Kategorien:
\index{Instruktionen!Kategorien}

\begin{enumerate}
  \item Kontrollinstruktionen,  die die Maschine in ihrer Gesamtheit
    steuern, wie z.B. den Betriebsmodus umschalten oder Ausschalten.
  \item Arithmetische Instruktionen, wie z.B. Addieren, Subtrahieren etc.
    Alle arithmetische Instruktionen operieren auf Register Inhalte.
  \item Logische Instruktionen, die eine logische Verknüpfung zwischen den Inhalten
    von Registern veranlassen.
  \item Vergleichsinstruktionen: Vergleichen von Registerinhalten.
  \item Sprünge im Programmcode.
  \item Load- und Store-Instruktionen die einzigen, die einen Zugriff auf den
    Speicher ausführen.
  \item Input-Output-Instruktionen operieren auf die I/O-Einheit der UMach VM.
  \item Andere Befehle: diese Kategorie enthält Instruktionen, die in keiner
    anderen Kategorie passen und zukünftige Erweiterungen.
\end{enumerate}

\paragraph{Verteilung des Befehlsraums}
\index{Befehlsraum}
Die oben angegebenen Instruktionskategorien unterteilen den \gls{Befehlsraum} in
8 Bereiche.
Es gibt $256$ Befehle, gemäß $2^{8} = 256$.


Im den folgenden Abschnitten werden die einzelnen Instruktionen beschrieben.
Zu jeder Intruktion wird der \glqq Mnemonic Code\grqq, der Maschinen Code, das
Instruktionsformat und Verwendungsbeispiele angegeben.



