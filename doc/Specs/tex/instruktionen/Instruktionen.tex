\section{Instruktionen}
\index{Instruktionen}

\paragraph{Verteilung des Befehlsraums}
\index{Befehlsraum}
Zur besseren Übersicht der verschiedenen UMach-\glspl{Instruktion}, unterteilen
wir den \gls{Instruktionssatz} der UMach virtuellen Maschine in den folgenden
Kategorien:
\index{Instruktionen!Kategorien}

\index{Befehlsraum!Verteilung}
\begin{enumerate}
  \item Kontrollinstruktionen,  die die Maschine in ihrer gesamten
    Funktionalität betreffen, wie z.B. den Betriebsmodus umschalten.
  \item Lade- und Speicherbefehle, die Register mit Werten aus dem Speicher, 
    anderen Registern oder direkten numerischen Angaben laden und die
    Registerinhalte in den Speicher schreiben.
  \item Arithmetische Instruktionen, die einfache arithmetische Operationen
    zwischen Registern veranlassen.
  \item Logische Instruktionen, die logische Verknüpfungen zwischen
    Registerinhalten oder Operationen auf Bit-Ebene in Registern anweisen.
  \item Vergleichsinstruktionen, die einen Vergleich zwischen
    Registerinhalten angeben.
  \item Sprunginstruktionen, die bedingt oder unbedingt sein können.
    Sie weisen die UMach Maschine an, die Programmausführung an einer anderen
    Stelle fortzufahren.
  \item Unterprogramm-Steuerung, bzw. Instruktionen, die die Ausführung von
    Unterprogrammen (Subroutinen) steuern.
  \item Systeminstruktionen, die die Unterstützung eines
    Betriebssystem ermöglichen.
\end{enumerate}

Die oben angegebenen Instruktionskategorien unterteilen den \gls{Befehlsraum} in
8 Bereiche. Es gibt $256$ mögliche Befehle, gemäß $2^{8} = 256$.
Die Verteilung der Kategorien auf die verschiedenen Maschinencode-Intervallen
wird in der Tabelle \ref{tab:Befehlraumverteilung} auf Seite
\pageref{tab:Befehlraumverteilung} angegeben.

\begin{table}
\index{Befehlsraum!Verteilungstabelle}
  \centering
  \begin{tabular}{|c|l|}                        \hline
    Maschinencodes   & Kategorie              \\\hline\hline
    \texttt{00 - 1F} & Kontrollbefehle        \\
    \texttt{20 - 3F} & Lade-/Speicherbefehle  \\
    \texttt{40 - 5F} & Arithmetische Befehle  \\
    \texttt{60 - 7F} & Logische Befehle       \\
    \texttt{80 - 9F} & Vergleichsbefehle      \\
    \texttt{A0 - BF} & Sprungbefehle          \\
    \texttt{C0 - DF} & Unterprogrambefehle    \\
    \texttt{E0 - FF} & Systembefehle          \\\hline
  \end{tabular}
  \caption[Verteilung des Befehlsraums]
          {Verteilung des Befehlsraums nach Befehlskategorien.
          Die Zahlen sind im Hexadezimalsystem angegeben.}
  \label{tab:Befehlraumverteilung}
\end{table}




\begin{table}%no, this is not normal human being LaTeX
\newcolumntype{C}[1]{>{\ttfamily\footnotesize\centering\let\newline\\\arraybackslash\hspace{0pt}}p{#1}}
\newcolumntype{L}{>{\ttfamily\footnotesize}l}
\newcommand{\nhex}[1]{\multirow{2}{*}{#1}}
\centering
\begin{tabular}{|L||*{8}{C{1.2cm}|}}                                             \hline
          &   0   &   1   &   2   &   3    & 4      & 5      & 6       & 7       \\\hline\hline
\nhex{0}  &  NOP  &       &       &        &        &        &         &         \\\cline{2-9}
          &       &       &       &        &        &        &         &         \\\hline
\nhex{1}  &       &       &       &        &        &        &         &         \\\cline{2-9}
          &       &       &       &        &        &        &         &         \\\cline{1-9}
\nhex{2}  &       &       &       &        &        &        &         &         \\\cline{2-9}
          &       &       &       &        &        &        &         &         \\\cline{1-9}
\nhex{3}  &       &       &       &        &        &        &         &         \\\cline{2-9}
          &       &       &       &        &        &        &         &         \\\cline{1-9}
\nhex{4}  &  ADD  & ADDI  & ADDU  & ADDUI  &        &        &         &         \\\cline{2-9}
          &       &       &       &        &        &        &         &         \\\cline{1-9}
\nhex{5}  &       &       &       &        &        &        &         &         \\\cline{2-9}
          &       &       &       &        &        &        &         &         \\\cline{1-9}
\nhex{6}  &       &       &       &        &        &        &         &         \\\cline{2-9}
          &       &       &       &        &        &        &         &         \\\cline{1-9}
\nhex{7}  &       &       &       &        &        &        &         &         \\\cline{2-9}
          &       &       &       &        &        &        &         &         \\\cline{1-9}
\nhex{8}  &       &       &       &        &        &        &         &         \\\cline{2-9}
          &       &       &       &        &        &        &         &         \\\cline{1-9}
\nhex{9}  &       &       &       &        &        &        &         &         \\\cline{2-9}
          &       &       &       &        &        &        &         &         \\\cline{1-9}
\nhex{A}  &       &       &       &        &        &        &         &         \\\cline{2-9}
          &       &       &       &        &        &        &         &         \\\cline{1-9}
\nhex{B}  &       &       &       &        &        &        &         &         \\\cline{2-9}
          &       &       &       &        &        &        &         &         \\\cline{1-9}
\nhex{C}  &       &       &       &        &        &        &         &         \\\cline{2-9}
          &       &       &       &        &        &        &         &         \\\cline{1-9}
\nhex{D}  &       &       &       &        &        &        &         &         \\\cline{2-9}
          &       &       &       &        &        &        &         &         \\\cline{1-9}
\nhex{E}  &       &       &       &        &        &        &         &         \\\cline{2-9}
          &       &       &       &        &        &        &         &         \\\cline{1-9}
\nhex{F}  &       &       &       &        &        &        &         &         \\\cline{2-9}
          &       &       &       &        &        &        &         &         \\\hline\hline
          &   8   &   9   &   A   &   B    &   C    &   D    &    E    &    F    \\\hline
\end{tabular}
\caption[Instruktionentabelle]
        {Instruktionentabelle}
\label{tab:Instruktionentabelle}
\end{table} 


Die Tabelle \ref{tab:Instruktionentabelle} auf der Seite 
\pageref{tab:Instruktionentabelle} enthält eine Übersicht aller Befehle und
deren Maschinencodes.
Diese Tabelle wird folgenderweise gelesen:
in der am weitesten linken Spalte wird die erste hexadezimale Ziffer eines
Befehls angegeben (ein Befehl ist zweistellig im Hexadezimalsystem).
Jede solche Ziffer hat rechts Zwei Zeilen, die von links nach rechts gelesen
werden: eine Zeile für die Ziffern von 0 bis 8, die anderen für die übrigen
Ziffern 9 bis F (im Hexadezimalsystem). Die Mnemonics der einzelnen Befehle sind
an der entsprechenden Stelle angegeben.

Im den folgenden Abschnitten werden die einzelnen Instruktionen beschrieben.
Zu jeder Instruktion wird der \glqq Mnemonic Code\grqq, der Maschinen Code, das
Instruktionsformat und Anwendungsbeispiele angegeben.

%%%%%%%%%%%% Jede Kategorie von Instruktionen in eigener Datei %%%%%%%%%%%%%
\section{Arithmetische Instruktionen} 

\opdef{ADD}{$X,Y,Z \in \Reg$}{0x50}{RRR}
Vorzeichen behaftete Addition der Registerinhalte $Y$ und $Z$.
Das Ergebnis der Addition wird in das Register $X$ gespeichert.
Entspricht dem algebraischen Ausdruck
\[
    X \gets Y + Z
\]
Beispiel:
\begin{lstlisting}
  SET   R1 1     # $R1 \gets 1$
  SET   R2 2     # $R2 \gets 2$
  ADD   R3 R1 R2 # $R3 \gets R1 + R2 = 1 + 2 = 3$
  #     X  Y  Z
  SET   R2 -2    # $R2 \gets -2$
  ADD   R3 R3 R2 # $R3 \gets R3 + R2 = 3 +(-2) = 1$
  ADD   R3 R4  5 # Fehler! 5 kein Register
\end{lstlisting}
Vorzeichenlose Addition wird durch den Befehl \texttt{ADDU} ausgeführt.


\opdef{ADDU}{$X,Y,Z \in \Reg$}{0x51}{RRR}
\glqq Add Unsigned\grqq.
Vorzeichenlose Addition der Register $Y$ und $Z$. Das Ergebnis wird in das
Register $X$ gespeichert. Enthält $Y$ oder $Z$ ein Vorzeichen (höchstwertiges
Bit auf 1 gesetzt), so wird es nicht als solches interpretiert, sondern als
Wertigkeit, die zum Betrag des Wertes hinzuaddiert wird ($+2^{31}$).

\begin{lstlisting}
  SET   R1 1     # $R1 \gets 1$
  SET   R2 -2    # $R2 \gets -2$
  ADDU  R3 R1 R2 # $R3 \gets (1 + 2 + 2^{31}) = 2147483651$
\end{lstlisting}



\opdef{ADDI}{$X,Y\in\Reg$, $N\in\mathds{Z}$}{0x52}{RRN}
\glqq Add Immediate\grqq.
Hinzuaddieren eines festen vorzeichenbehafteten ganzzahligen Wert $N$ zum Inhalt
des Registers $Y$ und speichern des Ergebnisses in das Register $X$.
Entspricht dem algebraischen Ausdruck
\[
  X \gets Y + N
\]
$N$ wird als vorzeichenbehaftete 8-Bit Zahl in Zweierkomplement-Darstellung
interpretiert und kann entsprechend Werte von $-128$ bis $127$ aufnehmen.

Beispiel:
\begin{lstlisting}
  SET   R1 1     # $R1 \gets 1$
  ADDI  R2 R1 2  # $R2 \gets R1 + 2 = 1 + 2 = 3$
  #     X  Y  N
  ADDI  R2 R2 -3 # $R2 \gets R2 + (-3) = 3 +(-2) = 1$
  ADDI  R2 R3 R4 # Fehler! R4 kein $n \in \mathds{Z}$
\end{lstlisting}



\opdef{ADDIU}{$X,Y \in \Reg$, $N\in\mathds{N}$}{0x53}{RRN}
\glqq Add Unsigned Immediate\grqq.
Vorzeichenlose Addition des ganzzahligen Wertes $N$ zum Inhalt des Registers $Y$
und speichern des Ergebnisses in das Register $X$.
Der Inhalt des Registers $Y$, die Zahl $N$ und das Ergebnis $Y + N$ werden als
vorzeichenlose Werte interpretiert.
Die Feste natürliche Zahl $N$ kann Werte aus dem Bereich $[0, 255]$ aufnehmen.
Wird eine größere Zahl angegeben, so wird sie modulo $256$ berechnet.



\opdef{SUB}{$X, Y, Z \in \Reg$}{0x58}{RRR}
Subtrahiert die Registerinhalte von $Y$ und $Z$ und speichert das Ergebnis in
das Register $X$. Entspricht dem Ausdruck
\[
    X \gets (Y - Z)
\]
Wobei $X$, $Y$ und $Z$ Register sind.



\opdef{SUBU}{$X, Y, Z \in \Reg$}{0x59}{RRR}
\glqq Subtract Unsigned\grqq.
Analog zur Instruktion \opref{SUB} mit dem Unterschied, dass alle Werte und
Operationen vorzeichenlos sind.


\opdef{SUBI}{$X, Y \in \Reg$, $N\in\mathds{Z}$}{0x5A}{RRN}
\glqq Subtract Immediate\grqq.
Funktioniert wie \opref{SUB} aber $N$ ist eine direkt angegebene Zahl
(kein Register).

\paragraph{Beispiel}
Folgendes Beispiel demonstriert die Verwendung von \opref{SUBI} und zeigt
zugleich einen Fehler.
\begin{lstlisting}
  SET  R1 50     # $R1 \gets 50$
  SUBI R2 R1 30  # $R2 \gets (R1 - 30) = 20$
  SUBI R2 R1 R1  # Fehler! da $R1 \not\in \mathds{Z}$
\end{lstlisting}


\opdef{SUBIU}{$X, Y \in \Reg$, $N\in\mathds{N}$}{0x5B}{RRN}
\glqq Subtract Immediate Unsigned\grqq.
Funktioniert wie die Instruktion \opref{SUBI} mit dem Unterschied, dass
\opref{SUBIU} ausschliesslich mit vorzeichenlosen Werten arbeitet. 



\opdef{MUL}{$X, Y \in \Reg$}{0x60}{RR0}
\glqq Multiply\grqq. 
Multipliziert die Inhalte der Register $X$ und $Y$ und speichert das Ergebnis in
die Spezialregister \texttt{HI} und \texttt{LO}. Diese zwei Spezialregister
werden als eine 64-Bit Einheit betrachtet, wobei jedes eine Hälfte des
64-Bit Ergebnisses enthält.
Dabei werden die höchstwertigen 32 Bit des Ergebnisses in das Register
\texttt{HI}\index{HI@\texttt{HI}}
und die 32 niedrigstwertigen Bits des Ergebnisses in das Register
\texttt{LO}\index{LO@\texttt{LO}} gespeichert.
Siehe auch die Tabelle \ref{tab:Spezialregister} auf der Seite
\pageref{tab:Spezialregister}.

Falls das Ergebnis der Multiplikation gänzlich in den 32 Bit des Registers
\texttt{LO} passt, wird das Register \texttt{HI} trotzdem auf Null gesetzt.

Äquivalenter algebraischer Ausdruck:
\[
    (HI, LO) \gets X \cdot Y
\]

\paragraph{Beispiel} Der folgende Code demonstriert die Verwendung der
\texttt{MUL} Instruktion.
\begin{lstlisting}
  SET  R1 4   # $R1 \gets 4$
  SET  R2 5   # $R2 \gets 5$
  MUL  R1 R2  # $HI \gets 0$
              # $LO \gets 20$

  SET  R1 0xAAAAAAAA
  MUL  R1 R1  # $R1^{2}$
              # HI = 0x71C71C70
              # LO = 0xE38E38E4 
  COPY R2 LO  # $R1^{2} \bmod 2^{32}$ 
\end{lstlisting}
Falls es bekannt ist, dass das Ergebnis der Multiplikation sich mit 32 Bit
darstellen lässt, kann man den Weg über die \texttt{LO} und \texttt{HI} Register
mit der Instruktion \opref{MULD} umgehen.




\opdef{MULU}{$X, Y\in \Reg$}{0x61}{RR0}
\glqq Multiply Unsigned\grqq.
Funktioniert wie die Instruktion \opref{MUL} mit dem Unterschied, dass die
Multiplikationoperanden $X$ und $Y$ vorzeichenlos behandelt werden. 



\opdef{MULI}{$X \in \Reg$, $N\in\mathds{Z}$}{0x62}{RNN}
\glqq Multiply Immediate\grqq.
Multipliziert den Inhalt des Registers $X$ mit der ganzen Zahl $N$ und speichert
das $64$-Bit Ergebnis in die Register \texttt{HI} und \texttt{LO}, die als ein
einziges $64$-Register betrachtet werden: \texttt{HI} enthält die ersten $32$
Bits (die höchstwertigen) und \texttt{LO} die letzten 32 Bits (die
niedrigstwertigen).
Siehe auch die Instruktion \opref{MUL}.


\opdef{MULIU}{$X \in \Reg$, $N\in\mathds{N}$}{0x63}{RNN}
\glqq Multiply Immediate Unsigned\grqq.
Funktioniert wie die Instruktion \opref{MULI} mit dem Unterschied, dass sowohl
die Operanden $X$ und $N$ als auch das Ergebnis vorzeichenlos sind.


\opdef{MULD}{$X, Y, Z \in \Reg$}{0x64}{RRR}
\glqq Multiply Direct\grqq.
Multipliziert die Inhalte der Register $Y$ und $Z$ und speichert die
niedrigstwertigen 32 Bits des Ergebnisses in das Register $X$. Anders wie bei
der Instruktion \opref{MUL}, werden die Register \texttt{HI} und \texttt{LO}
nicht verändert.

\texttt{MULD} entspricht der Instruktionen:
\begin{lstlisting}
  MUL  Y Z
  COPY X LO
\end{lstlisting}
wobei der alte Wert von \texttt{LO} erhalten bleibt.

Algebraisch geschrieben:
\[
    X \gets (Y \cdot Z) \bmod 2^{32}
\]



\opdef{DIV}{$X, Y \in \Reg$}{0x68}{RR0}
\glqq Divide\grqq, ganzzahlige Division.
Dividiert den Inhalt des Registers $X$ durch den Inhalt des Registers $Y$ und
speichert den Quotient in das Register \texttt{HI} und den Rest in das Register
\texttt{LO}.
Nach der Ausführung gilt
\[
    X = Y \cdot HI + LO
\]
Algebraisch ausgedrückt:
\begin{align*}
  HI & \gets \left\lfloor X/Y \right\rfloor \\
  LO & \gets X \bmod Y
\end{align*}
$\lfloor x \rfloor$ bedeutet in diesem Kontext, dass $x$ auf die betragsmässig
nächstkleinste ganze Zahl gerundet wird, oder die Nachkommastellen von $x$
werden abgeschnitten.

Um den Weg über die Register \texttt{HI} und \texttt{LO} zu umgehen, dafür aber
den Rest der Division zu verlieren, kann man die Instruktion \opref{DIVD}
verwenden.

\paragraph{Beispiel}
Der folgende Code demonstriert die Verwendung von \texttt{DIV}.
\begin{lstlisting}
  SET R1 10   # $R1 \gets 10$
  SET R2  3   # $R2 \gets 3$
  DIV R1 R2   # $HI \gets 3$
              # $LO \gets 1$
\end{lstlisting}


\opdef{DIVU}{$X, Y \in \Reg$}{0x69}{RR0}
\glqq Divide Unsigned\grqq.
Funktioniert wie \opref{DIV} mit dem Unterschied, dass ganzzahlige
vorzeichenlose Division durchgeführt werden. Die Ergebnis-Register \texttt{HI}
und \texttt{LO} enthalten entsprechend vorzeichenlose Werte.



\opdef{DIVI}{$X \in \Reg$, $N\in\mathds{Z}_{\setminus 0}$}{0x6A}{RNN}
\glqq Divide Immediate\grqq.
Dividiert den Inhalt des Registers $X$ durch die feste ganze Zahl $N$ und
speichert den Quotient in das Register \texttt{HI} und den Rest in das Register
\texttt{LO}.
$N$ nimmt Werte aus dem Intervall $[-2^{15}, 2^{15}-1] \setminus 0$.
Nach der Durchführung der Division gilt:
\[
    X = HI \cdot N + LO 
\]
\paragraph{Beispiel}
Der folgende Code demonstriert die Verwendung von \texttt{DIVI}.
\begin{lstlisting}
  SET  R1 10   # $R1 \gets 10$
  DIVI R1  3   # $HI \gets 3$
               # $LO \gets 1$
\end{lstlisting}



\opdef{DIVIU}{$X \in \Reg$, $N\in\mathds{N}_{\setminus 0}$}{0x6B}{RNN}
\glqq Divide Immediate Unsigned\grqq.
Analog zur Instruktion \opref{DIVI} mit dem Unterschied, dass $X$ und $N$
vorzeichenlose Werte haben. Insbesondere nimmt $N$ Werte aus dem Intervall
$[1, 2^{16}-1]$.



\opdef{DIVD}{$X, Y, Z \in \Reg$}{0x6C}{RRR}
\glqq Divide direct\grqq.
Dividiert ganzzahlig den Inhalt des Registers $Y$ durch den Inhalt des Registers
$Z$ und speichert den Quotient in das Register $X$. Die Spezialregister
\texttt{HI} und \texttt{LO} werden nicht verändert.
Entspricht den Instruktionen:
\begin{lstlisting}
  DIV  Y  Z
  COPY X HI
\end{lstlisting}
Algebraische Schreibweise:
\[
    X \gets \left\lfloor \frac{Y}{Z}  \right\rfloor
\]

\paragraph{Beispiel}
für die Verwendung der Instruktion \texttt{DIVD}:
\begin{lstlisting}
  SET  R1 10    # $R1 \gets 10$
  SET  R2  3    # $R2 \gets 3$
  DIVD R3 R1 R2 # $R3 \gets \lfloor 10/3 \rfloor = 3$
  MOD  R4 R1 R3 # $R4 \gets ( 10 \bmod 3 ) = 1$
\end{lstlisting}



\opdef{MOD}{$X, Y, Z \in \Reg$}{0x70}{RRR}
Modulo Operation.
Berechnet den Rest der Division $Y/Z$ und speichert den Rest in das Register
$X$.
Äquivalent zu den Instruktionen:
\begin{lstlisting}
  DIVU Y Z
  COPY X LO
\end{lstlisting}

Algebraische Schreibweise:
\[
    X \gets Y \bmod Z
\]
oder
\[
    X \gets \left(
      Y - \left\lfloor \frac{Y}{Z}  \right\rfloor \cdot Z
      \right)
\]


\opdef{MODI}{$X, Y \in \Reg$, $N \in \mathds{N}$}{0x72}{RRN}
\glqq Modulo Immediate\grqq. Analog zur Instruktion \opref{MOD}, berechnet
\texttt{MODI} den Rest der ganzzahligen Division $Y / N$ und speichert ihn in
das Register $X$. Der Unterschied liegt darin, dass $N$ eine fest angegebene
natürliche Zahl ist.
\[
    X \gets Y \bmod N
\]


\opdef{ABS}{$X, Y \in \Reg$}{0x78}{RR0}
\glqq Absolute\grqq.
Speichert den absoluten Wert des Registers $Y$ in das Register $X$.
Algebraisch ausgedrückt:
\[
    X \gets
    \begin{cases}
      Y            & \text{ falls } Y \geq 0 \\
      (-1) \cdot Y & \text{ falls } Y < 0
    \end{cases}
\]



\opdef{NEG}{$X, Y \in \Reg$}{0x80}{RR0}
\glqq Negate\grqq.
Wechselt das arithmetische Vorzeichen des Registers $Y$ und speichert das
Ergebnis in das Register $X$. Entspricht der Zweierkomplement Bildung.
Algebraische Schreibweise:
\[
    X \gets \big( (-1) \cdot Y \big)
\]
Um eine bitweise Inversion zu erreichen (Einerkomplement), siehe die
Instruktion \opref{NOT}.



\opdef{INC}{$X \in \Reg$}{0x81}{R00}
\glqq Increment\grqq.
Inkrementiert den Inhalt des Registers $X$.
\[
    X \gets ( X + 1 )
\]



\opdef{DEC}{$X \in \Reg$}{0x82}{R00}
\glqq Decrement\grqq.
Dekrementiert den Inhalt des Registers $X$.
\[
    X \gets ( X - 1 )
\]





