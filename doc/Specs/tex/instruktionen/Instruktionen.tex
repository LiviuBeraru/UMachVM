\section{Instruktionen}\index{Instruktionen}

Zur besseren Übersicht der verschiedenen UMach-\glspl{Instruktion}, unterteilen
wir den \gls{Instruktionssatz} der UMach virtuellen Maschine in den folgenden
Kategorien:
\index{Instruktionen!Kategorien}

\begin{enumerate}
  \item Kontrollinstruktionen,  die die Maschine in ihrer Gesamtheit
    steuern, wie z.B. den Betriebsmodus umschalten oder Ausschalten.
  \item Arithmetische Instruktionen, wie z.B. Addieren, Subtrahieren etc.
    Alle arithmetische Instruktionen operieren auf Register Inhalte.
  \item Logische Instruktionen, die eine logische Verknüpfung zwischen
    Registerinhalten veranlassen.
  \item Vergleichsinstruktionen, die einen arithmetischen Vergleich zwischen
    Registerinhalten angeben.
  \item Sprünge im Programmcode.
  \item Lade- und Speicherbefehle, die byte- oder wortweise Speicherinhalte
    aus dem Speicher in den Register lesen, oder aus einem Register in den
    Speicher schreiben.
  \item Eingabe- und Ausgabeinstruktionen operieren auf die I/O-Einheit der
    UMach VM.
  \item Andere Befehle: diese Kategorie wird für nicht-kategorisierbare
    Instruktionen reserviert.
\end{enumerate}

\paragraph{Verteilung des Befehlsraums}
\index{Befehlsraum}
Die oben angegebenen Instruktionskategorien unterteilen den \gls{Befehlsraum} in
8 Bereiche. Es gibt $256$ mögliche Befehle, gemäß $2^{8} = 256$.


Im den folgenden Abschnitten werden die einzelnen Instruktionen beschrieben.
Zu jeder Intruktion wird der \glqq Mnemonic Code\grqq, der Maschinen Code, das
Instruktionsformat und Verwendungsbeispiele angegeben.



