\section{Sprungbefehle}
\index{Sprungbefehle}
\label{sec:Sprungbefehle}

Alle Sprungbefehle, außer dem \opref{GO} Befehl, veranlassen einen relativen
Sprung im Programmcode -- relativ im Sinne, dass die Parameter der Sprungbefehle
einen ganzzahligen (negativen oder positiven) Versatz zur aktuellen
Programmadresse angeben. Dabei bedeutet der Versatz\index{Versatz}, oder
Offset\index{Offset}, die wievielte Instruktionen muss als nächste ausgeführt
werden. Zum Beispiel die Instruktion
\begin{lstlisting}
  JMP 2
\end{lstlisting}
bedeutet
\glqq Die zweite Instruktion nach dieser soll ausgeführt werden\grqq. Also es
wird eine Instruktion übersprungen, und mit der zweiten fortgefahren.
Die Instruktion
\begin{lstlisting}
  BE -5
\end{lstlisting}
bedeutet \glqq Falls das Register \texttt{CMPR} den Wert $0$ hat, überspringe
$4$ Instruktionen zurück (da negativ) und fahre mit der 5. fort\grqq, also die
5. Instruktion zurück. 

In dem Programm
\begin{lstlisting}
 SET R1 1
 SET R2 2
 JMP -2
\end{lstlisting}
bewirkt der \opref{JMP}-Befehl einen Sprung an die erste Instruktion zurück,
also an die zweite zurück.

Der Sprungversatz (Offset) wird stets nicht in Bytes angegeben, sondern in
Instruktionen -- wobei eine Instruktion der UMach Maschine 4 Bytes beträgt. Die
Maschine multipliziert diese Anzahl mit 4, um die richtige Programmadresse zu
berechnen.

Die Sprungbefehle bauen auf die Vergleichsbefehle auf (Abschnitt
\ref{sec:Vergleichsinstruktionen}): jeder Sprungbefehl (außer \opref{JMP}
und \opref{GO}) untersuchen das Spezialregister \texttt{CMPR} und verzweigen die
Programmausführung anhand seines Wertes.

\paragraph{Bemerkung}
Es ist zu bemerken, dass die hier verwendeten nummerischen Versatzwerte sich
auf die interne Funktionsweise der UMach Maschine beziehen. Ein Assembler
könnte solche nummerische Angaben für ungültig erklären und statt dessen,
textuelle Sprungmarken erwarten (Labels).


\opdef{BE}{$N \in \mathds{Z}$}{0x80}{NNN}
\glqq Branch Equal\grqq.
Wenn das Register \texttt{CMP} den Wert $0$ hat, wird zur $N$-ten Instruktion
vorwärts oder rückwärts gesprungen. Ein negatives $N$ bedeutet einen Sprung
rückwärts, ein positives $N$ bewirkt einen Sprung vorwärts. Der Sprung wird
dadurch erreicht, dass das Register \texttt{PC} gemäß der folgenden Formel
modifiziert wird:
\begin{align*}
   & PC \gets \$PC + 4 \cdot N \\
   & PC \gets \$PC - 4
\end{align*}

Die Multiplikation mit $4$ wird deshalbt ausgeführt, weil ein Befehl immer aus 4
Bytes besteht, sodass der Adressoffset zwischen zwei Befehlen immer 4 ist.

Die Subtraktion von $4$ wird deshalbt ausgeführt, weil die Maschine nach jeder
Instruktion das Register $PC$ automatisch um $4$ inkrementiert, so dass eine
Korrektur gebraucht wird.

\paragraph{Fehler}
Falls das Sprungziel eine Speicheradresse ist, die außerhalb des
Speicherbereichs liegt, wir der Interrupt mit Nummer 16 generiert (ungültige
Speicheradresse) und das Bit mit Nummer 8 im Register \texttt{ERR} gesetzt
(siehe auch Tabelle \ref{tab:ERR-register} auf Seite
\pageref{tab:ERR-register}).


\paragraph{Beispiel}
Der folgende Code lädt zwei Bytes in die Register $R2$ und $R3$ und addiert
diese arithmetisch, falls sie ungleiche Werte haben. Sind die Werte gleich,
wird stattdessen der Inhalt von $R2$ mit 2 multipliziert. Ein möglicher
arithmetischer Überlauf wird nicht berücksichtigt.
\begin{lstlisting}
  SET   R1 100    # R1 = 100
  LB    R2 R1     # Lade Byte von Adresse 100 nach R2
  INC   R1        # R1++, R1 = 101
  LB    R3 R1     # Lade Byte von Adresse 101 nach R3
  CMP   R2 R3     # Vergleiche Inhalt von R2 und R3
                  # Ergebnis geht ins CMPR
  #BE    equal    # Asm Schreibweise
  BE    5         # Wenn CMPR gleich 0 ist, springe
                  # zur 5. Instruktion (MULI)
                  # (gehe zum label equal)
  ADD   R2 R2 R3  # Addiere Inhalt von R2 und R3
  DEC   R1        # R1--, R1 = 100
  SB    R2 R1     # Speichere R2 nach Adresse 100
  #JMP   finish   # Asm Schreibweise
  JMP   3         # Gehe zur 3. Instruktion (label finish)

equal:
  MULI  R2 2      # Multipliziere Inhalt von R2 mit 2
  SB    LO R1     # Speichere Inhalt von LO nach Adresse in R1
finish:
  EOP
\end{lstlisting}



\opdef{BNE}{$N \in \mathds{Z}$}{0x81}{NNN}
\glqq Branch Not Equal \grqq.
Entspricht dem Verhalten von \opref{BE} mit dem Unterschied, dass der angegebene
Sprung ausgeführt wird, wenn \texttt{CMPR} nicht $0$ ist.

\paragraph{Fehler}
Wie bei der Instruktion \opref{BE}.



\opdef{BL}{$N \in \mathds{Z}$}{0x82}{NNN}
\glqq Branch Less\grqq. 
Springt zur $N$-ten Instruktion, wenn der Inhalt von \texttt{CMPR} kleiner $0$
ist.

\paragraph{Fehler}
Wie bei der Instruktion \opref{BE}.



\opdef{BLE}{$N \in \mathds{Z}$}{0x83}{NNN}
\glqq Branch Less Equal\grqq.
Springt zur $N$-ten Instruktion, wenn der Inhalt von \texttt{CMPR} kleiner
oder gleich $0$ ist.

\paragraph{Fehler}
Wie bei der Instruktion \opref{BE}.


\opdef{BG}{$N \in \mathds{Z}$}{0x84}{NNN}
\glqq Branch Greater\grqq.
Springt zur $N$-ten Instruktion, wenn der Inhalt von \texttt{CMPR} größer als
$0$ ist.

\paragraph{Fehler}
Wie bei der Instruktion \opref{BE}.



\opdef{BGE}{$N \in \mathds{Z}$}{0x85}{NNN}
\glqq Branch Greater Equal\grqq. 
Springt zur $N$-ten Instruktion, wenn der Inhalt von \texttt{CMPR} größer oder
gleich $0$ ist.

\paragraph{Fehler}
Wie bei der Instruktion \opref{BE}.



\opdef{JMP}{$N \in \mathds{N}$}{0x88}{NNN}
\glqq Jump\grqq.
Springt zur $N$-ten Instruktion, unabhängig vom Wert des Registers $CMPR$.

\paragraph{Fehler}
Wie bei der Instruktion \opref{BE}.


