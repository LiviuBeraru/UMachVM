\section{Übersprungsbefehle}
\label{sec:Sprungbefehle}

Alle Sprungbefehle, außer dem \opref{GO} Befehl, veranlassen einen relativen
Übersprung im Programmcode -- relativ im Sinne, dass die Parameter der
Übersprungbefehle einen ganzzahligen Versatz zur aktuellen Programmadresse
angeben. Dabei bedeutet der Versatz\index{Versatz}, wieviele Instruktionen
müssen bis zur Zielinstruktion übersprungen werden. Z.B. die Instruktion
\begin{lstlisting}
  JMP 2
\end{lstlisting}
bedeutet
\glqq Überspringe die nächsten 2 Instruktionen und fahre mit der 3. fort\grqq.
Die Instruktion
\begin{lstlisting}
  BE 5
\end{lstlisting}
bedeutet \glqq Falls das Register \texttt{CMPR} den Wert $0$ hat, überspringe
$5$ Instruktionen und fahre mit der 6. fort\grqq. Es wird also nicht die
wievielte Instruktion angegeben, sondern es wird die Anzahl an Instruktionen
angegeben, die übersprungen werden sollten (daher auch \glqq
Übersprungbefehl\grqq, statt \glqq Sprungbefehl\grqq).

Der Übersprungversatz wird stets nicht in Bytes angegeben, sondern in
Instruktionen -- wobei eine Instruktion der UMach Maschine 4 Bytes beträgt. Die
Maschine multipliziert diese Anzahl mit 4 um die richtige Programmadresse zu
berechnen.

Die Übersprungbefehle bauen auf die Vergleichsbefehle auf (Abschnitt
\ref{sec:Vergleichsinstruktionen}): jeder Übersprungbefehl (außer \opref{JMP}
und \opref{GO}) untersuchen das Spezialregister \texttt{CMPR} und verzweigen die
Programmausführung anhand seines Wertes.



\opdef{BE}{$N \in \mathds{Z}$}{0x80}{NNN}
\glqq Branch Equal\grqq.
Wenn das Register \texttt{CMP} den Wert $0$ hat, wird über $N$ Instruktionen
vorwärts oder rückwärts gesprungen. Ein negatives $N$ bedeutet einen Sprung
rückwärts, ein positives $N$ bewirkt einen Sprung vorwärts. Der Sprung wird
dadurch erreicht, dass das Register \texttt{PC} gemäß der folgenden Formel
modifiziert wird:
\[
    PC \gets \$PC + 4 \cdot N
\]
Die Multiplikation mit $4$ wird deshalbt ausgeführt, weil ein Befehl immer aus 4
Bytes besteht, sodass der Adressoffset zwischen zwei Befehlen immer 4 ist. Somit
ist $N$ die Anzahl der zu überspringenden Befehle bis zur nächsten Instruktion.
Der dazu benötigte Offset $N$ wird vom Assembler automatisch berechnet. 

\paragraph{Bemerkung}
Die UMach Maschine erhöht den Programmcounter (Register \texttt{PC}) nach
der Ausführung jeder Instruktion
(siehe \ref{subsec:Neumann-Zyklus}, Seite
\pageref{subsec:Neumann-Zyklus}).
Die Modifizierung des \texttt{PC}-Registers durch den \opref{BE} Befehl wirkt
sich nicht störend auf die automatische Inkrementierung des Programmcounters.

\paragraph{Fehler}
Falls das Sprungziel eine Speicheradresse ist, die außerhalb des
Speicherbereichs liegt, wir der Interrupt mit Nummer 16 generiert (ungültige
Speicheradresse) und das Bit mit Nummer 8 im Register \texttt{ERR} gesetzt
(siehe auch Tabelle \ref{tab:ERR-register} auf Seite
\pageref{tab:ERR-register}).


\paragraph{Beispiel}
Der folgende Code lädt zwei Bytes in die Register $R2$ und $R3$ und addiert
diese arithmetisch, falls sie ungleiche Werte haben. Sind die Werte gleich,
wird stattdessen der Inhalt von $R2$ mit 2 multipliziert. Ein möglicher
arithmetischer Überlauf wird nicht berücksichtigt.
\begin{lstlisting}
  SET   R1 100    # R1 = 100
  LB    R2 R1     # Lade Byte von Adresse 100 nach R2
  INC   R1        # R1++, R1 = 101
  LB    R3 R1     # Lade Byte von Adresse 101 nach R3
  CMP   R2 R3     # Vergleiche Inhalt von R2 und R3
                  # Ergebnis geht ins CMPR
  #BE    equal    # Asm Schreibweise
  BE    3         # Wenn CMPR gleich 0 ist, überspringe 
                  # die folgenden 3 Instruktionen
                  # (gehe zum label equal)
  ADD   R2 R2 R3  # Addiere Inhalt von R2 und R3
  DEC   R1        # R1--, R1 = 100
  SB    R2 R1     # Speichere R2 nach Adresse 100
  #JMP   finish   # Asm Schreibweise
  JMP   2         # Überspringe die folgenden 2 Befehle
                  # (gehe zu finish)
equal:
  MULI  R2 2      # Multipliziere Inhalt von R2 mit 2
  SB    LO R1     # Speichere Inhalt von LO nach Adresse in R1
finish:
  EOP
\end{lstlisting}



\opdef{BNE}{$N \in \mathds{Z}$}{0x81}{NNN}
\glqq Branch Not Equal \grqq.
Entspricht dem Verhalten von \opref{BE} mit dem Unterschied, dass der angegebene
Übersprung ausgeführt wird, wenn \texttt{CMPR} nicht $0$ ist.

\paragraph{Fehler}
Wie bei der Instruktion \opref{BE}.



\opdef{BL}{$N \in \mathds{Z}$}{0x82}{NNN}
\glqq Branch Less\grqq. Überspringt $N$ Instruktionen, wenn der Inhalt von
\texttt{CMPR} kleiner $0$ ist.

\paragraph{Fehler}
Wie bei der Instruktion \opref{BE}.



\opdef{BLE}{$N \in \mathds{Z}$}{0x83}{NNN}
\glqq Branch Less Equal\grqq.
Überspringt $N$ Instruktionen, wenn der Inhalt von \texttt{CMPR} kleiner oder
gleich $0$ ist.

\paragraph{Fehler}
Wie bei der Instruktion \opref{BE}.


\opdef{BG}{$N \in \mathds{Z}$}{0x84}{NNN}
\glqq Branch Greater\grqq.
Überspringt $N$ Instruktionen, wenn der Inhalt von \texttt{CMPR} größer als
$0$ ist.

\paragraph{Fehler}
Wie bei der Instruktion \opref{BE}.



\opdef{BGE}{$N \in \mathds{Z}$}{0x85}{NNN}
\glqq Branch Greater Equal\grqq. 
Überspringt $N$ Instruktionen, wenn der Inhalt von \texttt{CMPR} größer oder
gleich $0$ ist.

\paragraph{Fehler}
Wie bei der Instruktion \opref{BE}.



\opdef{JMP}{$N \in \mathds{N}$}{0x88}{NNN}
\glqq Jump\grqq.
Überspringt $N$ Instruktionen.

\paragraph{Fehler}
Wie bei der Instruktion \opref{BE}.


