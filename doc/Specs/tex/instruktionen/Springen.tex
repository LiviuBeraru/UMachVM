\section{Übersprungsbefehle}
\label{sec:Sprungbefehle}

Alle Sprungbefehle, außer dem \opref{GO} Befehl, veranlassen einen relativen
Übersprung im Programmcode -- relativ im Sinne, dass die Parameter der
Übersprungbefehle einen ganzzahligen Versatz zur aktuellen Programmadresse
angeben. Dabei bedeutet der Versatz\index{Versatz}, wieviele Instruktionen
müssen bis zur Zielinstruktion übersprungen werden. Z.B. die Instruktion
\begin{lstlisting}
  JMP 2
\end{lstlisting}
bedeutet
\glqq Überspringe 2 Instruktionen und fahre mit der 3. fort\grqq.
Die Instruktion
\begin{lstlisting}
  BE 5
\end{lstlisting}
bedeutet
\glqq Falls das Register \texttt{CMP} den Wert $0$ hat, überspringe $5$
Instruktionen und fahre mit der 6. fort\grqq.

Der Versatz wird nicht in Bytes angegeben, sondern in
Instruktionen -- wobei eine Instruktion der UMach Maschine 4 Bytes beträgt.

Die Übersprungbefehle bauen auf die Vergleichsbefehle auf: jeder
Übersprungbefehl (außer \opref{JMP} und \opref{GO}) untersuchen das
Spezialregister \texttt{CMP} und verzweigen die Programmausführung anhand seines
Wertes.


\opdef{BE}{$N \in \mathds{Z}$}{0xD0}{NNN}
\glqq Branch Equal\grqq.
Wenn das Register \texttt{CMP} den Wert $0$ hat, wird über $N$ Instruktionen
vorwärts oder rückwärts gesprungen. Ein negatives $N$ bedeutet einen Sprung
rückwärts, ein positives $N$ bewirkt einen Sprung vorwärts. Der Sprung wird
dadurch erreicht, dass das Register \texttt{PC} gemäß der folgenden Formel
modifiziert wird:
\[
    PC \gets PC + 4 \cdot N
\]
Die Multiplikation mit $4$ wird deshalbt ausgeführt, weil ein Befehl immer aus 4
Bytes besteht, sodass der Adressoffset zwischen zwei Befehlen immer 4 ist. Somit
ist $N$ die Anzahl der zu überspringenden Befehle bis zur nächsten Instruktion.
Der dazu benötigte Offset $N$ wird vom Assembler automatisch berechnet. 

\paragraph{Bemerkung}
Die UMach Maschine erhöht den Programmcounter (Register \texttt{PC}) nach
der Ausführung jeder Instruktion
(siehe \ref{subsec:Neumann-Zyklus}, Seite
\pageref{subsec:Neumann-Zyklus}).
Die Modifizierung des \texttt{PC}-Registers durch den \opref{BE} Befehl wirkt
sich nicht störend auf die automatische Inkrementierung des Programmcounters.


\paragraph{Beispiel}
Der folgende Code lädt zwei Bytes in die Register $R2$ und $R3$ und addiert
diese arithmetisch, falls sie ungleiche Werte haben. Sind die Werte gleich,
wird stattdessen der Inhalt von $R2$ mit 2 multipliziert. Ein mögliches
Überlaufen wird nicht berücksichtigt.
\begin{lstlisting}
  SET   R1 100    # $R1 \gets 100$
  LBUI  R2 R1  0  # Lade Byte von Adresse 100 nach R2
  LBUI  R3 R1  1  # Lade Byte von Adresse 100+1 nach R3
  CMPU  R2 R3     # Vergleiche Inhalt von R2 und R3
                  # Ergebnis geht ins CMP
  #BE    equal    # Asm Schreibweise
  BE    3         # Wenn CMP gleich 0 ist, überspringe 
                  # die folgenden 3 Instruktionen
                  # (gehe zum label equal)
                  # N ist in diesem Fall 3
  ADDU  R2 R2 R3  # Addiere Inhalt von R2 und R3
  SBUI  R2 R1 0   # Speichere R2 nach Adresse 100
  #JMP   finish   # Asm Schreibweise
  JMP   2         # Überspringe die folgenden 2 Befehle, 
                  # N von JMP ist 2.
equal:
  MULIU R2 2      # Multipliziere Inhalt von R2 mit 2
  SBUI  LO R1 0   # Speichere Inhalt von LO nach Adresse in R1
finish:
\end{lstlisting}



\opdef{BNE}{$N \in \mathds{Z}$}{0xD1}{NNN}
\glqq Branch Not Equal \grqq.
Entspricht dem Verhalten von \opref{BE} mit dem Unterschied, dass der angegebene
Übersprung ausgeführt wird, wenn \texttt{CMP} nicht $0$ ist.


\opdef{BL}{$N \in \mathds{Z}$}{0xD2}{NNN}
\glqq Branch Less\grqq. 
Überspringt $N$ Instruktionen, wenn der Inhalt von \texttt{CMP} kleiner $0$ ist.


\opdef{BLE}{$N \in \mathds{Z}$}{0xD3}{NNN}
\glqq Branch Less Equal\grqq.
Überspringt $N$ Instruktionen, wenn der Inhalt von \texttt{CMP} kleiner oder
gleich $0$ ist.


\opdef{BG}{$N \in \mathds{Z}$}{0xD4}{NNN}
\glqq Branch Greater\grqq.
Überspringt $N$ Instruktionen, wenn der Inhalt von \texttt{CMP} größer als
$0$ ist.


\opdef{BGE}{$N \in \mathds{Z}$}{0xD5}{NNN}
\glqq Branch Greater Equal\grqq. 
Überspringt $N$ Instruktionen, wenn der Inhalt von \texttt{CMP} größer oder
gleich $0$ ist.


\opdef{JMP}{$N \in \mathds{N}$}{0xD8}{NNN}
\glqq Jump\grqq.
Überspringt $N$ Instruktionen.

\opdef{JMPR}{$X \in \Reg$}{0xD9}{R00}
\glqq Jump Register\grqq.
Überspringt $X$ Instruktionen. $X$ ist ein Register.


\opdef{GO}{$N \in \mathds{N}$}{0xDF}{NNN}
Setzt $PC$ auf die angegebene absolute Adresse. Hierbei ist zu beachten,  dass
nicht in die Mitte eines Befehles gesprungen wird. Dies zu gewährleisten liegt
in der Verantwortung des Programmierers.

