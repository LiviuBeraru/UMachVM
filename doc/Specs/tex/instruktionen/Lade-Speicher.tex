\section{Lade- und Speicherbefehle}
\label{sec:Lade-Speicher-Instruktionen}

\opdef{SET}{$X\in\Reg$, $N\in\mathds{Z}$}{0x10}{RNN}
Setzt den Inhalt des Registers $X$ auf den ganzzahligen Wert $N$.
Da $N$ mit 16 Bit und im Zweierkomplement dargestellt wird, kann $N$ Werte von
$-2^{15}$ bis $2^{15} - 1$ aufnehmen, bzw. von $-32768$ bis $+32767$.
Werte außerhalb dieses Intervalls werden auf Assembler-Ebene entsprechend
gekürzt (es wird modulo berechnet, bzw. nur die ersten 16 Bits aufgenommen).

Beispiele:
\begin{lstlisting}
label:
  SET R1  8    # $R1 \gets 8$
  SET R2 -3    # $R2 \gets -3$
  SET R3 65536 # $R3 \gets 0$, da $65536 = 2^{16} \equiv 0 \bmod 2^{16}$
  SET R4 70000 # $R3 \gets 4464 = 70000 \bmod 2^{16}$
  SET R7 label # Adresse 'label' ins R7
\end{lstlisting}


\opdef{CP}{$X, Y \in \Reg$}{0x18}{RR0}
Kopiert den Inhalt des Registers $Y$ in das Register $X$. Register $Y$ wird
dabei nicht geändert. Entspricht
\[
    X \gets Y
\]

Beispiel:
\begin{lstlisting}
 SET  R1 5  # $R1 \gets 5$
 CP   R2 R1 # $R2 \gets 5$
\end{lstlisting}



\opdef{LB}{$X, Y \in \Reg$}{0x20}{RR0}
Lade ein Byte aus dem Speicher mit Adresse $Y$ in das niedrigstwertige
Byte des Registers $X$. Die anderen Bytes von $X$ werden von diesem Befehl nicht
betroffen. Insbesondere, werden sie nicht auf Null gesetzt.

Äquivalenter C Code:
\begin{lstlisting}
  x = (x & 0xFFFFFF00) | (mem(y) & 0x00FF);
\end{lstlisting}

\paragraph{Beispiel}
Angenommen, der Speicher an den Adressen $100$ und $101$ hat den Wert $5$,
bzw. $6$.
\begin{lstlisting}
  SET  R1   100   # Basisadresse R1 = 100
  SET  R2     0   # Index R2 = 0
  SET  R3     0
  LB   R3 R1 R2   # R3 = 5 ($mem(100+0)$)
  SHLI R3 R3  8   # shift left 8 Bit, R3 = 1280
  INC  R2         # R2++, R2 = 1
  LB   R3 R1 R2   # R3 = 1286 (R3 + $mem(100+1)$)
\end{lstlisting}
Hier werden zwei nacheinander folgenden Bytes aus dem Speicher gelesen und in
die zwei niedrigstwertigen Bytes von $R3$ abgelegt.



\opdef{LW}{$X, Y \in \Reg$}{0x24}{RR0}
\glqq Load Word\grqq.
Lade ein Wort (4 Byte) aus dem Speicher mit Adresse $Y$ in das Register $X$.
Alle Bytes von $X$ werden dabei überschrieben.
Die Bytes aus dem Speicher werden nacheinander gelesen. Es werden also die
Bytes mit Adressen $Y + 0$, $Y + 1$, $Y + 2$ und $Y + 3$ zu
einem 4-Byte Wort zusammengesetzt und so in $X$ ablegt.

\paragraph{Beispiel}
Abgenommen, die Adressen von 100 bis 103 sind mit den Werten 0, 1, 2 und 3
belegt und bilden somit den Wert $66051$.
\begin{lstlisting}
  SET R1   100
  SET R2     0
  LW  R3 R1 R2  # $R3 \gets mem_{4}(R1 + R2) = 66051$
  LW  R3 R1  8  # Fehler! 8 ist kein Register nutze LWI dafuer
\end{lstlisting}




\opdef{SB}{$X, Y \in \Reg$}{0x30}{RR0}
\glqq Store Byte\grqq.
Speichert den Inhalt des niedrigstwertigen Byte von $X$ an der Speicherstelle
$Y$. $X$ und $Y$ sind dabei Register.

Entspricht dem algebraischen Ausdruck
\[
    X \to mem_{1} (Y)
\]
$mem_{1}(x)$ bedeutet dabei 1 Byte an der Adresse $x$.

\paragraph{Beispiel}
\begin{lstlisting}
  SET R1 128     # R1 = Speicheradresse 128
  SET R2 513     # R2 = 0x0201
  SB  R2 R1 ZERO # Speicher mit Adresse 128 wird auf 1 gesetzt
\end{lstlisting}
\texttt{ZERO} ist dabei ein Spezialregister mit konstantem Wert $0$.




\opdef{SW}{$X, Y\in \Reg$}{0x34}{RR0}
\glqq Store Word\grqq.
Speichert den Inhalt aller Bytes in $X$ an die Speicheradressen $Y$ bis 
$X + 3$.
\[
    X \to mem_{4}(Y)
\]
\paragraph{Beispiel}
Es wird das Register $R2$ mit dem Wert \texttt{0x01020304} geladen und an die
Adresse $128$ gespeichert. Dabei werden die Byte-Werten in 
\glqq big-endian\grqq\ Reihenfolge gespeichert: das höchstwertige Byte aus $R2$
(\texttt{0x01}) wird an der Adresse $128$ gespeichert, das niedrigstwertige
Byte (\texttt{0x04}) an die Adresse $131$.

\begin{lstlisting}
  SET R1 128         # R1 = Speicheradresse 128
  SET R2 0x01020304  # Wert zum Speichern
  SH  R2 R1 ZERO     # mem[128] = 0x01
                     # mem[129] = 0x02
                     # mem[130] = 0x03
                     # mem[131] = 0x04
\end{lstlisting}




\opdef{PUSH}{$X \in \Reg$}{0x42}{R00}
\glqq Push Word\grqq.
Erniedrigt das Register \texttt{SP} um 4 und kopiert das ganze Register $X$ auf
den Stack, wobei der \glqq Stack\grqq\ ist der Speicherbereich mit
Anfangsadresse in \texttt{SP}.
Die Byte-Reihenfolge der Lese- und Schreiboperationen ist \glqq Big-Endian\grqq\
und wird in der nachfolgenden Tabelle dargestellt:
\begin{center}
\begin{tabular}{l|cccc}
  \toprule
  $X$  Wertigkeiten &
  $2^{31} \leftrightarrow 2^{24}$ &
  $2^{23} \leftrightarrow 2^{16}$ &
  $2^{15} \leftrightarrow 2^{8}$  &
  $2^{7}  \leftrightarrow 2^{0}$ 
  \\
  &
  $\downarrow$ & $\downarrow$ & $\downarrow$ & $\downarrow$ 
  \\
  \text{Stack-Bereich} &
  \texttt{mem[SP + 0]} &
  \texttt{mem[SP + 1]} &
  \texttt{mem[SP + 2]} &
  \texttt{mem[SP + 3]}
  \\\bottomrule
\end{tabular}
\end{center}


Entspricht
\begin{align*}
  SP & \gets SP - 4    \\
  X  & \to mem_{4}(SP)
\end{align*}

\paragraph{Beispiel}
Der folgende Code speichert das 4-Byte Wort \texttt{0x01020304} auf den Stack.
Die Stack-Struktur wird in Kommentaren gezeigt.
\begin{lstlisting}
  SET  R1 0x01020304 # Wert zum pushen
  PUSH R1            # mem[SP + 0] = 0x01
                     # mem[SP + 1] = 0x02
                     # mem[SP + 2] = 0x03
                     # mem[SP + 3] = 0x04
\end{lstlisting}




\opdef{POP}{$X \in \Reg$}{0x4A}{R00}
\glqq Pop Word\grqq.
Speichert 4 Bytes ab der Adresse \texttt{SP} in das Register $X$ und erhöht
\texttt{SP} um 4.
Die Byte-Reihenfolge der Lese- und Schreiboperationen ist \glqq Big-Endian\grqq\
und wird in der nachfolgenden Tabelle dargestellt.

\begin{center}
\begin{tabular}{l|cccc}
  \toprule
  $X$  Wertigkeiten &
  $2^{31} \leftrightarrow 2^{24}$ &
  $2^{23} \leftrightarrow 2^{16}$ &
  $2^{15} \leftrightarrow 2^{8}$  &
  $2^{7}  \leftrightarrow 2^{0}$ 
  \\
  &
  $\uparrow$ & $\uparrow$ & $\uparrow$ & $\uparrow$ 
  \\
  \text{Stack-Bereich} &
  \texttt{mem[SP + 0]} &
  \texttt{mem[SP + 1]} &
  \texttt{mem[SP + 2]} &
  \texttt{mem[SP + 3]}
  \\\bottomrule
\end{tabular}
\end{center}

Diese Instruktion kann algebraisch so ausgedrückt werden:
\begin{align*}
  X  & \gets mem_{4}(SP) \\
  SP & \gets SP + 4
\end{align*}
Die Instruktion \texttt{POP} ist äquivalent zu den folgenden Instruktionen:
\begin{lstlisting}
  LWI   X SP 0
  ADDI SP SP 4
\end{lstlisting}


\paragraph{Beispiel}
Angenommen, die ersten 4 Bytes vom Stack-Bereiche sind
\texttt{0xAA 0xBB 0xCC 0xDD}.
\begin{lstlisting}
  POPH R1      # R1 = 0xAABBCCDD
\end{lstlisting}

