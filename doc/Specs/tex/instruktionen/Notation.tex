\subsection{Notationen}
\index{Notation}


Mit \Reg\index{\Reg} wird die Menge aller Register
gekennzeichnet\footnote{Nicht verwechseln mit den Symbolen $\mathds{R}$ und
$\mathbb{R}$, die die Menge aller reellen Zahlen bedeuten.}.
Die Notation $X \in \Reg$ bedeutet, dass $X$ ein Element aus dieser Menge ist,
mit anderen Worten, dass $X$ ein Register ist. Analog bedeutet die
Schreibweise $X,Y \in \Reg$, dass $X$ und $Y$ beide Register sind.


Wenn $X$ ein Register bezeichnet, dann bezeichnet $\$X$ den Inhalt des
Registers $X$. Diese Notation wird verwendet, um in Aussagen wie 
\glqq $\$X + \$Y$\grqq\ klar zu machen, dass es sich um die Registerinhalte von
$X$ und $Y$ handelt, nicht um die Register selbst.

Die Tabelle \ref{tab:Notationen} auf der Seite \pageref{tab:Notationen}
wiedergibt alle Notationen, die in diesem Kapitel verwendet werden.

\begin{longtable}{ll}
 \caption{Verwendete Notationen}
 \label{tab:Notationen}
 \\\toprule
 Notation & Bedeutung \\\midrule
 \endfirsthead
 \\\toprule
 Notation & Bedeutung \\\midrule
 \endhead
 $\mathds{N}$ & Menge aller natürlichen Zahlen: $0,1,2,\ldots$         \\
 $\mathds{N}_{\setminus 0}$ 
              & $\mathds{N}$ ohne die Null: $1,2,\ldots$               \\
 $\mathds{Z}$ & Menge aller ganzen Zahlen: $\ldots,-2,-1,0,1,2,\ldots$ \\
 $\mathds{Z}_{\setminus 0}$ 
              & $\mathds{Z}$ ohne die Null: $\ldots,-2,-1,1,2,\ldots$  \\
 $N \in \mathds{N}$
              & $N$ ist Element von $\mathds{N}$, oder liegt im Bereich von
              $\mathds{N}$                                             \\
 $\Reg$       & Menge aller Register                                   \\
 $X \in \Reg$ & $X$ ist ein Register                                   \\
 $\$X$        & Inhalt des Registers $X$                               \\
 $a \gets b$  & $a$ wird auf $b$ gesetzt                               \\
 $X \gets \$Y$& $X$ wird auf den Inhalt des Registers $Y$ gesetzt      \\
 $mem(X)$     & Speicherinhalt an der Adresse $X$ (1 Byte)             \\
 $mem_{n}(X)$ & $n$-Bytes-Block im Speicher ab Adresse $X$             \\
 \texttt{mem[n]}
              & äquivalent zu $mem(n)$                                 \\
 $lsb_{n}(X)$ \index{las@$lsb$}
              & \glqq least significant $n$ bits\grqq\ von $X$        \\
 \bottomrule
\end{longtable}

