\section{Logische Instruktionen}

\opdef{AND}{$X, Y, Z \in \Reg$}{0x50}{RRR}
Berechnet bitweise $\$Y \land \$Z$ (und-Verknüpfung) und speichert das Ergebnis
in das Register $X$.
\[
    X \gets (\$Y \land \$Z)
\]


\opdef{ANDI}{$X, Y \in \Reg$, $N \in \mathds{N}$}{0x51}{RRN}
\glqq And Immediate\grqq. Berechnet bitweise $\$Y \land N$ (und-Verknüpfung) und
speichert das Ergebnis in das Register $X$. $N$ ist dabei eine feste natürliche
Zahl aus dem Bereich $[0, 255]$. Wird eine größere Zahl angegeben, so wird sie
modulo $256$ berechnet -- mit anderen Worten, nur das letzte Byte zählt.
Andere Schreibweise:
\[
    X \gets (\$Y \land (N \bmod 256)), \qquad N \in [0, 255]
\]
\paragraph{Bemerkung}
Die natürliche Zahl $N$ wird auf die Länge von 32 Bit mit Nullen verlängert
(Links-Verlängerung). Deshalb setzt diese Instruktion alle Bits mit Wertigkeiten
von $2^{8}$ bis $2^{31}$ im Register $X$ auf Null. Da diese Bits auf Null
gesetzt sind, wird das Ergebnis immer kleiner als $2^{8} = 256$ sein.

\paragraph{Beispiel}
für die Verwendung von \texttt{ANDI}:
\begin{lstlisting}
  SET  R1    0x0102
  ANDI R2 R1 0x01   # R2 = 0
\end{lstlisting}



\opdef{OR}{$X, Y, Z \in \Reg$}{0x52}{RRR}
Berechnet bitweise $\$Y \lor \$Z$ (oder-Verknüpfung) und speichert das Ergebnis
in das Register $X$.
\[
    X \gets (\$Y \lor \$Z)
\]


\opdef{ORI}{$X, Y \in \Reg$, $N \in \mathds{N}$}{0x53}{RRN}
\glqq Or Immediate\grqq.
Berechnet wie \opref{OR} ein bitweises \glqq oder\grqq\ zwischen $\$Y$ und $N$
und speichert das Ergebnis in das Register $X$. Dabei wird die natürliche Zahl
$N \in [0, 255]$ auf die Länge von 32 Bit mit Nullen verlängert
(Links-Verlängerung). Wird für $N$ einen Wert größer als $255$ angegeben, so
wird er modulo $256$ berechnet.
\[
    X \gets (\$Y \lor (N \bmod 256)), \qquad N \in [0, 255]
\]


\opdef{XOR}{$X, Y, Z \in \Reg$}{0x54}{RRR}
Berechnet bitweise $\$Y \oplus \$Z$ (xor-Verknüpfung) und speichert das Ergebnis
in das Register $X$.
\[
    X \gets (\$Y \oplus \$Z)
\]


\opdef{XORI}{$X, Y \in \Reg$, $N \in \mathds{N}$}{0x55}{RRN}
\glqq Xor Immediate\grqq.
Analog zur Instruktion \opref{XOR} mit dem Unterschied, dass $N$ eine direkt
angegebene Zahl aus dem Intervall $[0, 255]$ ist. Wird eine Zahl angegeben, die
größer als $255$ ist, die also mehr als 8 Bit braucht, wird sie auf 8 Bit
reduziert. 

Die Zahl $N$ wird links auf die Länge von 32 Bit mit Nullen aufgefüllt.
\[
    X \gets (\$Y \oplus (N \bmod 256)), \qquad N \in [0, 255]
\]


\opdef{NOT}{$X, Y \in \Reg$}{0x56}{RR0}
Invertiert alle Bits aus dem Register $Y$ und speichert das Ergebnis in das
Register $X$. Entspricht der Einerkomplement-Bildung.
\[
    X \gets \overline{\$Y}
\]
oder
\[
    X \gets \lnot \$Y
\]

\paragraph{Beispiel} 
Der folgende Code invertiert alle Bits aus dem Register $R2$ und speichert das
Ergebnis in das Register $R1$.
\begin{lstlisting}
  NOT R1 R2 # $R1 \gets \overline{R2}$
\end{lstlisting}



\opdef{NOTI}{$X\in \Reg$, $N \in \mathds{N}$}{0x57}{RNN}
\glqq Not Immediate\grqq.
Wie die Instruktion \opref{NOT} aber $N$ ist eine natürliche Zahl aus dem
Intervall $[0, 2^{15}-1]$. Diese konstante Zahl nach links auf die Länge von 32
Bit mit Nullen verlängert.
\[
    X \gets \overline{(N \bmod 256)}, \qquad N \in [0, 255]
\]

\paragraph{Beispiel} 
Der folgende Code invertiert alle Bits der Zahl $5$ und speichert das Ergebnis
in das Register $R1$.
\begin{lstlisting}
 NOTI R1 5 # $R1 \gets \overline{5}$
\end{lstlisting}


\opdef{NAND}{$X, Y, Z \in \Reg$}{0x58}{RRR}
Berechnet $\$Y \; \overline{\land} \; \$Z$ (nand-Verknüpfung) und speichert das
Ergebnis in das Register $X$. Gemäß dem Format \nameref{RRR} sind $X$, $Y$ und
$Z$ Register. Algebraische Schreibweise:
\[
    X \gets (\$Y \; \overline{\land} \; \$Z)
\]
oder
\[
    X \gets \overline{ (\$Y \land \$Z) }
\]


\opdef{NANDI}{$X, Y \in \Reg$, $N \in \mathds{N}$}{0x59}{RRN}
\glqq Nand Immediate\grqq.
Berechnet eine nand-Verknüpfung zwischen dem Inhalt des Registers $Y$ und der
konstanten Zahl $N$ und speichert das Ergebnis in das Register $X$.
$N \in [0, 255]$.
\[
    X \gets \big(\$Y \; \overline{\land} \; (N \bmod 256) \big),
    \qquad N \in [0, 255]
\]
Die Zahl $N$ wird auf die Länge von 32 Bit links mit Nullen aufgefüllt.


\opdef{NOR}{$X, Y, Z \in \Reg$}{0x5A}{RRR}
Verknüpft bitweise die Inhalte der Register $Y$ und $Z$ mit der nor-Operation
und speichert das Ergebnis in das Register $X$. \glqq nor\grqq\ ist die Negation
von \glqq or\grqq.
\[
    X \gets (\$Y \; \overline{\lor} \; \$Z)
\]

\paragraph{Beispiel}
\begin{lstlisting}
  NOR R1 R2 R3 # $R1 \gets  \overline{R2 \lor R3}  $
\end{lstlisting}


\opdef{NORI}{$X, Y \in \Reg$, $N \in \mathds{N}$}{0x5B}{RRN}
\glqq Nor Immediate\grqq.
Analog zur Instruktion \opref{NOR} mit dem Unterschied, dass $N$ eine konstante
Zahl aus dem Bereich $[0, 255]$ ist. Diese Zahl wird modulo $256$ berechnet und
auf der linken Seite auf die Länge von 32 Bit mit Nullen aufgefüllt.



\opdef{SHL}{$X, Y, Z \in \Reg$}{0x60}{RRR}
\glqq Shift Left\grqq.
Verschiebt die Bits aus dem Register $Y$ $\$Z$ Stellen nach links und speichert
das Ergebnis in das Register $X$. Auf der rechten Seite wird $X$ mit Nullen
aufgefüllt.

Die Stellenangabe im Register $Z$ muss positiv sein, bzw. wird als eine
vorzeichenlose Zahl interpretiert. $\$Z \in \mathds{N}$.

Andere Schreibweise:
\[
    X \gets \$Y << \$Z
\]
oder 
\[
    X \gets 
    ( \$Y \cdot 2^{\$Z}  ) \bmod 2^{32}
\]



\opdef{SHLI}{$X, Y \in \Reg$, $N \in \mathds{N}$}{0x61}{RRN}
\glqq Shift Left Immediate\grqq.
Verschiebt die Bits im Register $Y$ $N$ Stellen nach links und speichert das
Ergebnis in das Register $X$.
$N$ ist eine positive Zahl im Bereich $[0, 255]$.
Anstelle der nach links verschobenen Bits werden Nullen aufgefüllt.

$N$ muss positiv sein und es wird als solches interpretiert: $N \in \mathds{N}$.



\opdef{SHR}{$X, Y, Z \in \Reg$}{0x62}{RRR}
\glqq Shift Right logical\grqq.
Verschiebt die Bits aus dem Register $Y$ $\$Z$ Stellen nach rechts und speichert
das Ergebnis in das Register $X$. Anstelle der verschobenen Bits werden im
Register $X$ auf der linken Seite Nullen aufgefüllt. Gemäß dem
Instruktionsformat \nameref{RRR} sind $X$, $Y$ und $Z$ Register. Alle
Register-Inhalte werden vorzeichenlos interpretiert: 
$\$X, \$Y, \$Z \in \mathds{N}$.

\[
    X \gets (\$Y >> \$Z)
\]



\paragraph{Bemerkung}
Wird in dem Register $Y$ eine negative Zahl gespeichert, so löscht diese
Verschiebung das Vorzeichen, da auf der linken Seite Nullen aufgefüllt werden.
Um das Vorzeichen zu behalten, sollte man die Instruktion \opref{SHRA}
verwenden.

\paragraph{Beispiel}
Der folgende Code verschiebt die Bits im $R1$ eine Stelle nach rechts.
Es wird praktisch $R3 \gets (5 \div 2)$ berechnet.
\begin{lstlisting}
  SET  R1 5      # $R1 \gets 5$
  SET  R2 1      # $R2 \gets 1$
  SHR  R3 R1 R2  # $R3 \gets 2$
\end{lstlisting}


\opdef{SHRI}{$X, Y \in \Reg$, $N \in \mathds{N}$}{0x63}{RRN}
\glqq Shift Right Immediate\grqq.
Das Bitmuster im Register $Y$ wird $N$ Stellen nach rechts verschoben und das
Ergebnis in das Register $X$ gespeichert.
Dabei ist $N$ eine positive natürliche Zahl aus dem Intervall $[0, 255]$.
Auf der linken Seite werden die versetzten Bits mit Nullen ersetzt.
Siehe auch \nameref{opcode:SHRAI}.

\[
    X \gets (\$Y >> N)
\]

\opdef{SHRA}{$X, Y, Z \in \Reg$}{0x64}{RRR}
\glqq Shift Right Arithmetical\grqq.
Funktioniert wie die Instruktion \opref{SHR} mit dem Unterschied, dass auf der
linken Seite nicht mit Nullen, sondern mit dem ersten (höchstwertigen) Bit
aufgefüllt wird. Dies führt dazu, dass das Vorzeichenbit in $Y$ erhalten
bleibt.


\opdef{SHRAI}{$X, Y \in \Reg$, $N \in \mathds{N}$}{0x65}{RRN}
\glqq Shift Right Arithmetical Immediate\grqq.
Wie \opref{SHRA}, $N$ ist aber eine natürliche Zahl aus dem Intervall
$[0, 255]$.



\opdef{ROTL}{$X, Y, Z \in \Reg$}{0x68}{RRR}
\glqq Rotate Left\grqq. Die Bits aus dem Register $Y$ werden $\$Z$ Stellen nach
links \glqq rotiert\grqq\ und das Ergebnis in das Register $X$ gespeichert.
\[
    X \gets \$Y \circlearrowleft \$Z
\]
Der Registerinhalt $\$Z$ wird als vorzeichenlose Zahl interpretiert: 
$\$Z \in \mathds{N}$.

Die Rotation\index{Rotation} bedeutet, dass die Bits nach links verschoben
werden und diejenigen Bits, die über die linke Grenze hinausfallen auf der
rechten Seite wieder eingefügt werden. 
Dies bedeutet, dass mit jedem verschobenen Bit, ein Bit ganz links (Wertigkeit
$2^{31}$) fällt aus und wird wieder ganz rechts mit Wertigkeit $2^{0}$ eingefügt.
Nach 32 Rotationen eine Stelle nach links ist das ursprüngliche Bitmuster
wieder hergestellt.

\paragraph{Beispiel}
Der folgende Code zeigt ein Beispiel für die Verwendung dieser Instruktion.
\begin{lstlisting}
  SET  R1 2
  SET  R2 1
  ROTL R3 R1 R2      # links-rotation von R1 eine Stelle
                     # R3 = 4
  SET  R1 0x80000000 # nur das linke Bit gesetzt
  ROTL R4 R1 R2      # R4 = 0x01
\end{lstlisting}


\opdef{ROTLI}{$X, Y \in \Reg$, $N \in \mathds{N}$}{0x69}{RRN}
\glqq Rotate Left Immediate\grqq.
Wie \opref{ROTL} aber $N$ ist eine konstante Zahl aus dem Intervall $[0, 255]$.

\paragraph{Beispiel}
Verwendung:
\begin{lstlisting}
  ROTL R4 R1  6 # rotiere die Bits in R1 um 6 Stellen
  ROTL R4 R1 -6 # Fehler! da $-6 \not\in \mathds{N}$
\end{lstlisting}



\opdef{ROTR}{$X, Y, Z \in \Reg$}{0x6A}{RRR}
\glqq Rotate Right\grqq.
Funktioniert genauso wie die Instruktion \opref{ROTL} mit dem Unterschied, dass
die Rotationen (Verschiebungen) nach rechts geführt werden.
\[
    X \gets \$Y \circlearrowright \$Z
\]


\opdef{ROTRI}{$X, Y \in \Reg$, $N \in \mathds{N}$}{0x6B}{RRN}
\glqq Rotate Right Immediate\grqq.
Funktioniert genauso wie \opref{ROTLI}, nur die Rotationen sind nach rechts
geführt.



