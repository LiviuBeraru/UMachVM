\section{Systeminstruktionen}

\opdef{INT}{$N \in \mathds{N}$}{0xA0}{NNN}
\glqq Interrupt\grqq.
Generiert ein Interrupt\index{Interrupt} in der Programmausführung. Die Zahl $N$
ist 3 Byte groß und wird als Interruptnummer interpretiert (siehe dazu auch den
Abschnitt \ref{subsubsec:Interrupttabelle} auf der Seite
\pageref{subsubsec:Interrupttabelle}). Falls die Interruptnummer $N$ keine von
der Maschine erkannte Interruptnummer ist, wird der Interrupt mit Nummer 0
ausgeführt. Für mehr Informationen betreffend das Interrupt-System der UMach
Maschine siehe den Abschnitt \ref{sec:Interrupts} auf der Seite
\pageref{sec:Interrupts}.

Der Befehl \opref{INT} speichert, wie der Befehle \opref{CALL} auch, den
aktuellen Wert des Registers \texttt{PC} auf den Stack. Dies erlaubt einem
Interrupt-Handler, zur normalen Ausführung des Programms zurückzukehren, indem
er den Befehl \opref{RET} verwendet. Ein Interrupt-Handler ist somit wie
eine normale Subroutine, die entweder direkt mittels \opref{CALL}, oder indirekt
mit dem Befehl \opref{INT} aufgerufen werden kann.
