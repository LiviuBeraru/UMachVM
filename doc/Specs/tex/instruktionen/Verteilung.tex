\section{Verteilung des Befehlsraums}
\index{Befehlsraum}
Zur besseren Übersicht der verschiedenen UMach-\glspl{Instruktion}, unterteilen
wir den \gls{Instruktionssatz} der UMach virtuellen Maschine in den folgenden
Kategorien:
\index{Instruktionen!Kategorien}

\index{Befehlsraum!Verteilung}
\begin{enumerate}
  \item Kontrollinstruktionen,  die die Maschine in ihrer gesamten
    Funktionalität betreffen, wie z.B. den Betriebsmodus umschalten.
  \item Lade- und Speicherbefehle, die Register mit Werten aus dem Speicher, 
    anderen Registern oder direkten numerischen Angaben laden und die
    Registerinhalte in den Speicher schreiben.
  \item Arithmetische Instruktionen, die einfache arithmetische Operationen
    zwischen Registern veranlassen.
  \item Logische Instruktionen, die logische Verknüpfungen zwischen
    Registerinhalten oder Operationen auf Bit-Ebene in Registern anweisen.
  \item Vergleichsinstruktionen, die einen Vergleich zwischen
    Registerinhalten angeben.
  \item Sprunginstruktionen, die bedingt oder unbedingt sein können.
    Sie weisen die UMach Maschine an, die Programmausführung an einer anderen
    Stelle fortzufahren.
  \item Unterprogramm-Steuerung, bzw. Instruktionen, die die Ausführung von
    Unterprogrammen (Subroutinen) steuern.
  \item Systeminstruktionen, die die Unterstützung eines
    Betriebssystem ermöglichen.
\end{enumerate}

Die oben angegebenen Instruktionskategorien unterteilen den \gls{Befehlsraum} in
8 Bereiche. Es gibt $256$ mögliche Befehle, gemäß $2^{8} = 256$.
Die Verteilung der Kategorien auf die verschiedenen Maschinencode-Intervallen
wird in der Tabelle \ref{tab:Befehlraumverteilung} auf Seite
\pageref{tab:Befehlraumverteilung} angegeben.

\begin{table}
\index{Befehlsraum!Verteilungstabelle}
  \centering
  \begin{tabular}{|c|l|}                        \hline
    Maschinencodes   & Kategorie              \\\hline\hline
    \texttt{00 - 0F} & Kontrollbefehle        \\
    \texttt{10 - 2F} & Lade-/Speicherbefehle  \\
    \texttt{30 - 4F} & Arithmetische Befehle  \\
    \texttt{50 - 6F} & Logische Befehle       \\
    \texttt{70 - 7F} & Vergleichsbefehle      \\
    \texttt{D0 - DF} & Sprungbefehle          \\
    \texttt{E0 - EF} & Unterprogrambefehle    \\
    \texttt{F0 - FF} & Systembefehle          \\\hline
  \end{tabular}
  \caption[Verteilung des Befehlsraums]
          {Verteilung des Befehlsraums nach Befehlskategorien.
          Die Zahlen sind im Hexadezimalsystem angegeben.}
  \label{tab:Befehlraumverteilung}
\end{table}




\begin{table}%no, this is not normal human being LaTeX
\newcolumntype{C}[1]{>{\ttfamily\footnotesize\centering\let\newline\\\arraybackslash\hspace{0pt}}p{#1}}
\newcolumntype{L}{>{\ttfamily\footnotesize}l}
\newcommand{\nhex}[1]{\multirow{2}{*}{#1}}
\centering
\caption[Befehlentabelle]
        {Befehlentabelle}
\label{tab:Befehlentabelle}
\begin{tabular}{|L||*{8}{C{1.2cm}|}}
\toprule\hline
          &   0   &   1   &   2   &   3    &   4    &   5    &    6    &    7    \\\hline\hline
\nhex{0}  &  NOP  &  RST  &  CRM  &  CSM   &  DIE   &  RSR   &  AUTSM  &  SOCL   \\\cline{2-9}
          & HATE  & TRST  &  ZMB  &  ALIV  &        &        &         &         \\\hline\hline
\nhex{1}  & SET   & SETU  & COPY  &  MOVE  &        &        &         &         \\\cline{2-9}
          &       &       &       &        &        &        &         &         \\\cline{1-9}
\nhex{2}  &  LB   &  LBU  &  LH   &  LHU   &  LW    &  LWU   &         &         \\\cline{2-9}
          &       &       &       &        &        &        &         &         \\\cline{1-9}
\nhex{3}  &  SB   &  SBU  &  SH   &  SHU   &  SW    &  SWU   &         &         \\\cline{2-9}
          &       &       &       &        &        &        &         &         \\\cline{1-9}
\nhex{4}  & PUSHB & PUSHH & PUSH  &        &        &        &         &         \\\cline{2-9}
          & POPB  & POPH  &  POP  &        &        &        &         &         \\\hline\hline
\nhex{5}  &  ADD  & ADDU  & ADDI  & ADDIU  &        &        &         &         \\\cline{2-9}
          &  SUB  & SUBU  & SUBI  & SUBIU  &        &        &         &         \\\cline{1-9}
\nhex{6}  &  MUL  & MULU  & MULI  & MULIU  &        &        &         &         \\\cline{2-9}
          &  DIV  & DIVU  & DIVI  & DIVIU  &        &        &         &         \\\cline{1-9}
\nhex{7}  &  MOD  &       & MODI  &        &        &        &         &         \\\cline{2-9}
          &  NEG  &  ABS  &  INC  &  DEC   &        &        &         &         \\\cline{1-9}
\nhex{8}  &       &       &       &        &        &        &         &         \\\cline{2-9}
          &       &       &       &        &        &        &         &         \\\hline\hline
\nhex{9}  &  AND  & ANDI  &  OR   &  ORI   &  XOR   &  XORI  &         &         \\\cline{2-9}
          & NAND  & NANDI &  NOR  &  NORI  &        &        &         &         \\\cline{1-9}
\nhex{A}  &  SHL  & SHLI  &  SHR  &  SHRI  &  SHRA  & SHRAI  &  ROTL   & ROTR    \\\cline{2-9}
          &  NOT  &       &       &        &        &        &         &         \\\hline\hline
\nhex{B}  &  CMP  & CMPU  & CMPI  & CMPIU  &        &        &         &         \\\cline{2-9}
          &       &       &       &        &        &        &         &         \\\hline\hline
\nhex{C}  &  BZ   &  BNZ  &  BLZ  &  BLEZ  &  BGZ   &  BGEZ  &         &         \\\cline{2-9}
          &  BEI  &  BNI  &  BLI  &  BLEI  &  BGI   &  BGEI  &         &         \\\cline{1-9}
\nhex{D}  &       &       &       &        &        &        &         &         \\\cline{2-9}
          &  JMP  &       &       &        &        &        &         &         \\\hline\hline
\nhex{E}  &  CALL &  RET  &       &        &        &        &         &         \\\cline{2-9}
          &       &       &       &        &        &        &         &         \\\hline\hline
\nhex{F}  &  WAKE &       &       &        &        &        &         &         \\\cline{2-9}
          &  KILL &       &       &        &        &        &         &         \\\hline\hline
          &   8   &   9   &   A   &   B    &   C    &   D    &    E    &    F    \\\hline
\bottomrule
\end{tabular}
\end{table} 


Die Tabelle \ref{tab:Befehlentabelle} auf der Seite 
\pageref{tab:Befehlentabelle} enthält eine Übersicht aller Befehle und
deren Maschinencodes.
Diese Tabelle wird folgenderweise gelesen:
in der am weitesten linken Spalte wird die erste hexadezimale Ziffer eines
Befehls angegeben (ein Befehl ist zweistellig im Hexadezimalsystem).
Jede solche Ziffer hat rechts Zwei Zeilen, die von links nach rechts gelesen
werden: eine Zeile für die Ziffern von 0 bis 8, die anderen für die übrigen
Ziffern 9 bis F (im Hexadezimalsystem). Die Assemblernamen (Mnemonics) der
einzelnen Befehle sind an der entsprechenden Stelle angegeben.

\paragraph{Definitionsstruktur}
Im den folgenden Abschnitten werden die einzelnen Instruktionen beschrieben. Zu
jeder Instruktion wird der \gls{Assemblername}, die Parameter, der Maschinencode
(\gls{Maschinenname}) und das Instruktionsformat, das die Typen der Parameter
definiert, formal angegeben. Zudem werden Anwendungsbeispiele angegeben. Die
Instruktionsformate können im Abschnitt \ref{sec:Instruktionsformate} ab der
Seite \pageref{sec:Instruktionsformate} nachgeschlagen werden.

\paragraph{Zur Notation}
Mit \Reg\index{\Reg} wird die Menge aller Register
gekennzeichnet\footnote{Nicht verwechseln mit den Symbolen $\mathds{R}$ und
$\mathbb{R}$, die die Menge aller reellen Zahlen bedeuten.}.
Die Notation $X \in \Reg$ bedeutet, dass $X$ ein Element aus dieser Menge ist,
mit anderen Worten, dass $X$ ein Register ist. Analog bedeutet die
Schreibweise $X,Y \in \Reg$, dass $X$ und $Y$ beide Register sind.

Gilt $X, Y\in \Reg$ und ist $\sim$ eine durch einen Befehl definierte Relation
zwischen $X$ und $Y$, so bezieht sich die Schreibweise $X \sim Y$ nicht auf die
Maschinennamen von $X$ und $Y$, sondern auf deren Inhalte. Zum Beispiel, haben
die Register $R1$ und $R2$ die Maschinencodes \texttt{0x01} und \texttt{0x02}
und sind sie mit den Werten $4$ bzw. $5$ belegt, so bedeutet $R1 + R2$ das
gleiche wie $4 + 5 = 9$ und nicht \texttt{0x01 + 0x02 = 0x03}.

Andere verwendeten Schreibweisen:
\begin{center}
\begin{tabular}{l|l}\toprule
 $\mathds{N}$ & Menge aller natürlichen Zahlen: $0,1,2,\ldots$         \\
 $\mathds{Z}$ & Menge aller ganzen Zahlen: $\ldots,-2,-1,0,1,2,\ldots$ \\
 $N \in \mathds{N}$
              & $N$ ist Element von $\mathds{N}$, oder liegt im Bereich von
              $\mathds{N}$                                             \\
 $X \gets Y$  & $X$ wird auf $Y$ gesetzt                               \\
 $mem(X)$     & Speicherinhalt an der Adresse $X$ (1 Byte)             \\
 $mem_{n}(X)$ & Speicherinhalt an der Adresse $X$ ($n$ Bytes)          \\
\bottomrule
\end{tabular}
\end{center} 
