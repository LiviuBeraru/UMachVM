\section{Unterprogramminstruktionen}

\opdef{GO}{$X \in \Reg$}{0x90}{R00}
Setzt $PC$ auf die angegebene absolute Adresse. Hierbei ist zu beachten,  dass
nicht in die Mitte eines Befehles gesprungen wird. Dies zu gewährleisten liegt
in der Verantwortung des Programmierers (die Maschine prüft nicht, ob die
angegebene Adresse $\$X$ ein Vielfaches von 4 ist).
\[
    PC \gets \$X
\]

\paragraph{Fehler}
Falls das Sprungziel $\$X$ eine Speicheradresse ist, die außerhalb des
Code-Segments liegt, wir der Interrupt mit Nummer 17 generiert
(Zugriffsverletzung, segfault) und das Bit mit Nummer 9 im Register \texttt{ERR}
gesetzt (siehe auch Tabelle \ref{tab:ERR-register} auf Seite
\pageref{tab:ERR-register}).


\opdef{CALL}{$N \in \mathds{N}$}{0x91}{NNN}
Funktionert wie der Befehl \opref{JMP} mit dem Unterschied, dass bevor
\texttt{PC} neugesetzt wird, wird es auf den Stack gepusht. Enspricht
\begin{align*}
 & \text{\opref{PUSH} \texttt{PC}} \\
 & \text{\opref{JMP}  \texttt{N}} 
\end{align*}

\paragraph{Beispiel}
Dieser Befehl dient zur Implementierug und Verwendung von Funktionen (function
call). Verwendet man diesen Befehl zum Sprung zu einem anderen Programmteil
(Aufruf einer Subroutine), so kann aus diesem Teil zurückgekehrt werden, indem
man den Befehl \opref{RET} verwendet. Dies ermöglicht ein Code wie der folgende:
\begin{lstlisting}
CALL routine
CP   R1 R32
EOP

routine:
  SET R32 66
  RET
\end{lstlisting}
Hier wird nach der Ausführung des ersten Befehls (ein \opref{CALL}) mit der 4.
Instruktion weitergemacht (nach dem Label \texttt{routine}). Der letzte Befehl
(\opref{RET}) bewirkt ein Sprung nach dem \opref{CALL}-Befehl, also zum Befehl
\opref{CP}. Der letzte Befehl \opref{EOP} wird verwendet, um das Ende des
Programms zu signalisieren, sonst würde die Maschine wieder die Routine
ausführen, wobei der Befehl \opref{RET} einen Fehler erzeugen würde (da dieser
Befehl die Return-Adresse aus dem Stack holt und dort steht möglicherweise
keine Adresse).


\paragraph{Fehler}
Wie bei dem Befehl \opref{GO}.


\opdef{RET}{keine}{0x92}{000}
\opref{POP}t eine Adresse vom Stack in das Register \texttt{PC}. Dieser Befehl
entspricht einem Rückkehr aus einer Subroutine, analog der Anweisung
\texttt{return}\index{return} aus den höheren Programmiersprachen.
\[
  \text{\opref{POP} \texttt{PC}} 
\]
Dieser Befehl wird in Zusammenspiel mit den Befehlen \opref{CALL} und
\opref{INT} verwendet.

\paragraph{Fehler}
Wie bei der \opref{POP} und \opref{GO} Operation.

