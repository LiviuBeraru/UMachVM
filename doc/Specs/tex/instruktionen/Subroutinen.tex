\section{Unterprogramminstruktionen}

\opdef{GO}{$X \in \Reg$}{0x90}{R00}
Setzt $PC$ auf die angegebene absolute Adresse. Hierbei ist zu beachten,  dass
nicht in die Mitte eines Befehles gesprungen wird. Dies zu gewährleisten liegt
in der Verantwortung des Programmierers (die Maschine prüft nicht, ob die
angegebene Adresse $\$X$ ein Vielfaches von 4 ist).
\[
    PC \gets \$X
\]

\paragraph{Fehler}
Falls das Sprungziel $\$X$ eine Speicheradresse ist, die außerhalb des
Code-Segments liegt, wir der Interrupt mit Nummer 17 generiert
(Zugriffsverletzung, segfault) und das Bit mit Nummer 9 im Register \texttt{ERR}
gesetzt (siehe auch Tabelle \ref{tab:ERR-register} auf Seite
\pageref{tab:ERR-register}).


\opdef{CALL}{$N \in \mathds{N}$}{0x91}{NNN}
Funktionert wie der Befehl \opref{JMP} mit dem Unterschied, dass bevor
\texttt{PC} neugesetzt wird, wird es auf den Stack gepusht. Enspricht
\begin{align*}
 & \text{\opref{PUSH} \texttt{PC}} \\
 & \text{\opref{JMP}  \texttt{N}} 
\end{align*}
Dieser Befehl dient der Implementierug und Verwendung von Funktionen (function
call).

\paragraph{Fehler}
Wie bei dem Befehl \opref{GO}.


\opdef{RET}{keine}{0x92}{000}
\opref{POP}t eine Adresse vom Stack in das Register \texttt{PC}. Dieser Befehl
entspricht einem Rückkehr aus einer Subroutine, analog der Anweisung
\texttt{return}\index{return} aus den höheren Programmiersprachen.

\[
  \text{\opref{POP} \texttt{PC}} 
\]

\paragraph{Fehler}
Wie bei der \opref{POP} und \opref{GO} Operation.



