\section{Vergleichsinstruktionen}

\opdef{CMP}{$X,Y,Z \in \Reg$}{0xB0}{RRR}
\glqq Compare\grqq.
Vergleicht $Y$ mit $Z$. Sind die Werte gleich, so erhält $X$ den Wert $0$
zugewiesen. Ist $Y$ kleiner $Z$, so erhält $X$ den Wert $-1$, $Y$ größer $Z$ den
Wert $+1$.
\[
    X \gets
    \begin{cases}
        -1 & \text{ falls } Y < Z  \\
     \pm 0 & \text{ falls } Y = Z  \\
        +1 & \text{ falls } Y > Z 
    \end{cases}
\]


\opdef{CMPU}{$X,Y,Z \in \Reg$}{0xB1}{RRR}
\glqq Compare Unsigned\grqq.
Vergleicht analog zu \opref{CMP} $Y$ mit $Z$.
Werte werden als vorzeichenlos interpretiert.
\[
    X \gets
    \begin{cases}
        -1 & \text{ falls } Y < Z  \\
     \pm 0 & \text{ falls } Y = Z  \\
        +1 & \text{ falls } Y > Z 
    \end{cases}
\]


\opdef{CMPI}{$X,Y \in \Reg$, $N \in \mathds{Z}$}{0xB2}{RRN}
\glqq Compare Immediate\grqq. 
Vergleicht $Y$ mit angegebenen Wert $N$. Dabei nimmt $N$ Werte aus dem Intervall
$[-128, 127]$. $X$ wird entsprechend der folgenden Formel gesetzt:
\[
    X \gets
    \begin{cases}
        -1 & \text{ falls } Y < N  \\
     \pm 0 & \text{ falls } Y = N  \\
        +1 & \text{ falls } Y > N 
    \end{cases}
\]


\opdef{CMPIU}{$X,Y \in \Reg$, $N \in \mathds{N}$}{0xB3}{RRN}
\glqq Compare Immediate Unsigned\grqq.
Vergleicht, wie \opref{CMPI} auch, $Y$ mit angegebenen Wert $N$ und setzt
entsprechend das Register $X$. Der Unterschied ist, dass $Y$ und $N$ als 
vorzeichenlos betrachtet werden.
\[
    X \gets
    \begin{cases}
       -1 & \text{ falls } Y < N  \\
     \pm0 & \text{ falls } Y = N  \\
       +1 & \text{ falls } Y > N 
    \end{cases}
\]

