\section{Vergleichsinstruktionen}
\label{sec:Vergleichsinstruktionen}

Die Vergleichsinstruktionen vergleichen den Inhalt eines Registers mit dem
Inhalt eines anderen Registers oder mit einer angegebenen ganzen Zahl. Das
Ergebnis der Vergleichsinstruktion wird in das Register \texttt{CMP}
gespeichert. Dieses Ergebnis ist $-1$, $0$ oder $+1$ und wird folgenderweise
berechnet:
werden zwei Werte $x$ und $y$ vergliechen, so wird das Register \texttt{CMP}
wie folgt gesetzt:
\[
    CMP \gets
    \begin{cases}
        -1 & \text{ falls } x < y  \\
     \pm 0 & \text{ falls } x = y  \\
        +1 & \text{ falls } x > y 
    \end{cases}
\]

\opdef{CMP}{$X, Y \in \Reg$}{0xB0}{RR0}
\glqq Compare\grqq.
Vergleicht die Inhalte der Register $X$ und $Y$.


\opdef{CMPU}{$X, Y \in \Reg$}{0xB1}{RR0}
\glqq Compare Unsigned\grqq.
Vergleicht analog zu \opref{CMP} -- aber vorzeichenlos -- $X$ mit $Y$
(die Inhalte der Register $X$ und $Y$ werden als vorzeichenlose Werte
ausgewertet).


\opdef{CMPI}{$X \in \Reg$, $N \in \mathds{Z}$}{0xB2}{RNN}
\glqq Compare Immediate\grqq. 
Vergleicht $X$ mit angegebenen festen Wert $N$.
Dabei nimmt $N$ Werte aus dem Intervall
$[-2^{15}, 2^{15}-1]$, entsprechend dem Datentyp \glqq Half\grqq.


\opdef{CMPIU}{$X \in \Reg$, $N \in \mathds{N}$}{0xB3}{RNN}
\glqq Compare Immediate Unsigned\grqq.
Vergleicht, wie \opref{CMPI} auch, $X$ mit angegebenen Wert $N$ und setzt
entsprechend das Register \texttt{CMP}.
Der Unterschied ist, dass $X$ und $N$ als vorzeichenlos betrachtet werden.
$N$ hat den Datentyp \glqq Half\grqq.

