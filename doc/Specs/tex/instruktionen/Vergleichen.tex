\section{Vergleichsinstruktionen}
\label{sec:Vergleichsinstruktionen}

Die Vergleichsinstruktionen vergleichen den Inhalt eines Registers mit dem
Inhalt eines anderen Registers oder mit einer angegebenen ganzen Zahl. Das
Ergebnis der Vergleichsinstruktion wird in das Register \texttt{CMPR}
gespeichert (siehe auch Tabelle \ref{tab:Spezialregister} auf der Seite
\pageref{tab:Spezialregister}). Dieses Ergebnis ist $-1$, $0$ oder $+1$ und wird
folgenderweise berechnet: werden zwei Werte $x$ und $y$ vergliechen, so wird das
Register \texttt{CMPR} wie folgt gesetzt:
\[
    CMPR \gets
    \begin{cases}
        -1 & \text{ falls } x < y  \\
     \pm 0 & \text{ falls } x = y  \\
        +1 & \text{ falls } x > y 
    \end{cases}
\]

Alle diese Befehle generieren den Interrupt mit Nummer 9 (ungültige
Registernummer), falls die angegebene Registernummern ungültig sind -- das
heißt, falls es die Register nicht gibt.



\opdef{CMP}{$X, Y \in \Reg$}{0x70}{RR0}
\glqq Compare\grqq.
Vergleicht die Inhalte der Register $X$ und $Y$.


\opdef{CMPU}{$X, Y \in \Reg$}{0x71}{RR0}
\glqq Compare Unsigned\grqq.
Vergleicht analog zu \opref{CMP} -- aber vorzeichenlos -- $\$X$ mit $\$Y$
(die Inhalte der Register $X$ und $Y$ werden als vorzeichenlose Werte
ausgewertet, $\$X, \$Y \in \mathds{N}$).


\opdef{CMPI}{$X \in \Reg$, $N \in \mathds{Z}$}{0x72}{RNN}
\glqq Compare Immediate\grqq. 
Vergleicht $\$X$ mit angegebenen festen Wert $N$. Dabei nimmt $N$ Werte aus dem
Intervall $[-2^{15}, 2^{15}-1]$.


