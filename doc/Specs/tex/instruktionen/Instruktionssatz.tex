\chapter{Instruktionssatz}\index{Instruktionssatz}

In diesem Abschnitt werden alle Instruktionen (Befehle) der UMach VM
vorgestellt.

\section{Instruktionsformate}
\blindtext

\section{Klassifizierung der Instruktionen}
Zur besseren Übersicht der verschiedenen UMach-Befehlen, unterteilen wir die
Menge aller Befehlen in den folgenden Kategorien
(Abschnitt \ref{sec:Instruktionen} ab der Seite
\pageref{sec:Instruktionen} beinhaltet eine komplette Liste aller
Instruktionen):

\begin{enumerate}
  \item Kontrolle-Befehle: das sind Befehlen, die die Maschine selbst
    kontrollieren, wie z.B. den Betriebsmodus umschalten oder Ausschalten.
  \item Arithmetische Befehle: Addieren, Subtrahieren etc. Alle arithmetische
    Befehle operieren auf Register Inhalte.
  \item Logische Befehle.
  \item Programflußkontrolle.
  \item Sprünge.
  \item Load- und Store-Befehle: die einzigen Befehle, die einen Zugriff auf den
    Speicher ausführen.
  \item Input-Output-Befehle: operieren auf die I/O-Einheit der UMach VM.
\end{enumerate}


\section{Instruktionen}\index{Instruktionen}
\label{sec:Instruktionen}

\blindtext


