\section{Arithmetische Instruktionen} 

\opdef{ADD}{$X,Y,Z \in \Reg$}{0x50}{RRR}
Vorzeichen behaftete Addition der Registerinhalte $Y$ und $Z$ und
speichern des Ergebnisses in das Register $X$. Entspricht dem algebraischen
Ausdruck
\[
    X \gets Y + Z
\]
Beispiel:
\begin{lstlisting}
  SET   R1 1     # $R1 \gets 1$
  SET   R2 2     # $R2 \gets 2$
  ADD   R3 R1 R2 # $R3 \gets R1 + R2 = 1 + 2 = 3$
  #     X  Y  Z
  SET   R2 -2    # $R2 \gets -2$
  ADD   R3 R3 R2 # $R3 \gets R3 + R2 = 3 +(-2) = 1$
  ADD   R3 R4  5 # Fehler! 5 kein Register
\end{lstlisting}
Vorzeichenlose Addition wird durch den Befehl \texttt{ADDU} ausgeführt.


\opdef{ADDU}{$X,Y,Z \in \Reg$}{0x51}{RRR}
\glqq Add Unsigned\grqq.
Vorzeichenlose Addition der Register $Y$ und $Z$. Das Ergebnis wird in das
Register $X$ gespeichert. Enthält $Y$ oder $Z$ ein Vorzeichen (höchstwertiges
Bit auf 1 gesetzt), so wird es nicht als solches interpretiert, sondern als
Wertigkeit, die zum Betrag des Wertes hinzuaddiert wird ($+2^{31}$).

\begin{lstlisting}
  SET   R1 1     # $R1 \gets 1$
  SET   R2 -2    # $R2 \gets -2$
  ADDU  R3 R1 R2 # $R3 \gets (1 + 2 + 2^{31}) = 2147483651$
\end{lstlisting}



\opdef{ADDI}{$X,Y\in\Reg$, $N\in\mathds{Z}$}{0x52}{RRN}
\glqq Add Immediate\grqq.
Hinzuaddieren eines festen vorzeichenbehafteten ganzzahligen Wert $N$ zum Inhalt
des Registers $Y$ und speichern des Ergebnisses in das Register $X$.
Entspricht dem algebraischen Ausdruck
\[
  X \gets Y + N
\]
$N$ wird als vorzeichenbehaftete 8-Bit Zahl in Zweierkomplement-Darstellung
interpretiert und kann entsprechend Werte von $-128$ bis $127$ aufnehmen.

Beispiel:
\begin{lstlisting}
  SET   R1 1     # $R1 \gets 1$
  ADDI  R2 R1 2  # $R2 \gets R1 + 2 = 1 + 2 = 3$
  #     X  Y  N
  ADDI  R2 R2 -3 # $R2 \gets R2 + (-3) = 3 +(-2) = 1$
  ADDI  R2 R3 R4 # Fehler! R4 kein $n \in \mathds{Z}$
\end{lstlisting}



\opdef{ADDIU}{$X,Y \in \Reg$, $N\in\mathds{N}$}{0x53}{RRN}
\glqq Add Unsigned Immediate\grqq.
Vorzeichenlose Addition des ganzzahligen Wertes $N$ zum Inhalt des Registers $Y$
und speichern des Ergebnisses in das Register $X$.
Der Inhalt des Registers $Y$, die Zahl $N$ und das Ergebnis $Y + N$ werden als
vorzeichenlose Werte interpretiert.



\opdef{SUB}{$X, Y, Z \in \Reg$}{0x58}{RRR}


\opdef{SUBU}{$X, Y, Z \in \Reg$}{0x59}{RRR}
\glqq Subtract Unsigned\grqq.


\opdef{SUBI}{$X, Y \in \Reg$, $N\in\mathds{Z}$}{0x5A}{RRN}
\glqq Subtract Immediate\grqq.


\opdef{SUBIU}{$X, Y \in \Reg$, $N\in\mathds{N}$}{0x5B}{RRN}
\glqq Subtract Immediate Unsigned\grqq.



\opdef{MUL}{$X, Y, Z \in \Reg$}{0x60}{RRR}



\opdef{MULU}{$X, Y, Z \in \Reg$}{0x61}{RRR}
\glqq Multiplicate Unsigned\grqq.



\opdef{MULI}{$X, Y \in \Reg$, $N\in\mathds{Z}$}{0x62}{RRN}
\glqq Multiplicate Immediate\grqq.



\opdef{MULIU}{$X, Y \in \Reg$, $N\in\mathds{N}$}{0x63}{RRN}
\glqq Multiplicate Immediate Unsigned\grqq.




\opdef{DIV}{$X, Y, Z \in \Reg$}{0x68}{RRR}



\opdef{DIVU}{$X, Y, Z \in \Reg$}{0x69}{RRR}
\glqq Divide Unsigned\grqq.



\opdef{DIVI}{$X, Y \in \Reg$, $N\in\mathds{Z}$}{0x6A}{RRN}
\glqq Divide Immediate\grqq.



\opdef{DIVIU}{$X, Y \in \Reg$, $N\in\mathds{N}$}{0x6B}{RRN}
\glqq Divide Immediate Unsigned\grqq.



\opdef{MOD}{$X, Y, Z \in \Reg$}{0x70}{RRR}
Modulo.


\opdef{MODI}{$X, Y, Z \in \Reg$, $N \in \mathds{N}$}{0x72}{RRN}
Modulo Immediate.



\opdef{NEG}{$X, Y \in \Reg$}{0x78}{RR0}
\glqq Negate\grqq.
Wechselt das Vorzeichen des Registers $Y$ und speichert das Ergebnis in das 
Register $X$.



\opdef{ABS}{$X, Y \in \Reg$}{0x79}{RR0}
\glqq Absolute\grqq.
Speichert den absoluten Wert des Registers $Y$ in das Register $X$.


\opdef{INC}{$X \in \Reg$}{0x7A}{R00}
\glqq Increment\grqq.
Inkrementiert den Inhalt des Registers $X$.



\opdef{DEC}{$X \in \Reg$}{0x7B}{R00}
\glqq Decrement\grqq.
Dekrementiert den Inhalt des Registers $X$.





