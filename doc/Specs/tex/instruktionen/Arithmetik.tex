\section{Arithmetische Instruktionen} 

\opdef{ADD}{$X,Y,Z \in \Reg$}{0x50}{RRR}
Vorzeichen behaftete Addition der Registerinhalte $Y$ und $Z$.
Das Ergebnis der Addition wird in das Register $X$ gespeichert.
Entspricht dem algebraischen Ausdruck
\[
    X \gets Y + Z
\]
Beispiel:
\begin{lstlisting}
  SET   R1 1     # $R1 \gets 1$
  SET   R2 2     # $R2 \gets 2$
  ADD   R3 R1 R2 # $R3 \gets R1 + R2 = 1 + 2 = 3$
  #     X  Y  Z
  SET   R2 -2    # $R2 \gets -2$
  ADD   R3 R3 R2 # $R3 \gets R3 + R2 = 3 +(-2) = 1$
  ADD   R3 R4  5 # Fehler! 5 kein Register
\end{lstlisting}
Vorzeichenlose Addition wird durch den Befehl \texttt{ADDU} ausgeführt.


\opdef{ADDU}{$X,Y,Z \in \Reg$}{0x51}{RRR}
\glqq Add Unsigned\grqq.
Vorzeichenlose Addition der Register $Y$ und $Z$. Das Ergebnis wird in das
Register $X$ gespeichert. Enthält $Y$ oder $Z$ ein Vorzeichen (höchstwertiges
Bit auf 1 gesetzt), so wird es nicht als solches interpretiert, sondern als
Wertigkeit, die zum Betrag des Wertes hinzuaddiert wird ($+2^{31}$).

\begin{lstlisting}
  SET   R1 1     # $R1 \gets 1$
  SET   R2 -2    # $R2 \gets -2$
  ADDU  R3 R1 R2 # $R3 \gets (1 + 2 + 2^{31}) = 2147483651$
\end{lstlisting}



\opdef{ADDI}{$X,Y\in\Reg$, $N\in\mathds{Z}$}{0x52}{RRN}
\glqq Add Immediate\grqq.
Hinzuaddieren eines festen vorzeichenbehafteten ganzzahligen Wert $N$ zum Inhalt
des Registers $Y$ und speichern des Ergebnisses in das Register $X$.
Entspricht dem algebraischen Ausdruck
\[
  X \gets Y + N
\]
$N$ wird als vorzeichenbehaftete 8-Bit Zahl in Zweierkomplement-Darstellung
interpretiert und kann entsprechend Werte von $-128$ bis $127$ aufnehmen.

Beispiel:
\begin{lstlisting}
  SET   R1 1     # $R1 \gets 1$
  ADDI  R2 R1 2  # $R2 \gets R1 + 2 = 1 + 2 = 3$
  #     X  Y  N
  ADDI  R2 R2 -3 # $R2 \gets R2 + (-3) = 3 +(-2) = 1$
  ADDI  R2 R3 R4 # Fehler! R4 kein $n \in \mathds{Z}$
\end{lstlisting}



\opdef{SUB}{$X, Y, Z \in \Reg$}{0x58}{RRR}
Subtrahiert die Registerinhalte von $Y$ und $Z$ und speichert das Ergebnis in
das Register $X$. Entspricht dem Ausdruck
\[
    X \gets (Y - Z)
\]
Wobei $X$, $Y$ und $Z$ Register sind.



\opdef{SUBU}{$X, Y, Z \in \Reg$}{0x59}{RRR}
\glqq Subtract Unsigned\grqq.
Analog zur Instruktion \opref{SUB} mit dem Unterschied, dass alle Werte und
Operationen vorzeichenlos sind.


\opdef{SUBI}{$X, Y \in \Reg$, $N\in\mathds{Z}$}{0x5A}{RRN}
\glqq Subtract Immediate\grqq.
Funktioniert wie \opref{SUB} aber $N$ ist eine direkt angegebene Zahl
(kein Register).

\paragraph{Beispiel}
Folgendes Beispiel demonstriert die Verwendung von \opref{SUBI} und zeigt
zugleich einen Fehler.
\begin{lstlisting}
  SET  R1 50     # $R1 \gets 50$
  SUBI R2 R1 30  # $R2 \gets (R1 - 30) = 20$
  SUBI R2 R1 R1  # Fehler! da $R1 \not\in \mathds{Z}$
\end{lstlisting}



\opdef{MUL}{$X, Y \in \Reg$}{0x60}{RR0}
\glqq Multiply\grqq. 
Multipliziert die Inhalte der Register $X$ und $Y$ und speichert das Ergebnis in
die Spezialregister \texttt{HI} und \texttt{LO}. Diese zwei Spezialregister
werden als eine 64-Bit Einheit betrachtet, wobei jedes eine Hälfte des
64-Bit Ergebnisses enthält.
Dabei werden die höchstwertigen 32 Bit des Ergebnisses in das Register
\texttt{HI}\index{HI@\texttt{HI}}
und die 32 niedrigstwertigen Bits des Ergebnisses in das Register
\texttt{LO}\index{LO@\texttt{LO}} gespeichert.
Siehe auch die Tabelle \ref{tab:Spezialregister} auf der Seite
\pageref{tab:Spezialregister}.

Falls das Ergebnis der Multiplikation gänzlich in den 32 Bit des Registers
\texttt{LO} passt, wird das Register \texttt{HI} trotzdem auf Null gesetzt.

Äquivalenter algebraischer Ausdruck:
\[
    (HI, LO) \gets X \cdot Y
\]

\paragraph{Beispiel} Der folgende Code demonstriert die Verwendung der
\texttt{MUL} Instruktion.
\begin{lstlisting}
  SET  R1 4   # $R1 \gets 4$
  SET  R2 5   # $R2 \gets 5$
  MUL  R1 R2  # $HI \gets 0$
              # $LO \gets 20$

  SET  R1 0xAAAAAAAA
  MUL  R1 R1  # $R1^{2}$
              # HI = 0x71C71C70
              # LO = 0xE38E38E4 
  COPY R2 LO  # $R1^{2} \bmod 2^{32}$ 
\end{lstlisting}



\opdef{MULU}{$X, Y\in \Reg$}{0x61}{RR0}
\glqq Multiply Unsigned\grqq.
Funktioniert wie die Instruktion \opref{MUL} mit dem Unterschied, dass die
Multiplikationoperanden $X$ und $Y$ vorzeichenlos behandelt werden. 



\opdef{MULI}{$X \in \Reg$, $N\in\mathds{Z}$}{0x62}{RNN}
\glqq Multiply Immediate\grqq.
Multipliziert den Inhalt des Registers $X$ mit der ganzen Zahl $N$ und speichert
das $64$-Bit Ergebnis in die Register \texttt{HI} und \texttt{LO}, die als ein
einziges $64$-Register betrachtet werden: \texttt{HI} enthält die ersten $32$
Bits (die höchstwertigen) und \texttt{LO} die letzten 32 Bits (die
niedrigstwertigen).
Siehe auch die Instruktion \opref{MUL}.



\opdef{DIV}{$X, Y \in \Reg$}{0x68}{RR0}
\glqq Divide\grqq, ganzzahlige Division.
Dividiert den Inhalt des Registers $X$ durch den Inhalt des Registers $Y$ und
speichert den Quotient in das Register \texttt{HI} und den Rest in das Register
\texttt{LO}.
Nach der Ausführung gilt
\[
    X = Y \cdot HI + LO
\]
Algebraisch ausgedrückt:
\begin{align*}
  HI & \gets \left\lfloor X/Y \right\rfloor \\
  LO & \gets X \bmod Y
\end{align*}
$\lfloor x \rfloor$ bedeutet in diesem Kontext, dass $x$ auf die betragsmässig
nächstkleinste ganze Zahl gerundet wird, oder die Nachkommastellen von $x$
werden abgeschnitten.


\paragraph{Beispiel}
Der folgende Code demonstriert die Verwendung von \texttt{DIV}.
\begin{lstlisting}
  SET R1 10   # $R1 \gets 10$
  SET R2  3   # $R2 \gets 3$
  DIV R1 R2   # $HI \gets 3$
              # $LO \gets 1$
\end{lstlisting}


\opdef{DIVU}{$X, Y \in \Reg$}{0x69}{RR0}
\glqq Divide Unsigned\grqq.
Funktioniert wie \opref{DIV} mit dem Unterschied, dass ganzzahlige
vorzeichenlose Division durchgeführt werden. Die Ergebnis-Register \texttt{HI}
und \texttt{LO} enthalten entsprechend vorzeichenlose Werte.



\opdef{DIVI}{$X \in \Reg$, $N\in\mathds{Z}_{\setminus 0}$}{0x6A}{RNN}
\glqq Divide Immediate\grqq.
Dividiert den Inhalt des Registers $X$ durch die feste ganze Zahl $N$ und
speichert den Quotient in das Register \texttt{HI} und den Rest in das Register
\texttt{LO}.
$N$ nimmt Werte aus dem Intervall $[-2^{15}, 2^{15}-1] \setminus 0$.
Nach der Durchführung der Division gilt:
\[
    X = HI \cdot N + LO 
\]
\paragraph{Beispiel}
Der folgende Code demonstriert die Verwendung von \texttt{DIVI}.
\begin{lstlisting}
  SET  R1 10   # $R1 \gets 10$
  DIVI R1  3   # $HI \gets 3$
               # $LO \gets 1$
\end{lstlisting}





\opdef{MOD}{$X, Y, Z \in \Reg$}{0x70}{RRR}
Modulo Operation.
Berechnet den Rest der Division $Y/Z$ und speichert den Rest in das Register
$X$. Dabei ist das Ergebnis immer positiv.
Äquivalent zu den Instruktionen:
\begin{lstlisting}
  DIVU Y Z
  COPY X LO
\end{lstlisting}

Algebraische Schreibweise:
\[
    X \gets Y \bmod Z
\]
oder
\[
    X \gets \left(
      Y - \left\lfloor \frac{Y}{Z}  \right\rfloor \cdot Z
      \right)
\]


\opdef{MODI}{$X, Y \in \Reg$, $N \in \mathds{N}$}{0x72}{RRN}
\glqq Modulo Immediate\grqq. Analog zur Instruktion \opref{MOD}, berechnet
\texttt{MODI} den Rest der ganzzahligen Division $Y / N$ und speichert ihn in
das Register $X$. Der Unterschied liegt darin, dass $N$ eine fest angegebene
natürliche Zahl ist.
\[
    X \gets Y \bmod N
\]


\opdef{ABS}{$X, Y \in \Reg$}{0x78}{RR0}
\glqq Absolute\grqq.
Speichert den absoluten Wert des Registers $Y$ in das Register $X$.
Algebraisch ausgedrückt:
\[
    X \gets
    \begin{cases}
      Y            & \text{ falls } Y \geq 0 \\
      (-1) \cdot Y & \text{ falls } Y < 0
    \end{cases}
\]



\opdef{NEG}{$X, Y \in \Reg$}{0x80}{RR0}
\glqq Negate\grqq.
Wechselt das arithmetische Vorzeichen des Registers $Y$ und speichert das
Ergebnis in das Register $X$. Entspricht der Zweierkomplement Bildung.
Algebraische Schreibweise:
\[
    X \gets \big( (-1) \cdot Y \big)
\]
Um eine bitweise Inversion zu erreichen (Einerkomplement), siehe die
Instruktion \opref{NOT}.



\opdef{INC}{$X \in \Reg$}{0x81}{R00}
\glqq Increment\grqq.
Inkrementiert den Inhalt des Registers $X$.
\[
    X \gets ( X + 1 )
\]



\opdef{DEC}{$X \in \Reg$}{0x82}{R00}
\glqq Decrement\grqq.
Dekrementiert den Inhalt des Registers $X$.
\[
    X \gets ( X - 1 )
\]



