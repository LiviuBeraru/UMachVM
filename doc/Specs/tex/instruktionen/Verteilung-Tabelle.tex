\begin{table}
\index{Befehlsraum!Verteilungstabelle}
  \centering
  \begin{tabular}{|c|l|}                        \hline
    Maschinencodes   & Kategorie              \\\hline\hline
    \texttt{00 - 0F} & Kontrollbefehle        \\
    \texttt{10 - 4F} & Lade-/Speicherbefehle  \\
    \texttt{50 - 8F} & Arithmetische Befehle  \\
    \texttt{90 - AF} & Logische Befehle       \\
    \texttt{B0 - BF} & Vergleichsbefehle      \\
    \texttt{C0 - DF} & Sprungbefehle          \\
    \texttt{E0 - EF} & Unterprogrambefehle    \\
    \texttt{F0 - FF} & Systembefehle          \\\hline
  \end{tabular}
  \caption[Verteilung des Befehlsraums]
          {Verteilung des Befehlsraums nach Befehlskategorien.
          Die Zahlen sind im Hexadezimalsystem angegeben.}
  \label{tab:Befehlraumverteilung}
\end{table}


