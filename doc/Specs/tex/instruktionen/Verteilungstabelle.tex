\begin{table}
\index{Befehlsraum!Verteilungstabelle}
  \centering
  \begin{tabular}{|c|l|}                        \hline
    Maschinencodes   & Kategorie              \\\hline\hline
    \texttt{00 - 1F} & Kontrollbefehle        \\
    \texttt{20 - 3F} & Lade-/Speicherbefehle  \\
    \texttt{40 - 5F} & Arithmetische Befehle  \\
    \texttt{60 - 7F} & Logische Befehle       \\
    \texttt{80 - 9F} & Vergleichsbefehle      \\
    \texttt{A0 - BF} & Sprungbefehle          \\
    \texttt{C0 - DF} & Unterprogrambefehle    \\
    \texttt{E0 - FF} & Systembefehle          \\\hline
  \end{tabular}
  \caption[Verteilung des Befehlsraums]
          {Verteilung des Befehlsraums nach Befehlskategorien.
          Die Zahlen sind im Hexadezimalsystem angegeben.}
  \label{tab:Befehlraumverteilung}
\end{table}


