\section{Kontrollinstruktionen}

\opdef{NOP}{keine}{0x00}{000}
Diese Instruktion (\glqq No Operation\grqq) bewirkt nichts.
Der Sinn dieser Instruktion ist, den Maschinencode mit Nullen füllen zu können,
ohne dabei die gesamte Ausführung zu beeinflussen, außer, Zeitlupen zu schaffen. 


\opdef{RST}{keine}{0x01}{000}
Setzt alle Register außer dem Befehlszähler auf Null.


\opdef{CRM}{$N \in \mathds{N}$}{0x02}{NNN}
\glqq Change Run Mode\grqq.
Setzt das \gls{Betriebsmodus}.\index{Betriebsmodus}
Dabei ist das Parameter einer der folgenden konstanten Werten:
\begin{center}
  \begin{tabular}[l]{ll}
    0 & Normalmodus \\
    1 & Einzelschrittmodus \\
  \end{tabular}
\end{center}
Siehe Abschnitt \ref{subsec:Betriebsmodi} auf Seite
\pageref{subsec:Betriebsmodi}.



\opdef{CSM}{$N \in \mathds{N}$}{0x03}{NNN}
Change System Mode.



\opdef{DIE}{keine}{0x04}{000}
Die Maschine ausschalten.



\opdef{RSR}{keine}{0x05}{000}
\glqq Resurrect\grqq.
Die Maschine neustarten.



\opdef{AUTSM}{keine}{0x06}{000}
\glqq Become autistic\grqq.
Bewirkt, dass alle Lese- und Schreibbefehle, die sich nicht ausschließlich auf
Register beziehen, wirkungslos sind. Praktisch wird die Kommunikation mit dem
Bussystem ausgeschaltet.


\opdef{SOCL}{keine}{0x07}{000}
\glqq Become social\grqq. Schaltet die Kommunikation mit dem Bussystem ein.


\opdef{HATE}{keine}{0x08}{000}
\glqq Hate\grqq. Schaltet alle Schutzmechanismen ein.


\opdef{TRST}{keine}{0x09}{000}
\glqq Trust\grqq. Schaltet alle Schutzmechanismen aus.


\opdef{ZMB}{keine}{0x0A}{000}
\glqq Become a zombie\grqq.
Schaltet alle Registeränderungen aus. Nach der Ausführung dieses Befehls, alle
Operationen, die einen Register modifizieren sollen, sind wirkungslos.
Die Maschine ändert ihren Zustand nicht mehr, außer, dass sie weitere Befehle
liest.


\opdef{ALIV}{keine}{0x0B}{000}
\glqq Become alive\grqq.
Nach der Ausführung dieses Befehls, alle Register können wie normal modifiziert
werden.


