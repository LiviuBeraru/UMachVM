\section{Unterbrechungen}
\label{sec:Unterbrechungen}
\index{Unterbrechungen}


Es gibt drei verschiedene Unterbrechungsursachen:
\begin{enumerate}
 \item Hardware-Unterbrechungen, die von den Input-Ports verursacht werden
       können,
 \item geziehlte Software-Unterbrechungen die vom der Software angefordert
       werden können, und
 \item Unterbrechungen, die aufgrund von Fehlern geschehen und eine rudimentäre
       Fehlerbehandlung bei schwerwiegenden Fehlern wie die Nulldivision
       ermöglichen sollen.
\end{enumerate}

Bei jeder Unterbrechung wird der PC auf den Stack gepuscht und bei Rückkehr
automatisch gepopt. Daher sind bei einer Unterbrechung zwei Dinge zu beachten:

Der Stackpointer muss auf der Ausgangsposition stehen. Verwendete Register
müssen gesichert und vor der Rückkeht wieder hergestellt werden.


\subsection{Hardware-Unterbrechungen}
\label{subsec:Hardware-Unterbrechungen}

Wird eine Hardware-Unterbrechung ausgelöst, so wird ohne Zutun des
Programmierers der aktuelle Programmablauf unterbrochen und der PC auf den Stack
gepuscht. Weiterhin wird die Nummer des für die Hardware-Unterbrechung
verantwortlichen Ports im Spezialregister HIR hinterlegt. Dies dient dazu um den
passenden Eintrag in der Unterbrechungstabelle aufzufinden. Dieser wird in den
PC geladen und der dort Adressierte Programmcode zur Behandlung der
Unterbrechung wird ausgeführt. 

Ein Hardwareinterrupt entspricht somit einem INT Portnummer, wobei die CPU
zusätzlich in einen Modus versetzt welcher jede weitere Hardwareunterbrechung
ignoriert. Dieser Modus wird automatisch mit dem IRET Befehl zurückgesetzt,
welcher dazu Dient aus der Unterbrechungsbehandlung zurückzukehren.

Das Ende einer Hardwareunterbrechung kann entweder vom Verursacher als für
beendet erklärt werden, indem er das HIR nach dem Ausführen einer IN Instruktion
auf Null setzt. Oder aber durch die Unterbrechungsbehandlung selbst, indem
einfach IRET aufgerufen wird. Dieser Befehl setzt das HIR auch auf Null zurück,
falls dies noch nicht der Fall sein sollte. Auserdem wird dem Unterbrecher das
Ende der Unterbrechungsbehandlung signalisiert.

Dies ist besonderst von Bedeutung, wenn noch Daten am Port anstehen sollten.
Diese werden im Falle eines vorzeitigigen Beendens der Unterbrechungsbehandlung
verworfen.

\subsection{Software-Unterbrechungen}



\subsection{Fehlerbedingte Unterbrechungen}

Im Normalfall wird eine Instruktion ohne Weiteres ausgeführt. In Ausnahmefällen
-- wenn die Instruktion nicht ausgeführt werden kann, meistens wegen ungültigen
Parametern -- wird eine Ausnahmesituation\index{Ausnahmesituation} signalisiert
und der Programmfluss unterbrochen.


Mit jeder Ausnahmesituation wird eine Unterbrechungsnummer assoziiert. Diese
Nummer wird als Index in der Unterbrechungstabelle verwendet, um die
Speicheradresse einer Unterbrechungsroutine zu finden (siehe den Abschnitt
\ref{subsubsec:Unterbrechungstabelle}, besonders die Tabelle
\ref{tab:Unterbrechungstabelle} auf der Seite
\pageref{tab:Unterbrechungstabelle}). Ist diese Adresse ungleich Null, so wird
das \texttt{PC} Register entsprechend gesetzt und die Unterbrechungsroutine
ausgeführt. Wird keine solche Routine gefunden, so wird die gesamte
Programmausführung unterbrochen und die Maschine hält an.


\paragraph{Beispiel}
Zum Beispiel, die \opref{DIV} Instruktion (Division) erzeugt die Unterbrechung
mit Nummer 16, falls ihr zweites Argument gleich Null ist (siehe auch die
Tabelle \ref{tab:Unterbrechungstabelle} auf der Seite
\pageref{tab:Unterbrechungstabelle}). Wenn die Unterbrechung erzeugt wird,
schaut die Maschine in der Unterbrechungstabelle an der Position 16 nach. Ist
der Eintrag ungleich Null, so wird der Eintrag als Adresse einer Routine
behandelt und dorthin gesprungen: das Register \texttt{PC} wird zuerst auf den
Stack gespeichert und dann gleich dem Tabelleneintrag gesetzt. Ist der Eintrag
dagegen Null, so hält die Maschine an.

