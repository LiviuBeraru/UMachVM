\section{Unterbrechungen}
\label{sec:Unterbrechungen}
\index{Unterbrechungen}

Im Normalfall wird eine Instruktion ohne Weiteres ausgeführt. In Ausnahmefällen
-- wenn die Instruktion nicht ausgeführt werden kann, meistens wegen ungültigen
Parametern -- wird eine Ausnahmesituation\index{Ausnahmesituation} signalisiert
und der Programmfluss unterbrochen.


Mit jeder Ausnahmesituation wird eine Unterbrechungsnummer assoziiert. Diese
Nummer wird als Index in der Unterbrechungstabelle verwendet, um die
Speicheradresse einer Unterbrechungsroutine zu finden (siehe den Abschnitt
\ref{subsubsec:Unterbrechungstabelle}, besonders die Tabelle
\ref{tab:Unterbrechungstabelle} auf der Seite
\pageref{tab:Unterbrechungstabelle}). Ist diese Adresse ungleich Null, so wird
das \texttt{PC} Register entsprechend gesetzt und die Unterbrechungsroutine
ausgeführt. Wird keine solche Routine gefunden, so wird die gesamte
Programmausführung unterbrochen und die Maschine hält an.


\paragraph{Beispiel}
Zum Beispiel, die \opref{DIV} Instruktion erzeugt die Unterbrechung mit Nummer
0, falls ihr zweites Argument gleich Null ist. Wenn die Unterbrechung erzeugt
wird, schaut die Maschine in der Unterbrechungstabelle an der Position 0. Ist
der Eintrag ungleich Null, so wird der Eintrag als Adresse einer Routine
behandelt und dorthin gesprungen: das Register \texttt{PC} wird
zuerst auf den Stack gespeichert und dann gleich dem Tabelleneintrag gesetzt.

