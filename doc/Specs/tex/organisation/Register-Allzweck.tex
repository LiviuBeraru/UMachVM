\subsection{Allzweckregister}
\index{Register!Allzweckregister}
\index{Allzweckregister}

Es gibt 32 Allzweckregister, welche dem Programmierer für generelle Zwecke zur
Verfügung stehen. Alle Allzweckregister werden beim Hochfahren der Maschine mit
dem Wert Null (\texttt{0x00}) initialisiert. Außer dieser Initialisierung
verändert die Maschine den Inhalt der Allzweckregister nur in Folge einer
expliziten Instruktion. 

Die Allzweckregister werden auf Maschinencode-Ebene fortlaufend mit den 
Registernummern 1 bis 32 gekennzeichnet (\texttt{0x01} bis einschliesslich
\texttt{0x20} im Hexadezimalsystem). Die Assemblernamen sind entsprechend
$R1, R2\dots$ bis $R32$. Die Zahl nach dem Buchstaben $R$ ist im Dezimalsystem
angegeben und ist fester Bestandteil des Registernamens.

\begin{center}
  \newcolumntype{T}{>{\ttfamily}c}
  \begin{tabular}{l||*{5}{T|}}
    Assemblername  & R1   & R2   & R3   & \dots & R32 \\
    Registernummer & 0x01 & 0x01 & 0x03 & \dots & 0x20
  \end{tabular}
\end{center}

Das Register mit der Registernummer Null, oder auch
Null-Register\index{Null-Register}\index{Register!Null-Register}, ist ein
Spezialregister, welches immer den Wert Null hat und nicht beschreibbar ist.
Der Assemblername ist \texttt{ZERO}. (siehe auch Abschnitt
\ref{subsec:Spezialregister} auf Seite \pageref{subsec:Spezialregister}).

Beispiel für die Verwendung von Registernamen:
\begin{center}
  \begin{tabular}{|l||*{4}{c|}}                                       \hline
    Assembler     & \texttt{ADD}  & \texttt{R1}   & \texttt{R2}   &
                    \texttt{R3}                                      \\
    Maschinencode & \texttt{0x50} & \texttt{0x01} & \texttt{0x02} &
                    \texttt{0x03}                                    \\\hline
    Bytes         & erstes Byte   & zweites Byte  & drittes Byte  & 
                    viertes Byte                                     \\\hline
    Algebraisch   & \multicolumn{4}{c|}{$R_{1} \gets R_{2} + R_{3}$} \\\hline
  \end{tabular}
\end{center}


