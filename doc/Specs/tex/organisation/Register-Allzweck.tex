\subsection{Allzweckregister}
\index{Register!Allzweckregister}
\index{Allzweckregister}
Es gibt 32 Allzweckregister, die dem Programmierer für allgemeine Zwecke zur
Verfügung stehen.
Diese 32 Register werden beim Hochfahren der Maschine auf Null (\texttt{0x00})
gesetzt. Außer dieser Initialisierung, verändert die Maschine den Inhalt der
Allzweckregister nur auf explizite Anfrage, bzw. infolge einer Instruktion. 

Die 32 Register werden auf Maschinencode-Ebene von 1 bis 32 nummeriert
(\texttt{0x01} bis einschliesslich \texttt{0x20} im Hexadezimalsystem).
Diese Nummer ist der Maschinenname des Registers.
Auf Assembler-Ebene, werden sie mit den Namen $R1, R2\dots$ bis $R32$
angesprochen (Assemblername).
Die Zahl nach dem Buchstaben $R$ ist im Dezimalsystem angegeben
und ist fester Bestandteil des Registernamens.

\begin{center}
  \newcolumntype{T}{>{\ttfamily}c}
  \begin{tabular}{l||*{5}{T|}}
    Assemblername & R1   & R2   & R3   & \dots & R32 \\
    Maschinenname & 0x01 & 0x01 & 0x03 & \dots & 0x20
  \end{tabular}
\end{center}

Register mit Nummer Null ($R0$) gibt es nicht und die Verwendung des
Null-Registers\index{Null-Register} wird von der Maschine als Fehler gemeldet.

Beispiel für die Verwendung von Registernamen:
\begin{center}
  \begin{tabular}{|l||*{4}{c|}}                                       \hline
    Assembler     & \texttt{ADD}  & \texttt{R1}   & \texttt{R2}   &
                    \texttt{R3}                                      \\
    Maschinencode & \texttt{0x50} & \texttt{0x01} & \texttt{0x02} &
                    \texttt{0x03}                                    \\\hline
    Bytes         & erstes Byte   & zweites Byte  & drittes Byte  & 
                    viertes Byte                                     \\\hline
    Algebraisch   & \multicolumn{4}{c|}{$R_{1} \gets R_{2} + R_{3}$} \\\hline
  \end{tabular}
\end{center}


