\section{Der Speicher}
\index{Speichermodell}

\subsection{Adressierungsarten}
\label{subsec:Adressierungsarten}
\index{Adressierungsarten}

Als RISC-orientierte Maschine, greift die UMach lediglich in zwei Situationen
auf den Speicher zu: zum Schreiben von Registerinhalten in den Speicher
(Schreibzugriff) und zum Lesen von Speicherinhalten in einen Register
(Lesezugriff).  Die \gls{Adressierungsart} beschreibt dabei, wie der Zugriff
auf den Speicher erfolgen sollte, bzw. wie die angesprochene Speicheradresse
angegeben wird. Anders ausgedrückt, beantwortet die Adressierungsart die Frage
\glqq wie kann eine Instruktion der Maschine eine Adresse angeben?\grqq. 

Die UMach Maschine kennt eine einzige Adressierungsart: die indirekte
Adressierung, die unten beschrieben wird. Die direkte Adressierung, die aus
einer direkten Angabe eines Speicheradresse besteht, wird von der indirekten
Adressierung überdeckt.

\subsubsection{Indirekte Adressierung}
\index{Adressierung!Indirekte}

Die indirekte Adressierung verwendet zwei Register $B$ und $I$, die
von der Maschine verwendet werden, um die endgültige Adresse zu berechnen:
Eine Instruktion, die diese Adressierung verwendet, hat also das Format RRR
(siehe auch \ref{RRR}).
\begin{center}
  \begin{tabular}{|*{4}{c|}|c|} \hline
    erstes Byte    & zweites Byte  & drittes Byte  & viertes Byte & Algebraisch
\\\hline\hline
    Ladebefehl     & $R$  & $B$  & $I$  & $R \gets mem(B + I)$ \\\hline
    Speicherbefehl & $R$  & $B$  & $I$  & $R \to   mem(B + I)$ \\\hline
  \end{tabular}
\end{center}
Die fünfte Spalte gibt jeweils den äquivalenten algebraischen Ausdruck wieder.
$mem(x)$ steht dabei für den Inhalt der Adresse $x$.


Die zweite Zeile (Ladebefehl) bedeutet, dass die UMach Maschine die Inhalte der
Register $B$ und $I$ aufaddieren soll, diese Summe als Adresse im
Speicher zu verwenden und den Inhalt an dieser Adresse in den Register $R$ zu
kopieren.

Die dritte Zeile (Speicherbefehl) bedeutet: die Maschine soll den Inhalt des
Registers $R$ an die Adresse $B + I$ schreiben.

Üblicherweise enthält $B$ eine Startadresse und $I$ einen Versatz oder Index zur
Adresse in $B$.

Vorteil der indirekten Adressierung ist, dass sie $2^{33} - 1$ mögliche Adressen
ansprechen kann. Nachteil ist, dass zwei oder mehrere Instruktionen gebraucht
werden, um diese Adressierung zu verwenden, denn die Register $B$ und $I$ erst
entsprechend geladen werden müssen.

Die Register $R$, $B$ und $I$ stehen für beliebige Register.

\subsection{Datentypen}
\label{subsec:Datentypen}
\index{Datentypen}

Die UMach Maschine kennt 3 Datentypen:
\begin{enumerate}
  \item Byte (8 Bit lang) 
  \item Half (2 Byte)
  \item Word (4 Byte)
\end{enumerate}



\subsection{Der Stack}
\label{subsec:Stack}
\index{Stack}

