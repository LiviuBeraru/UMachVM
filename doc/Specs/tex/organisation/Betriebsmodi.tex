\subsection{Betriebsmodi}
\label{subsec:Betriebsmodi}
\index{Betriebsmodus}

Die Umach VM kann in zwei verschiedenen Betriebsmodi oder auch
Betriebsarten betrieben werden:

\begin{enumerate}
  \item Normalmodus
  \item Debugmodus
\end{enumerate}

\paragraph{Normalmodus} Die virtuelle Maschine führt alle Instruktionen
ohne Unterbrechung aus. Nach der Ausführung befindet sich die Maschine in einem
Wartezustand, falls sie nicht ausdrücklich ausgeschaltet wird.

\paragraph{Debugmodus} Die virtuelle Maschine führt immer nur eine
einzige Instruktion aus und bleibt dannach stehen. Um mit der folgenden
Instruktion fortzufahren wird immer ein externen Signal benötigt.
Dieser Modus soll dem Entwickler erlauben, ein Programm schrittweise zu debuggen.
Siehe dazu Kapitel \ref{chap:Debugging}, Seite \pageref{chap:Debugging}.

Der Modus kann vor dem Hochfahren der Maschine festgelegt, jedoch im
Betrieb nicht mehr geändert werden. Wird kein Modus angegeben, startet die
Maschine im Normalmodus.



