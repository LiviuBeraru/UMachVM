\subsection{Betriebsmodi}
\label{subsec:Betriebsmodi}
\index{Betriebsmodus}

Ein \gls{Betriebsmodus} bezieht sich auf die Art, wie die UMach VM grundsätzlich
läuft. Die UMach VM kann in einem der folgenden Betriebsmodi laufen:

\begin{enumerate}
  \item Normalmodus
  \item Debugmodus
\end{enumerate}

Voreingestellt ist der Normalmodus. Der Modus kann vor dem Hochfahren der
Maschine festgelegt werden und kann während der Lauzeit der Maschine nicht mehr
geändert werden.

\paragraph{Normalmodus} Die virtuelle Maschine führt ohne Unterbrechung ein
Programm aus. Nach der Ausführung befindet sich die Maschine in einem
Wartezustand, falls sie nicht ausdrücklich ausgeschaltet wird.

\paragraph{Debugmodus} Die virtuelle Maschine führt immer eine einzige
Instruktion aus und nach der Ausführung wartet sie auf einen externen Signal um
mit der nächsten Instruktion fortzufahren. Dieser Modus soll dem Entwickler
erlauben, ein Programm schrittweise zu debuggen.
Siehe dazu Kapitel \ref{chap:Debugging}, Seite \pageref{chap:Debugging}.

