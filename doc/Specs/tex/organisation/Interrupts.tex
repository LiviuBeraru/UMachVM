\section{Interrupts}
\label{sec:Interrupts}
\index{Interrupt}

Ein Interrupt ist eine Unterbrechung der Programmausführung unter bestimmten
Umständen. Die UMach Maschine verwendet die folgenden Arten von Interrupts:

\begin{enumerate}
 \item Automatische Interrupts, oder Hardware-Interrupts, die aufgrund von
       Fehlern geschehen und eine rudimentäre Fehlerbehandlung bei
       schwerwiegenden Fehlern, wie die Nulldivision, ermöglichen sollen.
 \item Software-Interrupts, die vom der Software angefordert
       werden können.
\end{enumerate}

Jeder Interrupt hat eine bestimmte Nummer, die ganzzahlige Werte von 0 bis 63
annimmt. Die Interruptnummer wird zur Berechnung eines Index in der
Interrupttabelle verwendet, um die Speicheradresse einer Interruptroutine zu
finden (siehe den Abschnitt \ref{subsubsec:Interrupttabelle}, besonders die
Tabelle \ref{tab:Interrupttabelle} auf der Seite
\pageref{tab:Interrupttabelle}). Ist diese Adresse ungleich Null, so wird das
\texttt{PC} Register auf diese Adresse gesetzt und die Interruptroutine wird
ausgeführt. Wird keine solche Routine gefunden, so wird die gesamte
Programmausführung unterbrochen und die Maschine hält an.

Die Interruptroutinen müssen vom Hauptprogramm als Sprungadressen in der
Interrupttabelle gesetzt werden. Dabei ist die Struktur der Interrupttabelle zu
beachten, so wie sie im Abschnitt \ref{subsubsec:Interrupttabelle} auf Seite
\pageref{subsubsec:Interrupttabelle} beschrieben wird.



\subsection{Automatische Interrupts}

Im Normalfall wird eine Instruktion ohne Weiteres ausgeführt. In Ausnahmefällen
-- wenn die Instruktion nicht ausgeführt werden kann, meistens wegen ungültigen
Parametern -- wird eine Ausnahmesituation\index{Ausnahmesituation} durch eine
Interruptnummer signalisiert und der Programmfluss unterbrochen.




\paragraph{Beispiel}
Zum Beispiel, die \opref{DIV} Instruktion (Division) erzeugt den Interrupt mit
Nummer 1, falls ihr zweites Argument gleich Null ist (siehe auch die Tabelle
\ref{tab:Interrupttabelle} auf der Seite \pageref{tab:Interrupttabelle}). Wenn
der Interrupt bearbeitet wird, schaut die Maschine in der Interrupttabelle an
der Position 1 nach. Ist der Eintrag ungleich Null, so wird der Eintrag als
Adresse einer Routine behandelt und dorthin gesprungen: das Register \texttt{PC}
wird zuerst auf den Stack gesichert und dannach gleich dem Tabelleneintrag
gesetzt. Ist der Eintrag dagegen Null, so hält die Maschine an.



\subsection{Software-Interrupts}

Eine Unterbrechung der Programmausführung kann auch absichtlich erzeugt werden.
Dafür verwendet man den Befehl \opref{INT}. Wenn ein Software-Interrupt
generiert wird, verhält sich die Maschine wie im Falle eines automatischen
Interrupts.


\subsection{Interruptnummer}

% 0  - 15 kern (8 arithmetisch, 8 Befehl bezogen)
% 16 - 31 speicher (8 Argumente, 8 zugriff)
% 32 - 47 io (8 argumente 8 ?)
% 48 - 63 system (8 ? und 8 intern (Maschine/Implementierung kaputt))

\begin{longtable}{>{\ttfamily}ll}
\caption{Interruptnummer}
\\\toprule
{\rmfamily Nummer} & Beschreibung \\
\midrule
\endfirsthead
 0   & Standardinterrupt: Maschine ausschalten \\\label{tab:Interrupttabelle}
 1   & Division durch Null       \\
 2   & Arithmetischer Überlauf   \\
 8   & Ungültiger Befehlscode    \\
 9   & Ungültige Registernummer  \\
\midrule
 16  & Ungültige Speicheradresse  \\
 17  & Zugriffsverletzung (segfault) \\
 26  & Stack Überlauf             \\
\midrule
 32  & I/O Port existiert nicht   \\
\midrule
 56  & Interner Fehler            \\
 63  & Syscall, Aufruf an das Betriebssystem \\
\bottomrule
\end{longtable}

Siehe auch den Abschnitt \ref{subsec:Speicherfehler}
\nameref{subsec:Speicherfehler}, Seite \pageref{subsec:Speicherfehler}.
