\section{Interrupts}
\label{sec:Interrupts}
\index{Interrupts}

Ein Interrupt ist eine Unterbrechung der Programmausführung unter bestimmten
Umständen. Die UMach Maschine verwendet die folgenden Arten von Interrupts:

\begin{enumerate}
 \item Automatische Interrupts, oder Hardware-Interrupts, die aufgrund von
       Fehlern geschehen und eine rudimentäre Fehlerbehandlung bei
       schwerwiegenden Fehlern, wie die Nulldivision, ermöglichen sollen.
 \item Software-Interrupts, die vom der Software angefordert
       werden können.
\end{enumerate}

Jeder Interrupt hat eine bestimmte Nummer, die ganzzahlige Werte von 0 bis 64
annimmt. Die Interruptnummer wird als Index in der Interrupttabelle
verwendet, um die Speicheradresse einer Interruptroutine zu finden (siehe den
Abschnitt \ref{subsubsec:Interrupttabelle}, besonders die Tabelle
\ref{tab:Interrupttabelle} auf der Seite \pageref{tab:Interrupttabelle}). Ist
diese Adresse ungleich Null, so wird das \texttt{PC} Register auf diese Adresse
gesetzt und die Interruptroutine wird ausgeführt. Wird keine solche Routine
gefunden, so wird die gesamte Programmausführung unterbrochen und die Maschine
hält an.

Die Interruptroutinen müssen vom Hauptprogramm als Sprungadressen in der
Interrupttabelle gesetzt werden. Jede solche Interruptroutine sollte folgendes
beachten:
\begin{itemize}
 \item Bevor die Routine aufgerufen wird, sichert die Maschine auf den Stack
       nur das Register \texttt{PC}. Andere Register wie \texttt{SP} müssen
       von der Routine selbst gesichert und wiederhergestellt werden.
\end{itemize}


\subsection{Automatische Interrupts}

Im Normalfall wird eine Instruktion ohne Weiteres ausgeführt. In Ausnahmefällen
-- wenn die Instruktion nicht ausgeführt werden kann, meistens wegen ungültigen
Parametern -- wird eine Ausnahmesituation\index{Ausnahmesituation} durch eine
Interruptnummer signalisiert und der Programmfluss unterbrochen.




\paragraph{Beispiel}
Zum Beispiel, die \opref{DIV} Instruktion (Division) erzeugt den Interrupt mit
Nummer 1, falls ihr zweites Argument gleich Null ist (siehe auch die Tabelle
\ref{tab:Interrupttabelle} auf der Seite \pageref{tab:Interrupttabelle}). Wenn
der Interrupt bearbeitet wird, schaut die Maschine in der Interrupttabelle an
der Position 1 nach. Ist der Eintrag ungleich Null, so wird der Eintrag als
Adresse einer Routine behandelt und dorthin gesprungen: das Register \texttt{PC}
wird zuerst auf den Stack gesichert und dannach gleich dem Tabelleneintrag
gesetzt. Ist der Eintrag dagegen Null, so hält die Maschine an.



\subsection{Software-Interrupts}

Eine Unterbrechung der Programmausführung kann auch absichtlich erzeugt werden.
Dafür verwendet man den Befehl \opref{INT}. Wenn ein Software-Interrupt
generiert wird, verhält sich die Maschine wie im Falle eines automatischen
Interrupts.


\subsection{Interruptnummer}


\begin{longtable}{>{\ttfamily}ll}
\caption{Interruptnummer}
\label{tab:Interrupttabelle}
\\\toprule
{\rmfamily Nummer} & Beschreibung \\
\midrule
\endhead
 0   & Standardinterrupt: Maschine ausschalten   \\
 1   & Division durch Null       \\
 2   & Arithmetischer Überlauf   \\
 8   & Ungültiger Befehl         \\
 9   & Ungültige Registernummer  \\
\midrule
 16  & Ungültige Speicheradresse  \\
 24  & Stack Fehler               \\
 26  & Stack Überlauf             \\
 31  & Speicher unzureichend      \\
\midrule
 32  & I/O Port existiert nicht   \\
\bottomrule
\end{longtable}



Einige Interruptnummer dürften eine Erklärung brauchen:

\paragraph{Stack Fehler}
Ein Stack Fehler liegt vor, wenn Daten aus dem Stack angefordert werden, die
nicht zuvor auf den Stack gepusht wurden.


\paragraph{Stack Überlauf}
Ein Stack Überlauf liegt vor, wenn der Stack-Pointer \texttt{SP} in einem
Speicherbereich zeigt, der nicht mehr zum Stackbereich gehört. Das sind
Adressen, die dem Programm-Bereich gehören, oder sogar niedrigere Adressen.






