\subsection{Spezialregister}
\label{subsec:Spezialregister}
\index{Register!Spezialregister}
\index{Spezialregister}
Die Spezialregister werden von der UMach Maschine für spezielle Zwecke
verwendet, sind aber dem Programmierer sichtbar. Der Inhalt der Spezialregister
kann von der Maschine während der Ausführung eines Programms ohne Einfluss
seitens Programmierers verändert werden.

Nicht alle Spezialregister können durch Instruktionen überschrieben werden
(sind schreibgeschützt).

Die \glspl{Registernummer} der Spezialregister setzen die Nummerierung der
Allzweckregister zwar fort, die Assemblernamen aber nicht: es gibt kein Register
$R33$.
Die Tabelle \ref{tab:Spezialregister} auf Seite \pageref{tab:Spezialregister}
enthält die Liste aller Spezialregister. In der ersten
Spalte steht der Assemblername, so wie er vom Programmierer verwendet wird. In
der zweiten Spalte steht der Maschinenname im Dezimalsystem, so wie er im
Maschinencode steht.
Die dritte Spalte enthält eine kurze Beschreibung und Bemerkungen.
Falls die Beschreibung nicht anders spezifiziert, ist das Register nicht
schreibgeschützt.

\begin{longtable}
{
  >{\ttfamily}p{1cm} 
  >{\ttfamily}p{1cm}
  p{\textwidth-2cm-6\tabcolsep}
}
\caption[Spezialregister]{Liste der Spezialregister}
\label{tab:Spezialregister}
\\\toprule
PC   & 33 & \glqq Instruction Pointer\grqq, oder \glqq Program Counter\grqq.
            Enthält zu jeder Zeit die Adresse
            der nächsten Instruktion. Wird auf Null gesetzt, wenn die Maschine
            hochfährt.
            Wird nach dem Abfangen einer Instruktion in das Register
            \texttt{IR} automatisch inkrementiert.

            Schreibgeschützt.
            \index{PC@\texttt{PC}}
\\
SP   & 34 & \glqq Stack Pointer\grqq.
            Enthält die Speicheradresse des höchsten Eintrag auf dem Stack.
            Wird beim Hochfahren der Maschine auf die maximal erreichbare
            Adresse im Speicher gesetzt.
            Siehe auch \ref{subsubsec:Stack}, Seite \pageref{subsubsec:Stack}.
            \index{SP@\texttt{SP}}
\\
FP   & 35 & \glqq Frame Pointer\grqq.
            Enthält die Startadresse des Stack Frames einer Subroutine
            und unterstützt die Implementierung von Funktionen.
            \index{FP@\texttt{FP}}
\\
IR   & 36 & \glqq Instruction Register\grqq. Enthält die gerade ausgeführte
            Instruktion.

            Schreibgeschützt.
            \index{IR@\texttt{IR}}
\\
STAT & 37 & Enthält Status-Informationen. Diese Informationen sind in den 
            einzelnen Bits dieses Register gespeichert.
            Welche Informationen vorhanden sind liegt an der jeweiligen
            Anwendung.
            \index{STAT@\texttt{STAT}}
\\
ERR  & 38 & \glqq Error\grqq.
            Fehlerregister. Die einzelnen Bits dises Registers geben Auskunft
            über Fehler, die mit der Ausführung des Programms verbunden sind.
            Liegt kein Fehler vor, so ist der Inhalt dieses Registers gleich
            Null. Ist ein bestimmtes Bit gesetzt, so wird dadurch der
            entsprechende Fehler signalisiert. Für eine Liste der verwendeten
            Bits un deren Bedeutung, siehe Tabelle \ref{tab:ERR-register} auf
            der Seite \pageref{tab:ERR-register}.            
            \index{ERR@\texttt{ERR}}
\\
HI   & 39 & \glqq High\grqq.
            Falls eine Multiplikation durchgeführt wird, enthält dieses Register
            die höchstwertigen 32 Bits des Ergebnisses der Multiplikation und
            bildet zusammen mit dem Register \texttt{LO} das volle Ergebnis der
            Multiplikation.

            Falls eine Division durchgeführt wird, enthält dieses Register den
            Quotient der Division.
            \index{HI@\texttt{HI}}
\\
LO   & 40 & \glqq Low\grqq.
            Falls eine Multiplikation durchgeführt wird, enthält dieses Register
            die niedrigstwertigen 32 Bits des Ergebnisses der Multiplikation und
            bildet zusammen mit dem Register \texttt{HI} das volle Ergebnis der
            Multiplikation.

            Falls eine Division durchgeführt wird, enthält dieses Register den
            Rest der Division.
            \index{LO@\texttt{LO}}
\\
CMPR & 41 & \glqq Comparison Result\grqq.
            Enthält das Ergebnis eines Vergleichs.
            Siehe auch Abschnitt \ref{sec:Vergleichsinstruktionen}, Seite
            \pageref{sec:Vergleichsinstruktionen}.
            \index{CMPR (Reg)@\texttt{CMPR} (Reg)}
\\
ZERO & 00 & Enthält die Zahl Null.

            Schreibgeschützt.
            \index{ZERO@\texttt{ZERO}}
\\\bottomrule
\end{longtable}


\subsubsection{Das ERR Register}

Die Tabelle \ref{tab:ERR-register} listet alle Fehler auf, die in dem
\texttt{ERR} Register signalisiert werden können. Zu jedem Fehler gehört ein Bit
im Register. Die Bitstellen werden dabei entsprechend deren Stelligkeiten
durchnummeriert: Bit mit Stelligkeit $2^{0}$ hat Position $0$, Bit mit
Stelligkeit $2^{31}$ hat die Position $31$.

\begin{longtable}{>{\ttfamily}ll}
\caption[ERR Register]{Bedeutung der einzelnen Bits im ERR Register}
\label{tab:ERR-register}
\\\toprule
 0  & Division durch Null
\\\bottomrule
\end{longtable}
