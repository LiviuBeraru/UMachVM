\subsection{Spezialregister}
\index{Register!Spezialregister}
\index{Spezialregister}
Die Spezialregister werden von der UMach Maschine für spezielle Zwecke
verwendet, sind aber dem Programmierer sichtbar. Der Inhalt der Spezialregister
kann von der Maschine während der Ausführung eines Programms ohne Einfluss
seitens Programmierers verändert werden.

Nicht alle Spezialregister können durch Instruktionen überschrieben werden
(sind schreibgeschützt).

Die Maschinennamen der Spezialregister setzen die Nummerierung der
Allzweckregister zwar fort, die Assemblernamen aber nicht: es gibt kein Register
$R33$.
Die Tabelle \ref{tab:Spezialregister} auf Seite \pageref{tab:Spezialregister}
enthält die Liste aller Spezialregister. In der ersten
Spalte steht der Assemblername, so wie er vom Programmierer verwendet wird. In
der zweiten Spalte steht der Maschinenname im Hexadezimalsystem, so wie er im
Maschinencode steht.
Die dritte Spalte enthält eine kurze Beschreibung und Bemerkungen.
Falls die Beschreibung nicht spezifiziert, dass ein Register schreibgeschützt
ist, ist das Register nicht schreibgeschützt.

\begin{longtable}
{
  >{\ttfamily}p{1cm} 
  >{\ttfamily}p{1cm}
  p{\textwidth-2cm-6\tabcolsep}
}
\caption{Liste der Spezialregister}
\label{tab:Spezialregister}\\
IP   & 0x33 & \glqq Instruction Pointer\grqq. Enthält zu jeder Zeit die Adresse
            der nächsten Instruktion. Wird auf Null gesetzt, wenn die Maschine
            hochfährt.
            Wird nach dem Abfangen einer Instruktion in das Register
            \texttt{CIN} automatisch inkrementiert.

            Schreibgeschützt.
            \index{IP@\texttt{IP}}
\\
SP   & 0x34 & \glqq Stack Pointer\grqq.
            Enthält die Speicheradresse des höchsten Eintrag auf dem Stack.
            \index{SP@\texttt{SP}}
\\
FP   & 0x35 & \glqq Frame Pointer\grqq.
            Enthält die Startadresse der lokalen Variablen einer Subroutine
            und unterstützt höhere Programmiersprachen.
            \index{FP@\texttt{FP}}
\\
LIN  & 0x36 & \glqq Last Instruction\grqq. Enthält den Maschinencode der zuletzt
            ausgeführten Instruktion.

            Schreibgeschützt.
            \index{LIN@\texttt{LIN}}
\\
CIN  & 0x37 & \glqq Current Instruction\grqq. Enthält die gerade ausgeführte
            Instruktion.

            Schreibgeschützt.
            \index{CIN@\texttt{CIN}}
\\
NIN  & 0x38 & \glqq Next Instruction\grqq.
            Enthält die nächste Instruktion.
            Während der Ausführung eines Instruktion, zeigt der Register IP auf
            diese Instruktion, aber im Speicher.

            Schreibgeschützt.
            \index{NIN@\texttt{NIN}}
\\
STAT & 0x39 & Enthält Status-Informationen
\\
ERR  & 0x3A & Fehlerregister. Die einzelnen Bits geben Auskunft über Fehler, die
            mit der Ausführung des Programms verbunden sind.
            \index{ERR@\texttt{ERR}}
\\
ERRM & 0x3B & Enthält zusätzliche Informationen zu den Fehler im signalisiert im
            Register ERR.
            \index{ERRM@\texttt{ERRM}}
\\
ZERO & 0x3C & Enthält die Zahl Null.

            Schreibgeschützt.
            \index{ZERO@\texttt{ZERO}}
\\
\end{longtable}

