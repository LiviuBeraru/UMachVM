\subsection{Spezialregister}
\label{subsec:Spezialregister}
\index{Register!Spezialregister}
\index{Spezialregister}
Die Spezialregister werden von der UMach Maschine für gesonderte Zwecke
verwendet, sind aber dem Programmierer sichtbar. Der Inhalt der Spezialregister
kann von der Maschine während der Ausführung eines Programms ohne Einfluss
seitens Programmierers verändert werden.

Nicht alle Spezialregister können durch Instruktionen überschrieben werden
(sind schreibgeschützt).

Die Registernummer der Spezialregister setzen die Registernummerierung
der Allzweckregister zwar bei 33 fort, die Assemblernamen sind aber
zweckspezifisch: es gibt kein Spezialregister mit dem Namen $R33$.
Die Tabelle \ref{tab:Spezialregister} auf Seite \pageref{tab:Spezialregister}
enthält die Liste aller Spezialregister. In der ersten
Spalte steht der Assemblername, so wie er vom Programmierer verwendet wird. In
der zweiten Spalte ist der Maschinenname bzw. die Registernummer im
Dezimalsystem angegeben.
Die dritte Spalte enthält eine kurze Beschreibung und Bemerkungen.
Falls die Beschreibung nicht anders spezifiziert, ist das Register nicht
schreibgeschützt.

\begin{longtable}
{
  >{\ttfamily}p{1cm} 
  >{\ttfamily}p{1cm}
  p{\textwidth-2cm-6\tabcolsep}
}
\caption[Spezialregister]{Liste der Spezialregister}
\label{tab:Spezialregister}
\\\toprule
PC   & 33 & \glqq Instruction Pointer\grqq, oder \glqq Program Counter\grqq.
            Enthält zu jeder Zeit die Adresse der nächsten Instruktion. Wird auf
            256 gesetzt, wenn die Maschine hochfährt. Diese Adresse ist die
            Startadresse des Programms im Speicher (siehe auch Abschnitt
            \ref{subsec:Speicherstruktur}, Seite
            \pageref{subsec:Speicherstruktur}).
            Wird nach dem Abfangen einer Instruktion in das Register
            \texttt{IR} automatisch inkrementiert.

            Schreibgeschützt.
            \index{PC@\texttt{PC}}
\\
DS   & 34 & \glqq Data Segment\grqq, Datensegment.
            Enthält diejenige Adresse im Speicher, wo das Datensegment anfängt.
            Ab dieser Adresse werden im Speicher die eingebetten Programmdaten
            abgelegt.
            Die interne Logik der UMach-Maschie geht davon aus, dass ab
            dieser Adresse keine Instruktionen mehr vorhanden sind.
            
            Schreibgeschützt.
            \index{DS@\texttt{DS}}
\\
HS   & 35 & \glqq Heap Segment\grqq, \glqq Heap Start\grqq, oder Freispeicher
            Segment.
            Enthält diejenige Speicheradresse, ab wo das Programm neuen Speicher
            belegen kann. Dieser Speicherbereich folgt unmittelbar nach dem
            Datensegment und wird rechts vom Stack begrenzt.

            Schreibgeschützt.
            \index{HS@\texttt{HS}}
\\
HE   & 36 & \glqq Heap End\grqq, oder Ende des Heaps.
            Markiert das Ende des Heap-Bereichs, bzw. die Adresse des letzen
            Bytes, das noch zum Heap gehört.
            \index{HE@\texttt{HE}}
\\
SP   & 37 & \glqq Stack Pointer\grqq.
            Enthält die Speicheradresse des höchsten Eintrags auf dem Stack.
            Wird beim Hochfahren der Maschine auf die maximal erreichbare
            Speicheradresse plus 1 gesetzt. Die 1, die zur maximalen Adresse
            addiert wird, setzt
            den Stack Pointer auf eine ungültige Adresse und sorgt dafür, dass
            keine Werte mittels \opref{POP} gelesen werden, bevor Werte
            auf den Stack gepusht wurden.
            Siehe auch \ref{subsec:Stack}, Seite \pageref{subsec:Stack}.
            \index{SP@\texttt{SP}}
\\
FP   & 38 & \glqq Frame Pointer\grqq.
            Enthält die Startadresse des Stack Frames einer Subroutine
            und unterstützt die Implementierung von Funktionen.
            \index{FP@\texttt{FP}}
\\
IR   & 39 & \glqq Instruction Register\grqq. Enthält die gerade ausgeführte
            Instruktion.

            Schreibgeschützt.
            \index{IR@\texttt{IR}}
\\
STAT & 40 & Enthält Status-Informationen. Diese Informationen sind in den 
            einzelnen Bits dieses Register gespeichert.
            Welche Informationen vorhanden sind liegt an der jeweiligen
            Anwendung.
            \index{STAT@\texttt{STAT}}
\\
ERR  & 41 & \glqq Error\grqq.
            Fehlerregister. Die einzelnen Bits dises Registers geben Auskunft
            über Fehler, die mit der Ausführung eines Befehls verbunden sind.
            Ist ein bestimmtes Bit gesetzt, so wird dadurch der
            entsprechende Fehler signalisiert. Für eine Liste der verwendeten
            Bits un deren Bedeutung, siehe Tabelle \ref{tab:ERR-register} auf
            der Seite \pageref{tab:ERR-register}.            
            \index{ERR@\texttt{ERR}}
\\
HI   & 42 & \glqq High\grqq.
            Falls eine Multiplikation durchgeführt wird, enthält dieses Register
            die höchstwertigen 32 Bits des Ergebnisses der Multiplikation und
            bildet zusammen mit dem Register \texttt{LO} das volle Ergebnis der
            Multiplikation.

            Falls eine Division durchgeführt wird, enthält dieses Register den
            Quotient der Division.
            \index{HI@\texttt{HI}}
\\
LO   & 43 & \glqq Low\grqq.
            Falls eine Multiplikation durchgeführt wird, enthält dieses Register
            die niedrigstwertigen 32 Bits des Ergebnisses der Multiplikation und
            bildet zusammen mit dem Register \texttt{HI} das volle Ergebnis der
            Multiplikation.

            Falls eine Division durchgeführt wird, enthält dieses Register den
            Rest der Division.
            \index{LO@\texttt{LO}}
\\
CMPR & 44 & \glqq Comparison Result\grqq.
            Enthält das Ergebnis eines Vergleichs.
            Siehe auch Abschnitt \ref{sec:Vergleichsinstruktionen}, Seite
            \pageref{sec:Vergleichsinstruktionen}.
            \index{CMPR (Reg)@\texttt{CMPR} (Reg)}
\\
ZERO & 00 & Enthält die Zahl Null.

            Schreibgeschützt.
            \index{ZERO@\texttt{ZERO}}
\\\bottomrule
\end{longtable}


\subsection{Das ERR Register}
\index{ERR@\texttt{ERR}}

Jedes Bit im Register signalisiert das auftreten eines spezifischen Fehlers.
Die Bitstellen werden dabei entsprechend deren Stelligkeiten durchnummeriert:
Bit mit Stelligkeit $2^{0}$ hat Position $0$, Bit mit Stelligkeit $2^{31}$ hat
die Position $31$.

Die Bits im \texttt{ERR} Register werden in die folgenden 4 Gruppen unterteilt:

\begin{center}
\begin{tabular}{>{\ttfamily}ll}
\toprule
 0  -  7 & Kern Fehler  \\
 8  - 15 & Speicherfehler                         \\
 16 - 23 & I/O Fehler                             \\
 24 - 31 & Systemfehler                           \\
\bottomrule
\end{tabular}
\end{center}

Tabelle \ref{tab:ERR-register} listet alle Fehler auf, die in dem
\texttt{ERR} Register signalisiert werden können.


\begin{longtable}{>{\ttfamily}ll}
\caption[ERR Register]{Bedeutung der einzelnen Bits im ERR Register}
\label{tab:ERR-register}
\\\toprule
 0  & Division durch Null       \\
 1  & Arithmetischer Überlauf   \\
 5  & Ungültiger Befehlscode    \\
 6  & Ungültige Registernummer  \\
\midrule
 8  & Ungültige Speicheradresse \\
 9  & Zugriffsverletzung (segmentation fault) \\
11  & Stack Überlauf            \\
\midrule
16  & I/O Port existiert nicht \\
\bottomrule
\end{longtable}

Siehe auch den Abschnitt \ref{subsec:Speicherfehler}
\nameref{subsec:Speicherfehler}, Seite \pageref{subsec:Speicherfehler}.



