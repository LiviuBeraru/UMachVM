\subsection{Spezialregister}
\index{Register!Spezialregister}
\index{Spezialregister}
Die Spezialregister werden von der UMach Maschine für spezielle Zwecke
verwendet, sind aber dem Programmiere sichtbar. Der Inhalt der Spezialregister
kann von der Maschine während der Ausführung eines Programms ohne Einfluss
seitens Programmierers verändert werden.

Nicht alle Spezialregister können durch Instruktionen überschrieben werden
(sind schreibgeschützt).

Die Maschinennamen der Spezialregister setzen die Nummerierung der
Allzweckregister zwar fort, die Assemblernamen aber nicht: es gibt kein Register
$R33$.
Die Tabelle \ref{tab:Spezialregister} auf Seite \pageref{tab:Spezialregister}
enthält die Liste aller Spezialregister.  In der ersten
Spalte steht der Assemblername, so wie er vom Programmierer verwendet wird. In
der zweiten Spalte steht der Maschinenname im Hexadezimalsystem, so wie er im
Maschinencode steht.
Die dritte Spalte enthält eine kurze Beschreibung und Bemerkungen.
Falls die Beschreibung nicht spezifiziert, dass ein Register schreibgeschützt
ist, ist das Register nicht schreibgeschützt.

\begin{longtable}
{
  >{\ttfamily}p{1cm} 
  >{\ttfamily}p{0.5cm}
  p{\textwidth-1.5cm-6\tabcolsep}
}
\caption{Liste der Spezialregister}
\label{tab:Spezialregister}\\
IP   & 33 & \glqq Instruction Pointer\grqq. Enthält zu jeder Zeit die Adresse
            der nächsten Instruktion. Wird auf Null gesetzt, wenn die Maschine
            hochfährt.
            Wird nach dem Abfangen einer Instruktion in das Register
            \texttt{CIN} automatisch inkrementiert.

            Schreibgeschützt.
            \index{IP@\texttt{IP}}
\\
CIN  & ?? & \glqq Current Instruction\grqq. Enthält die nächste Instruktion.
            Vor der Ausführung einer Instruktion gilt, dass der Inhalt dieses
            Registers und der Speicher an der Adresse \texttt{IP} gleich sind.

            Schreibgeschützt.
            \index{CIN@\texttt{CIN}}
\\
ERR  & ?? & Fehlerregister. Die einzelnen Bits geben Auskunft über Fehler, die
            mit der Ausführung des Programms verbunden sind.

            Schreibgeschützt.
            \index{ERR@\texttt{ERR}}
\\
RMOD & ?? & Betriebsmodus. (\glqq Run Mode\grqq).
            \index{RMOD@\texttt{RMOD}}
\\
SMOD & ?? & Systemmodus. Kernel mode etc.
            \index{SMOD@\texttt{SMOD}}
\\
ZERO & ?? & Enthält die Zahl Null.

            Schreibgeschützt.
            \index{ZERO@\texttt{ZERO}}
\\
ARST & ?? & Arithmetischer Status. Division durch Null, Überlauf etc.

            Schreibgeschützt.
            \index{ARST@\texttt{ARST}}
\\
INT  & ?? & \glqq Interrupt Number\grqq, wird für syscalls benutzt.
            \index{INT@\texttt{INT}}
\\
DEAD & ?? & Gesetzt, wenn der Zustand der Maschine keine weiteren Ausführungen
            mehr erlaubt. Die Maschine ist \glqq tot\grqq, wenn dieses Register
            einen Wert ungleich Null hat.
            \index{DEAD@\texttt{DEAD}}
\end{longtable}

