\section{Register}
\label{sec:Register}
\index{Register}
\index{UMach!Register}

Die \glspl{Register} sind die Speichereinheiten im Prozessor.
Die meisten Anweisungen an die UMach Maschine operieren auf einer Art mit den
Registern.

Für alle Register gilt:
\begin{enumerate}
  \item Die Speicherkapazität beträgt 32 Bit.
  \item Es gibt eine eindeutige Zahl $n \in \mathds{N}$, die innerhalb der
    Maschine das Register identifiziert. Diese Zahl wird von einer Instruktion
    verwendet, wenn sie das Register anspricht.
  \item Die UMach Maschine erwartet die Angabe eines Registers als numerischer
    Wert. Jedoch verwendet der Programmierer der Maschine auf Assembler Ebene
    einen eindeutigen Namen dieses Registers.
\end{enumerate}

Die UMach Maschine hat zwei Gruppen von Registern: die Allzweckregister und
die Spezialregister. 


\subsection{Allzweckregister}
\index{Register!Allzweckregister}
\index{Allzweckregister}
Es gibt 32 Allzweckregister, die dem Programmierer zur Verfügung stehen.
Diese 32 Register werden beim Hochfahren der Maschine auf Null gesetzt. Außer
dieser Initialisierung, verändert die Maschine den Inhalt der Allzweckregister
nur auf explizite Anfrage, bzw. infolge einer Instruktion. 

\subsection{Spezialregister}


