\section{Register}
\label{sec:Register}
\index{Register}
\index{UMach!Register}

Die \glspl{Register} sind Speichereinheiten im Prozessorkern.
Die meisten Instruktionen operieren auf diesen
Registern.

Für alle Register gilt:
\begin{enumerate}
  \item Das Register ist ein Element aus der Menge \Reg aller Register
	\index{\Reg}
    Die Notation $x \in \Reg$ bedeutet somit, dass $x$ ein Register ist.
  \item Die Speicherkapazität beträgt 32 Bit.
  \item Es gibt eine eindeutige natürliche Nummer, die \gls{Registernummer}
    \index{Registernummer}\index{Register!Registernummer}. Diese dient zur
    Identifikation des Registers in der Maschine und auf Maschinencode-Ebene.
  \item Auf Assembler Ebene git es einen zusätzlichen eindeutigen Namen für das
    Register. Dieser Name heißt
    \gls{Assemblername}.\index{Assemblername}\index{Register!Assemblername}
\end{enumerate}

Die UMach Maschine besitzt zwei Arten von Registern: die Allzweck- und
Spezialregister.


\subsection{Allzweckregister}
\index{Register!Allzweckregister}
\index{Allzweckregister}

Es gibt 32 Allzweckregister, die dem Programmierer für allgemeine Zwecke zur
Verfügung stehen. Diese 32 Register werden beim Hochfahren der Maschine mit dem
Wert Null (\texttt{0x00}) initialisiert. Außer dieser Initialisierung, verändert
die Maschine den Inhalt der Allzweckregister nur auf explizite Anfrage, bzw.
infolge einer Instruktion. 

Die 32 Register werden auf Maschinencode-Ebene von 1 bis 32 nummeriert
(\texttt{0x01} bis einschliesslich \texttt{0x20} im Hexadezimalsystem). Diese
Nummer ist die \gls{Registernummer} des Registers. Auf Assembler-Ebene, werden
sie mit den Namen $R1, R2\dots$ bis $R32$ angesprochen (Assemblername). Die Zahl
nach dem Buchstaben $R$ ist im Dezimalsystem angegeben und ist fester
Bestandteil des Registernamens.

\begin{center}
  \newcolumntype{T}{>{\ttfamily}c}
  \begin{tabular}{l||*{5}{T|}}
    Assemblername  & R1   & R2   & R3   & \dots & R32 \\
    Registernummer & 0x01 & 0x01 & 0x03 & \dots & 0x20
  \end{tabular}
\end{center}

Das Register mit Nummer Null ($R0$), oder das
Null-Register\index{Null-Register}\index{Register!Null-Register}, ist ein
Spezialregister, dass immer den Wert Null hat und nicht beschreibbar ist (siehe
auch den Abschnitt \ref{subsec:Spezialregister} auf Seite
\pageref{subsec:Spezialregister}).

Beispiel für die Verwendung von Registernamen:
\begin{center}
  \begin{tabular}{|l||*{4}{c|}}                                       \hline
    Assembler     & \texttt{ADD}  & \texttt{R1}   & \texttt{R2}   &
                    \texttt{R3}                                      \\
    Maschinencode & \texttt{0x50} & \texttt{0x01} & \texttt{0x02} &
                    \texttt{0x03}                                    \\\hline
    Bytes         & erstes Byte   & zweites Byte  & drittes Byte  & 
                    viertes Byte                                     \\\hline
    Algebraisch   & \multicolumn{4}{c|}{$R_{1} \gets R_{2} + R_{3}$} \\\hline
  \end{tabular}
\end{center}



\subsection{Spezialregister}
Die Spezialregister werden von der UMach Maschine für spezielle Zwecke
verwendet, sind aber dem Programmiere sichtbar. Der Inhalt der Spezialregister
kann von der Maschine während der Ausführung eines Programms ohne Einfluss
seitens Programmierers verändert werden.



