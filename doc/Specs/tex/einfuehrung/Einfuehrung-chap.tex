\chapter{Einführung}
UMach ist eine einfache virtuelle Maschine (VM), die einen definierten
Instruktionssatz und eine definierte Architektur hat. UMach orientiert
sich dabei an Prinzipien von RISC Architekturen: feste Instruktionslänge,
kleine Anzahl von einfachen Befehlen, Speicherzugriff durch Load- und
Store-Befehlen, u.s.w. Die UMach Maschine ist Register-basiert.
Der genaue Aufbau dieser Rechenmaschine ist im Abschnitt
\ref{sec:Aufbau} ab der Seite \pageref{sec:Aufbau} beschrieben.


Für den Anwender der virtuellen Maschine wird zuerst eine Assembler-Sprache zur
Verfügung gestellt. In dieser Sprache werden Programme geschrieben die
anschließend kompiliert werden. Die kompilierte Dateien (Maschinen-Code) wird
von der virtuellen Maschine ausgeführt.

\section{Anwendungsbeispiel}



\begin{lstlisting}
LOAD R1 90
LOAD R2 09
REV  R3 R1
\end{lstlisting}



