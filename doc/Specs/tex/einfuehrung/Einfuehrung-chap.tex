\chapter{Einführung}

UMach ist eine einfache programmierbare virtuelle Maschine (VM), die einen
definierten Instruktionssatz und eine definierte Architektur hat. UMach
orientiert sich dabei an Prinzipien von RISC Architekturen: feste
Instruktionslänge, kleine Anzahl von einfachen Befehlen, Speicherzugriff durch
Load- und Store-Befehlen, usw. Die UMach Maschine ist Register-basiert. Der
genaue Aufbau dieser Rechenmaschine ist im Abschnitt \ref{sec:Aufbau} ab der
Seite \pageref{sec:Aufbau} beschrieben.


Für den Anwender der virtuellen Maschine wird zuerst eine Assembler-Sprache zur
Verfügung gestellt. In dieser Sprache werden Programme geschrieben und
anschließend kompiliert. Die kompilierte Datei (Maschinen-Code) wird
von der virtuellen Maschine ausgeführt.

Obwohl in diesem Dokument Namen von Assembler-Befehlen angegeben werden (siehe 
Kapitel \ref{chap:Instruktionssatz}, \nameref{chap:Instruktionssatz})
spezifiziert dieses Dokument die UMach Maschine auf Maschinencode Ebene
(Register, Bussystem, Instruktionen).
Die Implementierung eines Assemblers ist frei, zusätzliche Befehle,
Instruktionsformate, Aliase und sprachliche Konstrukte auf der Assembler-Ebene
zu definieren. Z.B. auf Maschinencode-Ebene, sind die Befehle
\begin{lstlisting}
  ADD R1 R2 5
  SUB R2 4
\end{lstlisting}
fehlerhaft, denn das Format der \opref{ADD} und \opref{SUB} Befehle verlangt die
Angabe von drei Registern. Der Assembler ist frei, diese zusätzliche Formate zu
definieren und zu erkennen, solange er gültigen UMach Maschinencode produziert.
Gültiger Maschinencode für die obigen Befehle wäre
\begin{lstlisting}
  ADDI R1 R2 5 # Maschinencode 0x32 0x01 0x02 0x05
  SUBI R2 R2 4 # Maschinencode 0x35 0x02 0x02 0x04
\end{lstlisting}



