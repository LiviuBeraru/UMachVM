\chapter{Einführung}

UMach ist eine möglichst einfach konzipierte und programmierbare virtuelle
Maschine (VM). Um diese Anforderung zu erfüllen wurde auf komplexere
Funktionalitäten bewusst verzichtet. Sie besitzt einen fest definierten 
Instruktionssatz und Architektur. UMach orientiert sich dabei an Prinzipien von
RISC Architekturen: feste Instruktionslänge, kleine Anzahl von einfachen
Befehlen, Speicherzugriff durch Load- und Store-Befehlen, usw. Die UMach
Maschine ist Register-basiert. Der genaue Aufbau dieser Rechenmaschine ist im
Abschnitt \ref{sec:Aufbau} ab der Seite \pageref{sec:Aufbau} beschrieben.


Für die Verwendung der virtuellen Maschine wird zuerst eine Assembler-Sprache
zur Verfügung gestellt. In dieser Sprache werden Programme geschrieben und
anschließend assembliert. Die assemblierte Datei (Maschinen-Code) kann dann
von der virtuellen Maschine ausgeführt werden.

Obwohl in diesem Dokument Namen von Assembler-Befehlen angegeben werden (siehe 
Kapitel \ref{chap:Instruktionssatz}, \nameref{chap:Instruktionssatz})
spezifiziert dieses Dokument die UMach Maschine auf Maschinencode Ebene
(Register, Bussystem, Instruktionen).
Bei der Implementierung eines Assemblers steht es frei, zusätzliche Befehle,
Instruktionsformate, Aliase und sprachliche Konstrukte auf der Assembler-Ebene
einzuführen. So sind z.B. die folgenden Befehle
\begin{lstlisting}
  ADD R1 R2 5
  SUB R2 4
\end{lstlisting}
auf Maschinencode-Ebene ungültig, denn hier verlangt die Formatdefinition der
\opref{ADD} und \opref{SUB} Befehle die Angabe von drei Registern. Ein Assembler
kann jedoch diese zusätzliche Formate definieren, solange er daraus
gültigen UMach Maschinencode erzeugt.
Gültiger Maschinencode für die genannten Beispiele wäre
\begin{lstlisting}
  ADDI R1 R2 5 # Maschinencode 0x32 0x01 0x02 0x05
  SUBI R2 R2 4 # Maschinencode 0x35 0x02 0x02 0x04
\end{lstlisting}



