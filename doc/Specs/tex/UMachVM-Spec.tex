\documentclass[fontsize=12pt,parskip=full,index=totoc,listof=totoc,
bibliography=totoc]
{scrreprt}

%\addtokomafont{disposition}{\rmfamily}
%\addtokomafont{descriptionlabel}{\rmfamily}

\usepackage[ilines,headsepline]{scrpage2}
\pagestyle{scrheadings}
\clearscrheadfoot
\automark{section}
\ohead{\leftmark}
\cfoot{\pagemark}

\usepackage[utf8]{inputenc}
\usepackage[T1]{fontenc}
\usepackage[ngerman]{babel}

\usepackage{lmodern}
\usepackage{amsmath,amsfonts,amssymb}
\usepackage{dsfont}
\usepackage{ellipsis}
\usepackage{microtype}
\usepackage{fixltx2e}
\usepackage{array}
\usepackage{longtable}
\usepackage{booktabs}
\usepackage{calc}
\usepackage{multirow}
\usepackage{xcolor}
\usepackage{graphicx}
\usepackage{marginnote}%thanks Markus Kohm

\usepackage{scrhack}%entfernt warnmeldungen von listings
\usepackage{listings}
\lstset{basicstyle=\small\ttfamily}
\lstset{commentstyle=\slshape}
\lstset{showstringspaces=false}
\lstset{frame=leftline}
\lstset{mathescape=true}
\lstset{
literate={Ä}{{\"A}}1
         {ä}{{\"a}}1
         {Ö}{{\"O}}1
         {ö}{{\"o}}1
         {Ü}{{\"U}}1
         {ü}{{\"u}}1
         {ß}{{\ss}}1
}

%\lstset{morecomment=[l][\color{blue!10!black}]{\#}}

\usepackage{makeidx}
\makeindex

\title{UMach Spezifikation}
\author{}

\usepackage[
    pdftitle={UMach VM Spezifikation},
    colorlinks=true,
    linkcolor=blue!50!black]
{hyperref} 

% Ausgabe einer Befehlsdefinition in tabellarischer Form
% Nutzung: \opdef{<opname>}{<params>}{<opcode>}{<format>}
% opname: ADD, SUB etc
% params: Parameter
% opcode: hexa opcode
% format: Abschnitt Name eines Instruktionsformats
% Beispiel: \opdef{ADD}{X Y Z}{0x40}{RRR}
% Referenzieren mit \nameref{opcode:<opname>}
% Beispiel: \nameref{opcode:ADD}
\newcommand{\opdef}[4]{%
\subsection{\texttt{#1}}\index{#1@\texttt{#1}}%
\label{opcode:#1}%
\begin{center}%
  \begin{tabular}{llll}                       \toprule%
    Assemblername & Parameter & Maschinencode & Format \\\midrule%
    \texttt{#1}   & #2        & \texttt{#3}   & \nameref{#4}\\\bottomrule%
  \end{tabular}%
\end{center}%
}

%% Referenziere einen Befehl. Es wird der Name des Befehls angezeigt
%% und man kann drauf klicken.
%% Nutzung: \opref{ADD}
\newcommand{\opref}[1]{\nameref{opcode:#1}}

% Die Menge aller Register bekommt ein eigenes Zeichen
\newcommand{\Reg}{\ensuremath{\mathcal{R}}}

\begin{document}
\maketitle
\tableofcontents
\listoftables
\listoffigures

%--- Jedes Kapitel in eigener Datei und eigenem Verzeichnis 
%--- Jede Kapitel-Datei fängt mit \chapter an (markiert Kapitel)
%--- und kann nach belieben andere \section enthalten
%--- Empfehlung: jeder Abschnitt \section in eigener Datei

\chapter{Einführung}

UMach ist eine möglichst einfach konzipierte und programmierbare virtuelle
Maschine (VM). Um diese Anforderung zu erfüllen wurde auf komplexere
Funktionalitäten bewusst verzichtet. Sie besitzt einen fest definierten 
Instruktionssatz und Architektur. UMach orientiert sich dabei an Prinzipien von
RISC Architekturen: feste Instruktionslänge, kleine Anzahl von einfachen
Befehlen, Speicherzugriff durch Load- und Store-Befehlen, usw. Die UMach
Maschine ist Register-basiert. Der genaue Aufbau dieser Rechenmaschine ist im
Abschnitt \ref{sec:Aufbau} ab der Seite \pageref{sec:Aufbau} beschrieben.


Für die Verwendung der virtuellen Maschine wird zuerst eine Assembler-Sprache
zur Verfügung gestellt. In dieser Sprache werden Programme geschrieben und
anschließend assembliert. Die assemblierte Datei (Maschinen-Code) kann dann
von der virtuellen Maschine ausgeführt werden.

Obwohl in diesem Dokument Namen von Assembler-Befehlen angegeben werden (siehe 
Kapitel \ref{chap:Instruktionssatz}, \nameref{chap:Instruktionssatz})
spezifiziert dieses Dokument die UMach Maschine auf Maschinencode Ebene
(Register, Bussystem, Instruktionen).
Bei der Implementierung eines Assemblers steht es frei, zusätzliche Befehle,
Instruktionsformate, Aliase und sprachliche Konstrukte auf der Assembler-Ebene
einzuführen. So sind z.B. die folgenden Befehle
\begin{lstlisting}
  ADD R1 R2 5
  SUB R2 4
\end{lstlisting}
auf Maschinencode-Ebene ungültig, denn hier verlangt die Formatdefinition der
\opref{ADD} und \opref{SUB} Befehle die Angabe von drei Registern. Ein Assembler
kann jedoch diese zusätzliche Formate definieren, solange er daraus
gültigen UMach Maschinencode erzeugt.
Gültiger Maschinencode für die genannten Beispiele wäre
\begin{lstlisting}
  ADDI R1 R2 5 # Maschinencode 0x32 0x01 0x02 0x05
  SUBI R2 R2 4 # Maschinencode 0x35 0x02 0x02 0x04
\end{lstlisting}





\chapter{Organisation der UMach VM}

\section{Aufbau}
\index{UMach!Aufbau}
\label{sec:Aufbau}
Dieser Abschnitt beschreibt den Aufbau der UMach virtuellen Maschine.
Die virtuelle Maschine besteht aus internen und aus externen Komponenten. Dabei
sind die externen Komponenten nicht wesentlich für die Funktionsfähigkeit der
gesamten Maschine, d.h. die Maschine kann im Prinzip auch ohne die externen
Komponenten funktionieren -- in diesem Fall fehlt ihr eine Menge von Funktionen.

\paragraph{Interne Komponenten}
sind diejenigen Komponenten, die für die Funktionsfähigkeit der UMach
Maschine wesentlich sind:
\begin{enumerate}
  \item Recheneinheit
  \item Logische Einheit
  \item Register
\end{enumerate}


\paragraph{Externe Komponenten}

\begin{enumerate}
  \item Anbindung an einem I/O-Port
\end{enumerate}


\section{Register}
\label{sec:Register}
\index{Register}
\index{UMach!Register}

Die \glspl{Register} sind die Speichereinheiten im Prozessor.
Die meisten Anweisungen an die UMach Maschine operieren auf einer Art mit den
Registern.

Für alle Register gilt:
\begin{enumerate}
  \item Die Speicherkapazität beträgt 32 Bit.
  \item Es gibt eine eindeutige Zahl $n \in \mathds{N}$, die innerhalb der
    Maschine das Register identifiziert. Diese Zahl wird von einer Instruktion
    verwendet, wenn sie das Register anspricht.
  \item Die UMach Maschine erwartet die Angabe eines Registers als numerischer
    Wert. Jedoch verwendet der Programmierer der Maschine auf Assembler Ebene
    einen eindeutigen Namen dieses Registers.
\end{enumerate}

Die UMach Maschine hat zwei Gruppen von Registern: die Allzweckregister und
die Spezialregister. 


\subsection{Allzweckregister}
\index{Register!Allzweckregister}
\index{Allzweckregister}
Es gibt 32 Allzweckregister, die dem Programmierer zur Verfügung stehen.
Diese 32 Register werden beim Hochfahren der Maschine auf Null gesetzt. Außer
dieser Initialisierung, verändert die Maschine den Inhalt der Allzweckregister
nur auf explizite Anfrage, bzw. infolge einer Instruktion. 

\subsection{Spezialregister}





\subsection{Rechen- und logische Einheit}
\blindtext

\subsection{I/O-Einheit}
\blindtext


\subsection{Betriebsmodi}
\label{subsec:Betriebsmodi}
\index{Betriebsmodus}

Ein \gls{Betriebsmodus} bezieht sich auf die Art, wie die UMach VM grundsätzlich
läuft. Die UMach VM kann in einem der folgenden Betriebsmodi laufen:

\begin{enumerate}
  \item Normalmodus
  \item Einzelschrittmodus
\end{enumerate}

Der standard-Modus ist der Normalmodus.
Der Modus wird vor dem Hochfahren der Maschine festgelegt.

\paragraph{Normalmodus} Die virtuelle Maschine führt ohne Unterbrechung ein
Programm aus. Nach der Ausführung befindet sich die Maschine in einem
Wartezustand, falls sie nicht ausdrücklich ausgeschaltet wird.

\paragraph{Einzelschrittmodus} Die virtuelle Maschine führt immer eine einzige
Instruktion aus und nach der Ausführung wartet sie auf einen externen Signal um
mit der nächsten Instruktion fortzufahren. Dieser Modus soll dem Entwickler
erlauben, ein Programm schrittweise zu debuggen.



\section{Register}
\label{sec:Register}
\index{Register}
\index{UMach!Register}

Die \glspl{Register} sind die Speichereinheiten im Prozessor.
Die meisten Anweisungen an die UMach Maschine operieren auf einer Art mit den
Registern.

Für alle Register gilt:
\begin{enumerate}
  \item Die Speicherkapazität beträgt 32 Bit.
  \item Es gibt eine eindeutige Zahl $n \in \mathds{N}$, die innerhalb der
    Maschine das Register identifiziert. Diese Zahl wird von einer Instruktion
    verwendet, wenn sie das Register anspricht.
  \item Die UMach Maschine erwartet die Angabe eines Registers als numerischer
    Wert. Jedoch verwendet der Programmierer der Maschine auf Assembler Ebene
    einen eindeutigen Namen dieses Registers.
\end{enumerate}

Die UMach Maschine hat zwei Gruppen von Registern: die Allzweckregister und
die Spezialregister. 


\subsection{Allzweckregister}
\index{Register!Allzweckregister}
\index{Allzweckregister}
Es gibt 32 Allzweckregister, die dem Programmierer zur Verfügung stehen.
Diese 32 Register werden beim Hochfahren der Maschine auf Null gesetzt. Außer
dieser Initialisierung, verändert die Maschine den Inhalt der Allzweckregister
nur auf explizite Anfrage, bzw. infolge einer Instruktion. 

\subsection{Spezialregister}



\section{Speichermodell}
\index{Speichermodell}

\subsection{Adressierungsarten}
\label{subsec:Adressierungsarten}
\index{Adressierungsarten}

Als RISC-orientierte Maschine, greift die UMach lediglich in zwei Situationen
auf den Speicher zu: zum Schreiben von Registerinhalten in den Speicher
(Schreibzugriff) und zum Lesen von Speicherinhalten in einen Register
(Lesezugriff).  Die \gls{Adressierungsart} beschreibt dabei, wie der Zugriff
auf den Speicher erfolgen sollte, bzw. wie die angesprochene Speicheradresse
angegeben wird. Anders ausgedrückt, beantwortet die Adressierungsart die Frage
\glqq wie kann eine Instruktion der Maschine eine Adresse angeben?\grqq. 

Die UMach Maschine kennt eine einzige Adressierungsart: die indirekte
Adressierung, die unten beschrieben wird. Die direkte Adressierung, die aus
einer direkten Angabe eines Speicheradresse besteht, wird von der indirekten
Adressierung überdeckt.

\subsubsection{Indirekte Adressierung}
\index{Adressierung!Indirekte}

Die indirekte Adressierung verwendet zwei Register $B$ und $I$, die
von der Maschine verwendet werden, um die endgültige Adresse zu berechnen:
Eine Instruktion, die diese Adressierung verwendet, hat also das Format RRR
(siehe auch \ref{subsec:RRR}).
\begin{center}
  \begin{tabular}{|*{4}{c|}|c|} \hline
    erstes Byte    & zweites Byte  & drittes Byte  & viertes Byte & Algebraisch
\\\hline\hline
    Ladebefehl     & $R$  & $B$  & $I$  & $R \gets mem(B + I)$ \\\hline
    Speicherbefehl & $R$  & $B$  & $I$  & $R \to   mem(B + I)$ \\\hline
  \end{tabular}
\end{center}
Die fünfte Spalte gibt jeweils den äquivalenten algebraischen Ausdruck wieder.
$mem(x)$ steht dabei für den Inhalt der Adresse $x$.


Die zweite Zeile (Ladebefehl) bedeutet, dass die UMach Maschine die Inhalte der
Register $B$ und $I$ aufaddieren soll, diese Summe als Adresse im
Speicher zu verwenden und den Inhalt an dieser Adresse in den Register $R$ zu
kopieren.

Die dritte Zeile (Speicherbefehl) bedeutet: die Maschine soll den Inhalt des
Registers $R$ an die Adresse $B + I$ schreiben.

Üblicherweise enthält $B$ eine Startadresse und $I$ einen Versatz oder Index zur
Adresse in $B$.

Vorteil der indirekten Adressierung ist, dass sie $2^{33} - 1$ mögliche Adressen
ansprechen kann. Nachteil ist, dass zwei oder mehrere Instruktionen gebraucht
werden, um diese Adressierung zu verwenden, denn die Register $B$ und $I$ erst
entsprechend geladen werden müssen.

Die Register $R$, $B$ und $I$ stehen für beliebige Register.

\section{Datentypen}
\blindtext



\chapter{Instruktionssatz}
\index{Instruktionssatz}

In diesem Kapitel werden alle Instruktionen\index{Instruktionen} der UMach VM 
vorgestellt.

\section{Instruktionsformate}
\label{sec:Instruktionsformate}
\index{Instruktionsformat}

Eine Instruktion besteht aus einer Folge von 4 Bytes.
Das \gls{Instruktionsformat} beschreibt die Struktur einer Instruktion auf
Byte-Ebene. Das Format gibt an, ob ein Byte als eine Registerangabe oder als reine
numerische Angabe zu interpretieren ist.

\paragraph{Instruktionsbreite}
\index{Instruktionsbreite}
Jede UMach-Instruktion hat eine feste Bitlänge von 32 Bit (4 mal 8 Bit).
Instruktionen, die für ihren
Informationsgehalt weniger als 32 Bit brauchen, wie z.B. \texttt{NOP},
werden mit Nullbits gefüllt. Alle Daten und Informationen, die mit einer
Instruktion übergeben werden, müssen in diesen 32 Bit untergebracht werden.

\paragraph{Byte Order}
\index{Byte Order}
Die Byte Order (Endianness) der gelesenen \glspl{Byte} ist big-endian.
Die zuerst gelesenen 8 Bits sind die 8 höchstwertigen (Wertigkeiten $2^{31}$ bis
$2^{24}$) und die zuletzt gelesenen Bits sind die niedrigstwertigen
(Wertigkeiten $2^{7}$ bis $2^{0}$).
Bits werden in Stücken von $n$ Bits gelesen, wobei $n = k \cdot 8$ mit
$k \in \{1, 4\}$ (byteweise oder wortweise).


\paragraph{Allgemeines Format}
Jede \gls{Instruktion} besteht aus zwei Teilen: der erste Teil ist
8 Bit lang und entspricht dem tatsächlichen \gls{Befehl}, bzw. der Operation,
die von der UMach virtuellen Maschine ausgeführt werden soll.
Dieser 8-Bit-Befehl belegt also die 8 höchstwertigen Bits einer
32-Bit-Instruktion.  Die übrigen 24 Bits, wenn sie verwendet werden, werden
für Operanden oder Daten benutzt. Beispiel einer Instruktionszerlegung:

\begin{center}
  \begin{tabular}{|l|*{4}{c|}}
    \hline
    Instruktion (32 Bit) &
    \texttt{00000001} & \texttt{00000010} & \texttt{00000011} & \texttt{00000100}
    \\\hline
    Hexa  &
    \texttt{01}   & \texttt{02}   & \texttt{03}   & \texttt{04}
    \\\hline
    Byte Order &
    erstes Byte   & zweites Byte  & drittes Byte  & viertes Byte
    \\\hline
    Interpretation &
    Befehl (8 Bit) &  \multicolumn{3}{c|}{Operanden, Daten oder Füllbits}
    \\\hline
  \end{tabular}
\end{center}

Die Instruktionsformate unterscheiden sich lediglich darin, wie sie die 24 Bits
nach dem 8-Bit \gls{Befehl} verwenden. Das wird auch in der 3-buchstabigen
Benennung deren Formate wiedergeben.

In den folgenden Abschnitten werden die UMach-Instruktionsformate vorgestellt.
Jede Angegebene Tabelle gibt in der ersten Zeile die Reihenfolge der Bytes an. 
Die nächste Zeile gibt die spezielle Belegung der einzelnen Bytes an.



\subsection{000}
\label{subsec:000}
\index{Instruktionsformat!000}
\index{000}

\begin{center}
  \begin{tabular}{|*{4}{c|}} \hline
    erstes Byte & zweites Byte  & drittes Byte  & viertes Byte \\\hline\hline
    Befehl      & \multicolumn{3}{c|}{nicht verwendet}         \\\hline
  \end{tabular}
\end{center}

Eine Instruktion, die das Format 000 hat, besteht lediglich aus einem Befehl
ohne Argumenten. Die letzen drei Bytes werden von der Maschine nicht
ausgewertet und sind somit Füllbytes. Es wird empfohlen, die letzten 3 Bytes mit
Nullen zu füllen.




\subsection{NNN}
\label{subsec:NNN}
\index{Instruktionsformat!NNN}
\index{NNN}

\begin{center}
  \begin{tabular}{|*{4}{c|}}
    \hline
    erstes Byte  & zweites Byte  & drittes Byte  & viertes Byte \\\hline\hline
    Befehl       & \multicolumn{3}{c|}{numerische Angabe $N$}   \\\hline
  \end{tabular}
\end{center}

Die Instruktion im Format NNN besteht aus einem Befehl im ersten Byte und aus
einer numerischen Angabe  $N$ (einer Zahl), die die letzten 3 Bytes belegt.
Die Interpretation der numerischen Angabe wird dem jeweiligen Befehl überlassen.


\subsection{RNN}
\label{subsec:RNN}
\index{Instruktionsformat!RNN}
\index{RNN}

\begin{center}
  \begin{tabular}{|*{4}{c|}} \hline
    erstes Byte & zweites Byte  & drittes Byte  & viertes Byte   \\\hline\hline
    Befehl      & $R_{1}$ & \multicolumn{2}{c|}{numerische Angabe $N$} \\\hline
  \end{tabular}
\end{center}

Eine Instruktion im Format RNN besteht aus einem Befehl, gefolgt von einer
Register Nummer $R_{1}$, gefolgt von einer festen Zahl $N$, die die letzten
2 Bytes der Instruktion belegt.
Die genaue Interpretation der Zahl $N$ wird dem jeweiligen Befehl überlassen.
Zum Beispiel, die Instruktion
\begin{center}
  \begin{tabular}{|*{4}{c|}} \hline
    erstes Byte & zweites Byte  & drittes Byte  & viertes Byte \\\hline\hline
    \texttt{0x20} & \texttt{0x01} & \texttt{0x02} & \texttt{0x03} \\\hline
  \end{tabular}
\end{center}
wird folgenderweise von der UMach Maschine interpretiert: die Operation mit
Nummer \texttt{0x20} soll ausgeführt werden, wobei die Argumenten dieser
Operation sind das Register mit Nummer \texttt{0x01} und die numerische
Angabe \texttt{0x0203}.



\subsection{RRN}
\label{subsec:RRN}
\index{Instruktionsformat!RRN}
\index{RRN}

\begin{center}
  \begin{tabular}{|*{4}{c|}} \hline
    erstes Byte  & zweites Byte  & drittes Byte  & viertes Byte \\\hline\hline
    Befehl       & $R_{1}$       & $R_{2}$ & numerische Angabe $N$  \\\hline
  \end{tabular}
\end{center}
Eine Instruktion im Format RRN besteht aus einem Befehl, gefolgt von der
Angabe zweier Registers $R_{1}$ und $R_{2}$, jeweils in einem Byte, gefolgt von
einer numerischen Angabe $N$ (festen Zahl) im letzten Byte. Zum Beispiel, die
Instruktion
\begin{center}
  \begin{tabular}{|*{4}{c|}} \hline
    erstes Byte & zweites Byte  & drittes Byte  & viertes Byte \\\hline\hline
    \texttt{0x41} & \texttt{0x01} & \texttt{0x02} & \texttt{0x03} \\\hline
  \end{tabular}
\end{center}
soll wie folgt interpretiert werden: 
die Operation mit Nummer \texttt{0x41} soll ausgeführt werden, wobei die
Argumenten dieser Operation sind Register mit Nummer \texttt{0x01}, Register mit
Nummer \texttt{0x02} und die Zahl \texttt{0x03}.



\subsection{RRR}
\label{subsec:RRR}
\index{Instruktionsformat!RRR}
\index{RRR}

\begin{center}
  \begin{tabular}{|*{4}{c|}} \hline
    erstes Byte & zweites Byte  & drittes Byte  & viertes Byte \\\hline\hline
    Befehl      & $R_{1}$       & $R_{2}$       & $R_{3}$      \\\hline
  \end{tabular}
\end{center}

Eine Instruktion im Format RRR besteht aus der Angabe eines Befehls im ersten
Byte, gefolgt von der Angabe dreier Register $R_{1}$, $R_{2}$ und $R_{3}$ in den
jeweiligen folgenden drei Bytes.
Die Register werden als Zahlen angegeben und deren Bedeutung hängt vom
jeweiligen Befehl ab.


\subsection{Zusammenfassung}
\label{subsec:Instr-Formate-Zusammenfassung}
Im folgenden werden die Instruktionsformate tabellarisch zusammengefasst.

\index{Instruktionsformat!Liste}
\begin{center}
  \begin{tabular}{|l||*{4}{c|}}
    \hline
    Format & erstes Byte & zweites Byte  & drittes Byte  & viertes Byte
    \\\hline\hline
    000 & Befehl & \multicolumn{3}{c|}{nicht verwendet}                 \\\hline
    NNN & Befehl & \multicolumn{3}{c|}{numerische Angabe $N$}           \\\hline
    RNN & Befehl & $R_{1}$ & \multicolumn{2}{c|}{numerische Angabe $N$} \\\hline
    RRN & Befehl & $R_{1}$ & $R_{2}$ & numerische Angabe $N$            \\\hline
    RRR & Befehl & $R_{1}$ & $R_{2}$ & $R_{3}$                          \\\hline
  \end{tabular}
\end{center}




\section{Verteilung des Befehlsraums}
\index{Befehlsraum}
Zur besseren Übersicht der verschiedenen UMach-\glspl{Instruktion}, unterteilen
wir den \gls{Instruktionssatz} der UMach virtuellen Maschine in den folgenden
Kategorien:
\index{Instruktionen!Kategorien}

\index{Befehlsraum!Verteilung}
\begin{enumerate}
  \item Kontrollinstruktionen,  die die Maschine in ihrer gesamten
    Funktionalität betreffen, wie z.B. den Betriebsmodus umschalten.
  \item Lade- und Speicherbefehle, die Register mit Werten aus dem Speicher, 
    anderen Registern oder direkten numerischen Angaben laden und die
    Registerinhalte in den Speicher schreiben.
  \item Arithmetische Instruktionen, die einfache arithmetische Operationen
    zwischen Registern veranlassen.
  \item Logische Instruktionen, die logische Verknüpfungen zwischen
    Registerinhalten oder Operationen auf Bit-Ebene in Registern anweisen.
  \item Vergleichsinstruktionen, die einen Vergleich zwischen
    Registerinhalten angeben.
  \item Sprunginstruktionen, die bedingt oder unbedingt sein können.
    Sie weisen die UMach Maschine an, die Programmausführung an einer anderen
    Stelle fortzufahren.
  \item Unterprogramm-Steuerung, bzw. Instruktionen, die die Ausführung von
    Unterprogrammen (Subroutinen) steuern.
  \item Systeminstruktionen, die die Unterstützung eines
    Betriebssystem ermöglichen.
  \item IO Instruktionen
\end{enumerate}

Die oben angegebenen Instruktionskategorien unterteilen den \gls{Befehlsraum} in
9 Bereiche. Es gibt $256$ mögliche Befehle, gemäß $2^{8} = 256$.
Die Verteilung der Kategorien auf die verschiedenen Maschinencode-Intervallen
wird in der Tabelle \ref{tab:Befehlraumverteilung} auf Seite
\pageref{tab:Befehlraumverteilung} angegeben.

\begin{table}
\index{Befehlsraum!Verteilungstabelle}
  \centering
  \begin{tabular}{|c|l|}                        \hline
    Maschinencodes   & Kategorie              \\\hline\hline
    \texttt{00 - 0F} & Kontrollbefehle        \\
    \texttt{10 - 4F} & Lade-/Speicherbefehle  \\
    \texttt{50 - 8F} & Arithmetische Befehle  \\
    \texttt{90 - AF} & Logische Befehle       \\
    \texttt{B0 - BF} & Vergleichsbefehle      \\
    \texttt{C0 - DF} & Sprungbefehle          \\
    \texttt{E0 - EF} & Unterprogrambefehle    \\
    \texttt{F0 - FF} & Systembefehle          \\\hline
  \end{tabular}
  \caption[Verteilung des Befehlsraums]
          {Verteilung des Befehlsraums nach Befehlskategorien.
          Die Zahlen sind im Hexadezimalsystem angegeben.}
  \label{tab:Befehlraumverteilung}
\end{table}




\begin{table}%no, this is not normal human being LaTeX
\newcolumntype{C}[1]{>{\ttfamily\footnotesize\centering\let\newline\\\arraybackslash\hspace{0pt}}p{#1}}
\newcolumntype{L}{>{\ttfamily\footnotesize}l}
\newcommand{\nhex}[1]{\multirow{2}{*}{#1}}
\centering
\begin{tabular}{|L||*{8}{C{1.2cm}|}}                                             \hline
          &   0   &   1   &   2   &   3    & 4      & 5      & 6       & 7       \\\hline\hline
\nhex{0}  &  NOP  &       &       &        &        &        &         &         \\\cline{2-9}
          &       &       &       &        &        &        &         &         \\\hline
\nhex{1}  &       &       &       &        &        &        &         &         \\\cline{2-9}
          &       &       &       &        &        &        &         &         \\\cline{1-9}
\nhex{2}  &       &       &       &        &        &        &         &         \\\cline{2-9}
          &       &       &       &        &        &        &         &         \\\cline{1-9}
\nhex{3}  &       &       &       &        &        &        &         &         \\\cline{2-9}
          &       &       &       &        &        &        &         &         \\\cline{1-9}
\nhex{4}  &  ADD  & ADDI  & ADDU  & ADDUI  &        &        &         &         \\\cline{2-9}
          &       &       &       &        &        &        &         &         \\\cline{1-9}
\nhex{5}  &       &       &       &        &        &        &         &         \\\cline{2-9}
          &       &       &       &        &        &        &         &         \\\cline{1-9}
\nhex{6}  &       &       &       &        &        &        &         &         \\\cline{2-9}
          &       &       &       &        &        &        &         &         \\\cline{1-9}
\nhex{7}  &       &       &       &        &        &        &         &         \\\cline{2-9}
          &       &       &       &        &        &        &         &         \\\cline{1-9}
\nhex{8}  &       &       &       &        &        &        &         &         \\\cline{2-9}
          &       &       &       &        &        &        &         &         \\\cline{1-9}
\nhex{9}  &       &       &       &        &        &        &         &         \\\cline{2-9}
          &       &       &       &        &        &        &         &         \\\cline{1-9}
\nhex{A}  &       &       &       &        &        &        &         &         \\\cline{2-9}
          &       &       &       &        &        &        &         &         \\\cline{1-9}
\nhex{B}  &       &       &       &        &        &        &         &         \\\cline{2-9}
          &       &       &       &        &        &        &         &         \\\cline{1-9}
\nhex{C}  &       &       &       &        &        &        &         &         \\\cline{2-9}
          &       &       &       &        &        &        &         &         \\\cline{1-9}
\nhex{D}  &       &       &       &        &        &        &         &         \\\cline{2-9}
          &       &       &       &        &        &        &         &         \\\cline{1-9}
\nhex{E}  &       &       &       &        &        &        &         &         \\\cline{2-9}
          &       &       &       &        &        &        &         &         \\\cline{1-9}
\nhex{F}  &       &       &       &        &        &        &         &         \\\cline{2-9}
          &       &       &       &        &        &        &         &         \\\hline\hline
          &   8   &   9   &   A   &   B    &   C    &   D    &    E    &    F    \\\hline
\end{tabular}
\caption[Instruktionentabelle]
        {Instruktionentabelle}
\label{tab:Instruktionentabelle}
\end{table} 


Die Tabelle \ref{tab:Befehlentabelle} auf der Seite 
\pageref{tab:Befehlentabelle} enthält eine Übersicht aller Befehle und
deren Maschinencodes.
Diese Tabelle wird folgenderweise gelesen:
in der am weitesten linken Spalte wird die erste hexadezimale Ziffer eines
Befehls angegeben (ein Befehl ist zweistellig im Hexadezimalsystem).
Jede solche Ziffer hat rechts Zwei Zeilen, die von links nach rechts gelesen
werden: eine Zeile für die Ziffern von 0 bis 8, die anderen für die übrigen
Ziffern 9 bis F (im Hexadezimalsystem). Die Assemblernamen (Mnemonics) der
einzelnen Befehle sind an der entsprechenden Stelle angegeben.

\paragraph{Definitionsstruktur}
Im den folgenden Abschnitten werden die einzelnen Instruktionen beschrieben. Zu
jeder Instruktion wird der \gls{Assemblername}, die Parameter, der Maschinencode
und das Instruktionsformat, das die Typen der Parameter definiert, formal
angegeben. Zudem werden Anwendungsbeispiele angegeben. Die Instruktionsformate
können im Abschnitt \ref{sec:Instruktionsformate} ab der Seite
\pageref{sec:Instruktionsformate} nachgeschlagen werden.

\paragraph{Zur Notation}
\index{Notation}
Mit \Reg\index{\Reg} wird die Menge aller Register
gekennzeichnet\footnote{Nicht verwechseln mit den Symbolen $\mathds{R}$ und
$\mathbb{R}$, die die Menge aller reellen Zahlen bedeuten.}.
Die Notation $X \in \Reg$ bedeutet, dass $X$ ein Element aus dieser Menge ist,
mit anderen Worten, dass $X$ ein Register ist. Analog bedeutet die
Schreibweise $X,Y \in \Reg$, dass $X$ und $Y$ beide Register sind.

Gilt $X, Y\in \Reg$ und ist $\sim$ eine durch einen Befehl definierte Relation
zwischen $X$ und $Y$, so bezieht sich die Schreibweise $X \sim Y$ nicht auf die
Maschinennamen von $X$ und $Y$, sondern auf deren Inhalte. Zum Beispiel, haben
die Register $R1$ und $R2$ die Maschinencodes \texttt{0x01} und \texttt{0x02}
und sind sie mit den Werten $4$ bzw. $5$ belegt, so bedeutet $R1 + R2$ das
gleiche wie $4 + 5 = 9$ und nicht \texttt{0x01 + 0x02 = 0x03}.

Genau so bezieht sich in einem Registerkontext eine Aussage wie \glqq $N$ Bytes
von der Adresse $A$\grqq\ nicht auf die Registernummer von $A$ und $N$, sondern
auf die Inhalte der Register $A$ und $N$. Wir haben diese vereinfachte
Schreibweise gewählt, da sie einigermassen leserlicher scheint als z.B. $*A$ und
$*N$ oder $\sim A$ und $\sim N$ oder ähnliche Schreibweisen.

Andere verwendeten Schreibweisen:
\begin{center}
\begin{tabular}{l|l}\toprule
 $\mathds{N}$ & Menge aller natürlichen Zahlen: $0,1,2,\ldots$         \\
 $\mathds{N}_{\setminus 0}$ 
              & $\mathds{N}$ ohne die Null: $1,2,\ldots$               \\
 $\mathds{Z}$ & Menge aller ganzen Zahlen: $\ldots,-2,-1,0,1,2,\ldots$ \\
 $\mathds{Z}_{\setminus 0}$ 
              & $\mathds{Z}$ ohne die Null: $\ldots,-2,-1,1,2,\ldots$  \\
 $N \in \mathds{N}$
              & $N$ ist Element von $\mathds{N}$, oder liegt im Bereich von
              $\mathds{N}$                                             \\
 $X \gets Y$  & $X$ wird auf $Y$ gesetzt                               \\
 $mem(X)$     & Speicherinhalt an der Adresse $X$ (1 Byte)             \\
 $mem_{n}(X)$ & $n$-Bytes-Block im Speicher ab Adresse $X$             \\
 \texttt{mem[n]}
              & äquivalent zu $mem(n)$                                 \\
\bottomrule
\end{tabular}
\end{center} 


\section{Arithmetische Instruktionen} 

\opdef{ADD}{$X,Y,Z \in \Reg$}{0x50}{RRR}
Vorzeichen behaftete Addition der Registerinhalte $Y$ und $Z$.
Das Ergebnis der Addition wird in das Register $X$ gespeichert.
Entspricht dem algebraischen Ausdruck
\[
    X \gets Y + Z
\]
Beispiel:
\begin{lstlisting}
  SET   R1 1     # $R1 \gets 1$
  SET   R2 2     # $R2 \gets 2$
  ADD   R3 R1 R2 # $R3 \gets R1 + R2 = 1 + 2 = 3$
  #     X  Y  Z
  SET   R2 -2    # $R2 \gets -2$
  ADD   R3 R3 R2 # $R3 \gets R3 + R2 = 3 +(-2) = 1$
  ADD   R3 R4  5 # Fehler! 5 kein Register
\end{lstlisting}
Vorzeichenlose Addition wird durch den Befehl \texttt{ADDU} ausgeführt.


\opdef{ADDU}{$X,Y,Z \in \Reg$}{0x51}{RRR}
\glqq Add Unsigned\grqq.
Vorzeichenlose Addition der Register $Y$ und $Z$. Das Ergebnis wird in das
Register $X$ gespeichert. Enthält $Y$ oder $Z$ ein Vorzeichen (höchstwertiges
Bit auf 1 gesetzt), so wird es nicht als solches interpretiert, sondern als
Wertigkeit, die zum Betrag des Wertes hinzuaddiert wird ($+2^{31}$).

\begin{lstlisting}
  SET   R1 1     # $R1 \gets 1$
  SET   R2 -2    # $R2 \gets -2$
  ADDU  R3 R1 R2 # $R3 \gets (1 + 2 + 2^{31}) = 2147483651$
\end{lstlisting}



\opdef{ADDI}{$X,Y\in\Reg$, $N\in\mathds{Z}$}{0x52}{RRN}
\glqq Add Immediate\grqq.
Hinzuaddieren eines festen vorzeichenbehafteten ganzzahligen Wert $N$ zum Inhalt
des Registers $Y$ und speichern des Ergebnisses in das Register $X$.
Entspricht dem algebraischen Ausdruck
\[
  X \gets Y + N
\]
$N$ wird als vorzeichenbehaftete 8-Bit Zahl in Zweierkomplement-Darstellung
interpretiert und kann entsprechend Werte von $-128$ bis $127$ aufnehmen.

Beispiel:
\begin{lstlisting}
  SET   R1 1     # $R1 \gets 1$
  ADDI  R2 R1 2  # $R2 \gets R1 + 2 = 1 + 2 = 3$
  #     X  Y  N
  ADDI  R2 R2 -3 # $R2 \gets R2 + (-3) = 3 +(-2) = 1$
  ADDI  R2 R3 R4 # Fehler! R4 kein $n \in \mathds{Z}$
\end{lstlisting}



\opdef{ADDIU}{$X,Y \in \Reg$, $N\in\mathds{N}$}{0x53}{RRN}
\glqq Add Unsigned Immediate\grqq.
Vorzeichenlose Addition des ganzzahligen Wertes $N$ zum Inhalt des Registers $Y$
und speichern des Ergebnisses in das Register $X$.
Der Inhalt des Registers $Y$, die Zahl $N$ und das Ergebnis $Y + N$ werden als
vorzeichenlose Werte interpretiert.
Die Feste natürliche Zahl $N$ kann Werte aus dem Bereich $[0, 255]$ aufnehmen.
Wird eine größere Zahl angegeben, so wird sie modulo $256$ berechnet.



\opdef{SUB}{$X, Y, Z \in \Reg$}{0x58}{RRR}
Subtrahiert die Registerinhalte von $Y$ und $Z$ und speichert das Ergebnis in
das Register $X$. Entspricht dem Ausdruck
\[
    X \gets (Y - Z)
\]
Wobei $X$, $Y$ und $Z$ Register sind.



\opdef{SUBU}{$X, Y, Z \in \Reg$}{0x59}{RRR}
\glqq Subtract Unsigned\grqq.
Analog zur Instruktion \opref{SUB} mit dem Unterschied, dass alle Werte und
Operationen vorzeichenlos sind.


\opdef{SUBI}{$X, Y \in \Reg$, $N\in\mathds{Z}$}{0x5A}{RRN}
\glqq Subtract Immediate\grqq.
Funktioniert wie \opref{SUB} aber $N$ ist eine direkt angegebene Zahl
(kein Register).

\paragraph{Beispiel}
Folgendes Beispiel demonstriert die Verwendung von \opref{SUBI} und zeigt
zugleich einen Fehler.
\begin{lstlisting}
  SET  R1 50     # $R1 \gets 50$
  SUBI R2 R1 30  # $R2 \gets (R1 - 30) = 20$
  SUBI R2 R1 R1  # Fehler! da $R1 \not\in \mathds{Z}$
\end{lstlisting}


\opdef{SUBIU}{$X, Y \in \Reg$, $N\in\mathds{N}$}{0x5B}{RRN}
\glqq Subtract Immediate Unsigned\grqq.
Funktioniert wie die Instruktion \opref{SUBI} mit dem Unterschied, dass
\opref{SUBIU} ausschliesslich mit vorzeichenlosen Werten arbeitet. 



\opdef{MUL}{$X, Y \in \Reg$}{0x60}{RR0}
\glqq Multiply\grqq. 
Multipliziert die Inhalte der Register $X$ und $Y$ und speichert das Ergebnis in
die Spezialregister \texttt{HI} und \texttt{LO}. Diese zwei Spezialregister
werden als eine 64-Bit Einheit betrachtet, wobei jedes eine Hälfte des
64-Bit Ergebnisses enthält.
Dabei werden die höchstwertigen 32 Bit des Ergebnisses in das Register
\texttt{HI}\index{HI@\texttt{HI}}
und die 32 niedrigstwertigen Bits des Ergebnisses in das Register
\texttt{LO}\index{LO@\texttt{LO}} gespeichert.
Siehe auch die Tabelle \ref{tab:Spezialregister} auf der Seite
\pageref{tab:Spezialregister}.

Falls das Ergebnis der Multiplikation gänzlich in den 32 Bit des Registers
\texttt{LO} passt, wird das Register \texttt{HI} trotzdem auf Null gesetzt.

Äquivalenter algebraischer Ausdruck:
\[
    (HI, LO) \gets X \cdot Y
\]

\paragraph{Beispiel} Der folgende Code demonstriert die Verwendung der
\texttt{MUL} Instruktion.
\begin{lstlisting}
  SET  R1 4   # $R1 \gets 4$
  SET  R2 5   # $R2 \gets 5$
  MUL  R1 R2  # $HI \gets 0$
              # $LO \gets 20$

  SET  R1 0xAAAAAAAA
  MUL  R1 R1  # $R1^{2}$
              # HI = 0x71C71C70
              # LO = 0xE38E38E4 
  COPY R2 LO  # $R1^{2} \bmod 2^{32}$ 
\end{lstlisting}
Falls es bekannt ist, dass das Ergebnis der Multiplikation sich mit 32 Bit
darstellen lässt, kann man den Weg über die \texttt{LO} und \texttt{HI} Register
mit der Instruktion \opref{MULD} umgehen.




\opdef{MULU}{$X, Y\in \Reg$}{0x61}{RR0}
\glqq Multiply Unsigned\grqq.
Funktioniert wie die Instruktion \opref{MUL} mit dem Unterschied, dass die
Multiplikationoperanden $X$ und $Y$ vorzeichenlos behandelt werden. 



\opdef{MULI}{$X \in \Reg$, $N\in\mathds{Z}$}{0x62}{RNN}
\glqq Multiply Immediate\grqq.
Multipliziert den Inhalt des Registers $X$ mit der ganzen Zahl $N$ und speichert
das $64$-Bit Ergebnis in die Register \texttt{HI} und \texttt{LO}, die als ein
einziges $64$-Register betrachtet werden: \texttt{HI} enthält die ersten $32$
Bits (die höchstwertigen) und \texttt{LO} die letzten 32 Bits (die
niedrigstwertigen).
Siehe auch die Instruktion \opref{MUL}.


\opdef{MULIU}{$X \in \Reg$, $N\in\mathds{N}$}{0x63}{RNN}
\glqq Multiply Immediate Unsigned\grqq.
Funktioniert wie die Instruktion \opref{MULI} mit dem Unterschied, dass sowohl
die Operanden $X$ und $N$ als auch das Ergebnis vorzeichenlos sind.


\opdef{MULD}{$X, Y, Z \in \Reg$}{0x64}{RRR}
\glqq Multiply Direct\grqq.
Multipliziert die Inhalte der Register $Y$ und $Z$ und speichert die
niedrigstwertigen 32 Bits des Ergebnisses in das Register $X$. Anders wie bei
der Instruktion \opref{MUL}, werden die Register \texttt{HI} und \texttt{LO}
nicht verändert.

\texttt{MULD} entspricht der Instruktionen:
\begin{lstlisting}
  MUL  Y Z
  COPY X LO
\end{lstlisting}
wobei der alte Wert von \texttt{LO} erhalten bleibt.

Algebraisch geschrieben:
\[
    X \gets (Y \cdot Z) \bmod 2^{32}
\]



\opdef{DIV}{$X, Y \in \Reg$}{0x68}{RR0}
\glqq Divide\grqq, ganzzahlige Division.
Dividiert den Inhalt des Registers $X$ durch den Inhalt des Registers $Y$ und
speichert den Quotient in das Register \texttt{HI} und den Rest in das Register
\texttt{LO}.
Nach der Ausführung gilt
\[
    X = Y \cdot HI + LO
\]
Algebraisch ausgedrückt:
\begin{align*}
  HI & \gets \left\lfloor X/Y \right\rfloor \\
  LO & \gets X \bmod Y
\end{align*}
$\lfloor x \rfloor$ bedeutet in diesem Kontext, dass $x$ auf die betragsmässig
nächstkleinste ganze Zahl gerundet wird, oder die Nachkommastellen von $x$
werden abgeschnitten.

Um den Weg über die Register \texttt{HI} und \texttt{LO} zu umgehen, dafür aber
den Rest der Division zu verlieren, kann man die Instruktion \opref{DIVD}
verwenden.

\paragraph{Beispiel}
Der folgende Code demonstriert die Verwendung von \texttt{DIV}.
\begin{lstlisting}
  SET R1 10   # $R1 \gets 10$
  SET R2  3   # $R2 \gets 3$
  DIV R1 R2   # $HI \gets 3$
              # $LO \gets 1$
\end{lstlisting}


\opdef{DIVU}{$X, Y \in \Reg$}{0x69}{RR0}
\glqq Divide Unsigned\grqq.
Funktioniert wie \opref{DIV} mit dem Unterschied, dass ganzzahlige
vorzeichenlose Division durchgeführt werden. Die Ergebnis-Register \texttt{HI}
und \texttt{LO} enthalten entsprechend vorzeichenlose Werte.



\opdef{DIVI}{$X \in \Reg$, $N\in\mathds{Z}_{\setminus 0}$}{0x6A}{RNN}
\glqq Divide Immediate\grqq.
Dividiert den Inhalt des Registers $X$ durch die feste ganze Zahl $N$ und
speichert den Quotient in das Register \texttt{HI} und den Rest in das Register
\texttt{LO}.
$N$ nimmt Werte aus dem Intervall $[-2^{15}, 2^{15}-1] \setminus 0$.
Nach der Durchführung der Division gilt:
\[
    X = HI \cdot N + LO 
\]
\paragraph{Beispiel}
Der folgende Code demonstriert die Verwendung von \texttt{DIVI}.
\begin{lstlisting}
  SET  R1 10   # $R1 \gets 10$
  DIVI R1  3   # $HI \gets 3$
               # $LO \gets 1$
\end{lstlisting}



\opdef{DIVIU}{$X \in \Reg$, $N\in\mathds{N}_{\setminus 0}$}{0x6B}{RNN}
\glqq Divide Immediate Unsigned\grqq.
Analog zur Instruktion \opref{DIVI} mit dem Unterschied, dass $X$ und $N$
vorzeichenlose Werte haben. Insbesondere nimmt $N$ Werte aus dem Intervall
$[1, 2^{16}-1]$.



\opdef{DIVD}{$X, Y, Z \in \Reg$}{0x6C}{RRR}
\glqq Divide direct\grqq.
Dividiert ganzzahlig den Inhalt des Registers $Y$ durch den Inhalt des Registers
$Z$ und speichert den Quotient in das Register $X$. Die Spezialregister
\texttt{HI} und \texttt{LO} werden nicht verändert.
Entspricht den Instruktionen:
\begin{lstlisting}
  DIV  Y  Z
  COPY X HI
\end{lstlisting}
Algebraische Schreibweise:
\[
    X \gets \left\lfloor \frac{Y}{Z}  \right\rfloor
\]

\paragraph{Beispiel}
für die Verwendung der Instruktion \texttt{DIVD}:
\begin{lstlisting}
  SET  R1 10    # $R1 \gets 10$
  SET  R2  3    # $R2 \gets 3$
  DIVD R3 R1 R2 # $R3 \gets \lfloor 10/3 \rfloor = 3$
  MOD  R4 R1 R3 # $R4 \gets ( 10 \bmod 3 ) = 1$
\end{lstlisting}



\opdef{MOD}{$X, Y, Z \in \Reg$}{0x70}{RRR}
Modulo Operation.
Berechnet den Rest der Division $Y/Z$ und speichert den Rest in das Register
$X$.
Äquivalent zu den Instruktionen:
\begin{lstlisting}
  DIVU Y Z
  COPY X LO
\end{lstlisting}

Algebraische Schreibweise:
\[
    X \gets Y \bmod Z
\]
oder
\[
    X \gets \left(
      Y - \left\lfloor \frac{Y}{Z}  \right\rfloor \cdot Z
      \right)
\]


\opdef{MODI}{$X, Y \in \Reg$, $N \in \mathds{N}$}{0x72}{RRN}
\glqq Modulo Immediate\grqq. Analog zur Instruktion \opref{MOD}, berechnet
\texttt{MODI} den Rest der ganzzahligen Division $Y / N$ und speichert ihn in
das Register $X$. Der Unterschied liegt darin, dass $N$ eine fest angegebene
natürliche Zahl ist.
\[
    X \gets Y \bmod N
\]


\opdef{ABS}{$X, Y \in \Reg$}{0x78}{RR0}
\glqq Absolute\grqq.
Speichert den absoluten Wert des Registers $Y$ in das Register $X$.
Algebraisch ausgedrückt:
\[
    X \gets
    \begin{cases}
      Y            & \text{ falls } Y \geq 0 \\
      (-1) \cdot Y & \text{ falls } Y < 0
    \end{cases}
\]



\opdef{NEG}{$X, Y \in \Reg$}{0x80}{RR0}
\glqq Negate\grqq.
Wechselt das arithmetische Vorzeichen des Registers $Y$ und speichert das
Ergebnis in das Register $X$. Entspricht der Zweierkomplement Bildung.
Algebraische Schreibweise:
\[
    X \gets \big( (-1) \cdot Y \big)
\]
Um eine bitweise Inversion zu erreichen (Einerkomplement), siehe die
Instruktion \opref{NOT}.



\opdef{INC}{$X \in \Reg$}{0x81}{R00}
\glqq Increment\grqq.
Inkrementiert den Inhalt des Registers $X$.
\[
    X \gets ( X + 1 )
\]



\opdef{DEC}{$X \in \Reg$}{0x82}{R00}
\glqq Decrement\grqq.
Dekrementiert den Inhalt des Registers $X$.
\[
    X \gets ( X - 1 )
\]




\section{Kontrollinstruktionen}

\opdef{NOP}{keine}{0x00}{000}
Diese Instruktion (\glqq No Operation\grqq) bewirkt nichts.


\opdef{EOP}{keine}{0x04}{000}
\glqq End Of Program\grqq.
Die Maschine ausschalten.

\section{Lade- und Speicherbefehle}
\label{sec:Lade-Speicher-Instruktionen}

\opdef{SET}{$X\in\Reg$, $N\in\mathds{Z}$}{0x10}{RNN}
Setzt den Inhalt des Registers $X$ auf den ganzzahligen Wert $N$.
Da $N$ mit 16 Bit und im Zweierkomplement dargestellt wird, kann $N$ Werte von
$-2^{15}$ bis $2^{15} - 1$ aufnehmen, bzw. von $-32768$ bis $+32767$.
Werte außerhalb dieses Intervalls werden auf Assembler-Ebene entsprechend
gekürzt (es wird modulo berechnet, bzw. nur die ersten 16 Bits aufgenommen).

Beispiele:
\begin{lstlisting}
label:
  SET R1  8    # $R1 \gets 8$
  SET R2 -3    # $R2 \gets -3$
  SET R3 65536 # $R3 \gets 0$, da $65536 = 2^{16} \equiv 0 \bmod 2^{16}$
  SET R4 70000 # $R3 \gets 4464 = 70000 \bmod 2^{16}$
  SET R7 label # Adresse 'label' ins R7
\end{lstlisting}


\opdef{CP}{$X, Y \in \Reg$}{0x11}{RR0}
Kopiert den Inhalt des Registers $Y$ in das Register $X$. Register $Y$ wird
dabei nicht geändert. Entspricht
\[
    X \gets Y
\]

Beispiel:
\begin{lstlisting}
 SET  R1 5  # $R1 \gets 5$
 CP   R2 R1 # $R2 \gets 5$
\end{lstlisting}



\opdef{LB}{$X, Y \in \Reg$}{0x12}{RR0}
Lade ein Byte aus dem Speicher mit Adresse $Y$ in das niedrigstwertige
Byte des Registers $X$. Die anderen Bytes von $X$ werden von diesem Befehl nicht
betroffen. Insbesondere, werden sie nicht auf Null gesetzt.

Äquivalenter C Code:
\begin{lstlisting}
  x = (x & 0xFFFFFF00) | (mem(y) & 0x00FF);
\end{lstlisting}

\paragraph{Beispiel}
Angenommen, der Speicher an den Adressen $100$ und $101$ hat den Wert $5$,
bzw. $6$.
\begin{lstlisting}
  SET  R1   100   # Basisadresse R1 = 100
  SET  R2     0   # Index R2 = 0
  SET  R3     0
  LB   R3 R1 R2   # R3 = 5 ($mem(100+0)$)
  SHLI R3 R3  8   # shift left 8 Bit, R3 = 1280
  INC  R2         # R2++, R2 = 1
  LB   R3 R1 R2   # R3 = 1286 (R3 + $mem(100+1)$)
\end{lstlisting}
Hier werden zwei nacheinander folgenden Bytes aus dem Speicher gelesen und in
die zwei niedrigstwertigen Bytes von $R3$ abgelegt.



\opdef{LW}{$X, Y \in \Reg$}{0x13}{RR0}
\glqq Load Word\grqq.
Lade ein Wort (4 Byte) aus dem Speicher mit Adresse $Y$ in das Register $X$.
Alle Bytes von $X$ werden dabei überschrieben.
Die Bytes aus dem Speicher werden nacheinander gelesen. Es werden also die
Bytes mit Adressen $Y + 0$, $Y + 1$, $Y + 2$ und $Y + 3$ zu
einem 4-Byte Wort zusammengesetzt und so in $X$ ablegt.

\paragraph{Beispiel}
Abgenommen, die Adressen von 100 bis 103 sind mit den Werten 0, 1, 2 und 3
belegt und bilden somit den Wert $66051$.
\begin{lstlisting}
  SET R1   100
  SET R2     0
  LW  R3 R1 R2  # $R3 \gets mem_{4}(R1 + R2) = 66051$
  LW  R3 R1  8  # Fehler! 8 ist kein Register nutze LWI dafuer
\end{lstlisting}




\opdef{SB}{$X, Y \in \Reg$}{0x14}{RR0}
\glqq Store Byte\grqq.
Speichert den Inhalt des niedrigstwertigen Byte von $X$ an der Speicherstelle
$Y$. $X$ und $Y$ sind dabei Register.

Entspricht dem algebraischen Ausdruck
\[
    X \to mem_{1} (Y)
\]
$mem_{1}(x)$ bedeutet dabei 1 Byte an der Adresse $x$.

\paragraph{Beispiel}
\begin{lstlisting}
  SET R1 128     # R1 = Speicheradresse 128
  SET R2 513     # R2 = 0x0201
  SB  R2 R1 ZERO # Speicher mit Adresse 128 wird auf 1 gesetzt
\end{lstlisting}
\texttt{ZERO} ist dabei ein Spezialregister mit konstantem Wert $0$.




\opdef{SW}{$X, Y\in \Reg$}{0x15}{RR0}
\glqq Store Word\grqq.
Speichert den Inhalt aller Bytes in $X$ an die Speicheradressen $Y$ bis 
$X + 3$.
\[
    X \to mem_{4}(Y)
\]
\paragraph{Beispiel}
Es wird das Register $R2$ mit dem Wert \texttt{0x01020304} geladen und an die
Adresse $128$ gespeichert. Dabei werden die Byte-Werten in 
\glqq big-endian\grqq\ Reihenfolge gespeichert: das höchstwertige Byte aus $R2$
(\texttt{0x01}) wird an der Adresse $128$ gespeichert, das niedrigstwertige
Byte (\texttt{0x04}) an die Adresse $131$.

\begin{lstlisting}
  SET R1 128         # R1 = Speicheradresse 128
  SET R2 0x01020304  # Wert zum Speichern
  SH  R2 R1 ZERO     # mem[128] = 0x01
                     # mem[129] = 0x02
                     # mem[130] = 0x03
                     # mem[131] = 0x04
\end{lstlisting}




\opdef{PUSH}{$X \in \Reg$}{0x18}{R00}
\glqq Push Word\grqq.
Erniedrigt das Register \texttt{SP} um 4 und kopiert das ganze Register $X$ auf
den Stack, wobei der \glqq Stack\grqq\ ist der Speicherbereich mit
Anfangsadresse in \texttt{SP}.
Die Byte-Reihenfolge der Lese- und Schreiboperationen ist \glqq Big-Endian\grqq\
und wird in der nachfolgenden Tabelle dargestellt:
\begin{center}
\begin{tabular}{l|cccc}
  \toprule
  $X$  Wertigkeiten &
  $2^{31} \leftrightarrow 2^{24}$ &
  $2^{23} \leftrightarrow 2^{16}$ &
  $2^{15} \leftrightarrow 2^{8}$  &
  $2^{7}  \leftrightarrow 2^{0}$ 
  \\
  &
  $\downarrow$ & $\downarrow$ & $\downarrow$ & $\downarrow$ 
  \\
  \text{Stack-Bereich} &
  \texttt{mem[SP + 0]} &
  \texttt{mem[SP + 1]} &
  \texttt{mem[SP + 2]} &
  \texttt{mem[SP + 3]}
  \\\bottomrule
\end{tabular}
\end{center}


Entspricht
\begin{align*}
  SP & \gets SP - 4    \\
  X  & \to mem_{4}(SP)
\end{align*}

\paragraph{Beispiel}
Der folgende Code speichert das 4-Byte Wort \texttt{0x01020304} auf den Stack.
Die Stack-Struktur wird in Kommentaren gezeigt.
\begin{lstlisting}
  SET  R1 0x01020304 # Wert zum pushen
  PUSH R1            # mem[SP + 0] = 0x01
                     # mem[SP + 1] = 0x02
                     # mem[SP + 2] = 0x03
                     # mem[SP + 3] = 0x04
\end{lstlisting}




\opdef{POP}{$X \in \Reg$}{0x19}{R00}
\glqq Pop Word\grqq.
Speichert 4 Bytes ab der Adresse \texttt{SP} in das Register $X$ und erhöht
\texttt{SP} um 4.
Die Byte-Reihenfolge der Lese- und Schreiboperationen ist \glqq Big-Endian\grqq\
und wird in der nachfolgenden Tabelle dargestellt.

\begin{center}
\begin{tabular}{l|cccc}
  \toprule
  $X$  Wertigkeiten &
  $2^{31} \leftrightarrow 2^{24}$ &
  $2^{23} \leftrightarrow 2^{16}$ &
  $2^{15} \leftrightarrow 2^{8}$  &
  $2^{7}  \leftrightarrow 2^{0}$ 
  \\
  &
  $\uparrow$ & $\uparrow$ & $\uparrow$ & $\uparrow$ 
  \\
  \text{Stack-Bereich} &
  \texttt{mem[SP + 0]} &
  \texttt{mem[SP + 1]} &
  \texttt{mem[SP + 2]} &
  \texttt{mem[SP + 3]}
  \\\bottomrule
\end{tabular}
\end{center}

Diese Instruktion kann algebraisch so ausgedrückt werden:
\begin{align*}
  X  & \gets mem_{4}(SP) \\
  SP & \gets SP + 4
\end{align*}
Die Instruktion \texttt{POP} ist äquivalent zu den folgenden Instruktionen:
\begin{lstlisting}
  LWI   X SP 0
  ADDI SP SP 4
\end{lstlisting}


\paragraph{Beispiel}
Angenommen, die ersten 4 Bytes vom Stack-Bereiche sind
\texttt{0xAA 0xBB 0xCC 0xDD}.
\begin{lstlisting}
  POPH R1      # R1 = 0xAABBCCDD
\end{lstlisting}






\appendix

\printindex

\end{document}
