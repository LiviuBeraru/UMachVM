\documentclass[fontsize=12pt,parskip=full,index=totoc]{scrreprt}

\addtokomafont{disposition}{\rmfamily}
\addtokomafont{descriptionlabel}{\rmfamily}

\usepackage[ilines,headsepline]{scrpage2}
\pagestyle{scrheadings}
\clearscrheadfoot
%\automark[section]{chapter}
\automark{section}
%\ihead{\leftmark}
\chead{\leftmark}
\cfoot{\pagemark}

\usepackage[utf8]{inputenc}
\usepackage[T1]{fontenc}
\usepackage[ngerman]{babel}

\usepackage{lmodern}
\usepackage{ellipsis}
\usepackage{microtype}
\usepackage{xcolor}
\usepackage{amsmath,amsfonts}

\usepackage{scrhack}%entfernt warnmeldungen von listings
\usepackage{listings}
\lstset{basicstyle=\small\ttfamily}
\lstset{commentstyle=\footnotesize\slshape}
\lstset{showstringspaces=false}
\lstset{frame=leftline}
\lstset{morecomment=[l][\color{blue!20!black}]{\#}}

\usepackage{makeidx}
\makeindex

\title{UMach Spezifikation}
\author{}

\usepackage[
    pdftitle={UMach VM Spezifikation},
    colorlinks=true,
    linkcolor=blue!50!black]
{hyperref} 

\usepackage[nonumberlist,toc]{glossaries}
\renewcommand*{\glspostdescription}{}% kein automatischer Punkt am Ende
\makeglossaries

% Glossary definitions
% Definition format: \newglossaryentry{label}{key=value...}
% Keys:
% name        - the name which will appear in text
% description - the description of the text, suround it with {}
% plural      - plural form
%
% Using a glossary term: 
% \gls{label}   for singular
% \glspl{label} for plural
% \Gls{label}   for first letter uppercase
% \Glspl{label} for first letter uppercase plural


\newglossaryentry{Adressierungsart}{
name={Adressierungsart},
description={Die Art, wie eine Instruktion die Umach Maschine dazu veranlasst,
einen Speicherbereich zu adressieren. Siehe auch Abschnitt
\ref{subsec:Adressierungsarten}.},
plural={Adressierungsarten}
}


\newglossaryentry{Byte}{
name=Byte,
description={Eine Reihe oder Gruppe von 8 Bit.},
plural={Bytes}
}

\newglossaryentry{Befehl}{
name=Befehl,
description={Die ersten 8 Bits in einer Instruktion. Operation code.},
plural={Befehle}
}

\newglossaryentry{Befehlsbreite}{
name=Befehlsbreite,
description={Die Länge eines Befehls in Bits. Ein UMach-Befehl ist 8 Bit lang.},
plural={Befehlsbreiten}
}

\newglossaryentry{Befehlsraum}{
name=Befehlsraum,
description={Die Anzahl der möglichen Befehlen, abhängig von der
Befehlsbreite. Beträgt die Befehlsbreite $8$ Bit, so ist der Befehlsraum
$2^{8} = 256$.},
plural=Befehlsräume
}

\newglossaryentry{Instruktion}{
name=Instruktion,
description={Eine Anweisung an die UMach VM etwas zu tun. Eine Instruktion
besteht aus einem Befehl (Operation Code) und eventuellen Argumenten.},
plural={Instructionen}
}

\newglossaryentry{Instruktionsbreite}{
name=Instruktionsbreite,
description={Die Länge einer Instruktion in Bits. Eine UMach-Instruktion ist 32 Bit lang.},
plural={Instruktionsbreiten}
}

\newglossaryentry{Instruktionsformat}{
name={Instruktionsformat},
description={Beschreibt die Struktur einer Instruktion auf Byte-Ebene und zwar
es gibt an, ob ein Byte als eine Registerangabe oder als reine
numerische Angabe zu interpretieren ist. Siehe \ref{sec:Instruktionsformate}.},
plural={Instruktionsformate}
}

\newglossaryentry{Instruktionssatz}{
name=Instruktionssatz,
description={Die Menge aller Instruktionen, die von der UMach Maschine
ausgeführt werden können.},
plural=Instruktionssätze
}

\newglossaryentry{Register}{
name={Register},
description={Eine sich im Prozessor befindende Speichereinheit. Der Register
ist dem Programmierer sichbar und kann mit Werten geladen werden. Siehe
Abschnitt \ref{subsec:Register}, Seite \pageref{subsec:Register}.},
plural={Register}
}


\usepackage{blindtext}

\begin{document}
\maketitle
\tableofcontents

%--- Jedes Kapitel in eigener Datei und eigenem Verzeichnis 
%--- Jede Kapitel-Datei fängt mit \chapter an (markiert Kapitel)
%--- und kann nach belieben andere \section enthalten
%--- Empfehlung: jeder Abschnitt \section in eigener Datei

\chapter{Einführung}
UMach ist eine einfache virtuelle Maschine (VM), die einen definierten Instruktionssatz
und eine definierte Architektur hat. UMach orientiert sich dabei an Prinzipien
von RISC Architekturen: feste Instruktionslänge,
kleine Anzahl von einfachen Befehlen, Speicherzugriff durch Load- und
Store-Befehlen, u.s.w.


Für den Anwender der virtuellen Maschine wird zuerst eine Assembler-Sprache zur
Verfügung gestellt. In dieser Sprache werden Programme geschrieben die
anschließend kompiliert werden. Die kompilierte Dateien (Maschinen-Code) wird
von der virtuellen Maschine ausgeführt.


\section{Anwendungsbeispiel}



\begin{lstlisting}
LOAD R1 90
LOAD R2 09
REV  R3 R1
\end{lstlisting}





\chapter{Organisation der UMach VM}
\blindtext

\subsection{Betriebsmodi}
\label{subsec:Betriebsmodi}
\index{Betriebsmodus}

Ein \gls{Betriebsmodus} bezieht sich auf die Art, wie die UMach VM grundsätzlich
läuft. Die UMach VM kann in einem der folgenden Betriebsmodi laufen:

\begin{enumerate}
  \item Normalmodus
  \item Einzelschrittmodus
\end{enumerate}

Der standard-Modus ist der Normalmodus.
Der Modus wird vor dem Hochfahren der Maschine festgelegt.

\paragraph{Normalmodus} Die virtuelle Maschine führt ohne Unterbrechung ein
Programm aus. Nach der Ausführung befindet sich die Maschine in einem
Wartezustand, falls sie nicht ausdrücklich ausgeschaltet wird.

\paragraph{Einzelschrittmodus} Die virtuelle Maschine führt immer eine einzige
Instruktion aus und nach der Ausführung wartet sie auf einen externen Signal um
mit der nächsten Instruktion fortzufahren. Dieser Modus soll dem Entwickler
erlauben, ein Programm schrittweise zu debuggen.



\section{Aufbau}
\index{UMach!Aufbau}
\label{sec:Aufbau}
Dieser Abschnitt beschreibt den Aufbau der UMach virtuellen Maschine.
Die virtuelle Maschine besteht aus internen und aus externen Komponenten. Dabei
sind die externen Komponenten nicht wesentlich für die Funktionsfähigkeit der
gesamten Maschine, d.h. die Maschine kann im Prinzip auch ohne die externen
Komponenten funktionieren -- in diesem Fall fehlt ihr eine Menge von Funktionen.

\paragraph{Interne Komponenten}
sind diejenigen Komponenten, die für die Funktionsfähigkeit der UMach
Maschine wesentlich sind:
\begin{enumerate}
  \item Recheneinheit
  \item Logische Einheit
  \item Register
\end{enumerate}


\paragraph{Externe Komponenten}

\begin{enumerate}
  \item Anbindung an einem I/O-Port
\end{enumerate}


\section{Register}
\label{sec:Register}
\index{Register}
\index{UMach!Register}

Die \glspl{Register} sind die Speichereinheiten im Prozessor.
Die meisten Anweisungen an die UMach Maschine operieren auf einer Art mit den
Registern.

Für alle Register gilt:
\begin{enumerate}
  \item Die Speicherkapazität beträgt 32 Bit.
  \item Es gibt eine eindeutige Zahl $n \in \mathds{N}$, die innerhalb der
    Maschine das Register identifiziert. Diese Zahl wird von einer Instruktion
    verwendet, wenn sie das Register anspricht.
  \item Die UMach Maschine erwartet die Angabe eines Registers als numerischer
    Wert. Jedoch verwendet der Programmierer der Maschine auf Assembler Ebene
    einen eindeutigen Namen dieses Registers.
\end{enumerate}

Die UMach Maschine hat zwei Gruppen von Registern: die Allzweckregister und
die Spezialregister. 


\subsection{Allzweckregister}
\index{Register!Allzweckregister}
\index{Allzweckregister}
Es gibt 32 Allzweckregister, die dem Programmierer zur Verfügung stehen.
Diese 32 Register werden beim Hochfahren der Maschine auf Null gesetzt. Außer
dieser Initialisierung, verändert die Maschine den Inhalt der Allzweckregister
nur auf explizite Anfrage, bzw. infolge einer Instruktion. 

\subsection{Spezialregister}





\subsection{Rechen- und logische Einheit}
\blindtext

\subsection{I/O-Einheit}
\blindtext





\chapter{Instruktionssatz}\index{Instruktionssatz}

In diesem Abschnitt werden alle Instruktionen (Befehle) der UMach VM
vorgestellt.

\section{Instruktionsformate}
\label{sec:Instruktionsformate}
\index{Instruktionsformat}

Eine Instruktion besteht aus einer Folge von 4 Bytes.
Das \gls{Instruktionsformat} beschreibt die Struktur einer Instruktion auf
Byte-Ebene. Das Format gibt an, ob ein Byte als eine Registerangabe oder als reine
numerische Angabe zu interpretieren ist.

\paragraph{Instruktionsbreite}
\index{Instruktionsbreite}
Jede UMach-Instruktion hat eine feste Bitlänge von 32 Bit (4 mal 8 Bit).
Instruktionen, die für ihren
Informationsgehalt weniger als 32 Bit brauchen, wie z.B. \texttt{NOP},
werden mit Nullbits gefüllt. Alle Daten und Informationen, die mit einer
Instruktion übergeben werden, müssen in diesen 32 Bit untergebracht werden.

\paragraph{Byte Order}
\index{Byte Order}
Die Byte Order (Endianness) der gelesenen \glspl{Byte} ist big-endian.
Die zuerst gelesenen 8 Bits sind die 8 höchstwertigen (Wertigkeiten $2^{31}$ bis
$2^{24}$) und die zuletzt gelesenen Bits sind die niedrigstwertigen
(Wertigkeiten $2^{7}$ bis $2^{0}$).
Bits werden in Stücken von $n$ Bits gelesen, wobei $n = k \cdot 8$ mit
$k \in \{1, 4\}$ (byteweise oder wortweise).


\paragraph{Allgemeines Format}
Jede \gls{Instruktion} besteht aus zwei Teilen: der erste Teil ist
8 Bit lang und entspricht dem tatsächlichen \gls{Befehl}, bzw. der Operation,
die von der UMach virtuellen Maschine ausgeführt werden soll.
Dieser 8-Bit-Befehl belegt also die 8 höchstwertigen Bits einer
32-Bit-Instruktion.  Die übrigen 24 Bits, wenn sie verwendet werden, werden
für Operanden oder Daten benutzt. Beispiel einer Instruktionszerlegung:

\begin{center}
  \begin{tabular}{|l|*{4}{c|}}
    \hline
    Instruktion (32 Bit) &
    \texttt{00000001} & \texttt{00000010} & \texttt{00000011} & \texttt{00000100}
    \\\hline
    Hexa  &
    \texttt{01}   & \texttt{02}   & \texttt{03}   & \texttt{04}
    \\\hline
    Byte Order &
    erstes Byte   & zweites Byte  & drittes Byte  & viertes Byte
    \\\hline
    Interpretation &
    Befehl (8 Bit) &  \multicolumn{3}{c|}{Operanden, Daten oder Füllbits}
    \\\hline
  \end{tabular}
\end{center}

Die Instruktionsformate unterscheiden sich lediglich darin, wie sie die 24 Bits
nach dem 8-Bit \gls{Befehl} verwenden. Das wird auch in der 3-buchstabigen
Benennung deren Formate wiedergeben.

In den folgenden Abschnitten werden die UMach-Instruktionsformate vorgestellt.
Jede Angegebene Tabelle gibt in der ersten Zeile die Reihenfolge der Bytes an. 
Die nächste Zeile gibt die spezielle Belegung der einzelnen Bytes an.



\subsection{000}
\label{subsec:000}
\index{Instruktionsformat!000}
\index{000}

\begin{center}
  \begin{tabular}{|*{4}{c|}} \hline
    erstes Byte & zweites Byte  & drittes Byte  & viertes Byte \\\hline\hline
    Befehl      & \multicolumn{3}{c|}{nicht verwendet}         \\\hline
  \end{tabular}
\end{center}

Eine Instruktion, die das Format 000 hat, besteht lediglich aus einem Befehl
ohne Argumenten. Die letzen drei Bytes werden von der Maschine nicht
ausgewertet und sind somit Füllbytes. Es wird empfohlen, die letzten 3 Bytes mit
Nullen zu füllen.




\subsection{NNN}
\label{subsec:NNN}
\index{Instruktionsformat!NNN}
\index{NNN}

\begin{center}
  \begin{tabular}{|*{4}{c|}}
    \hline
    erstes Byte  & zweites Byte  & drittes Byte  & viertes Byte \\\hline\hline
    Befehl       & \multicolumn{3}{c|}{numerische Angabe $N$}   \\\hline
  \end{tabular}
\end{center}

Die Instruktion im Format NNN besteht aus einem Befehl im ersten Byte und aus
einer numerischen Angabe  $N$ (einer Zahl), die die letzten 3 Bytes belegt.
Die Interpretation der numerischen Angabe wird dem jeweiligen Befehl überlassen.


\subsection{RNN}
\label{subsec:RNN}
\index{Instruktionsformat!RNN}
\index{RNN}

\begin{center}
  \begin{tabular}{|*{4}{c|}} \hline
    erstes Byte & zweites Byte  & drittes Byte  & viertes Byte   \\\hline\hline
    Befehl      & $R_{1}$ & \multicolumn{2}{c|}{numerische Angabe $N$} \\\hline
  \end{tabular}
\end{center}

Eine Instruktion im Format RNN besteht aus einem Befehl, gefolgt von einer
Register Nummer $R_{1}$, gefolgt von einer festen Zahl $N$, die die letzten
2 Bytes der Instruktion belegt.
Die genaue Interpretation der Zahl $N$ wird dem jeweiligen Befehl überlassen.
Zum Beispiel, die Instruktion
\begin{center}
  \begin{tabular}{|*{4}{c|}} \hline
    erstes Byte & zweites Byte  & drittes Byte  & viertes Byte \\\hline\hline
    \texttt{0x20} & \texttt{0x01} & \texttt{0x02} & \texttt{0x03} \\\hline
  \end{tabular}
\end{center}
wird folgenderweise von der UMach Maschine interpretiert: die Operation mit
Nummer \texttt{0x20} soll ausgeführt werden, wobei die Argumenten dieser
Operation sind das Register mit Nummer \texttt{0x01} und die numerische
Angabe \texttt{0x0203}.



\subsection{RRN}
\label{subsec:RRN}
\index{Instruktionsformat!RRN}
\index{RRN}

\begin{center}
  \begin{tabular}{|*{4}{c|}} \hline
    erstes Byte  & zweites Byte  & drittes Byte  & viertes Byte \\\hline\hline
    Befehl       & $R_{1}$       & $R_{2}$ & numerische Angabe $N$  \\\hline
  \end{tabular}
\end{center}
Eine Instruktion im Format RRN besteht aus einem Befehl, gefolgt von der
Angabe zweier Registers $R_{1}$ und $R_{2}$, jeweils in einem Byte, gefolgt von
einer numerischen Angabe $N$ (festen Zahl) im letzten Byte. Zum Beispiel, die
Instruktion
\begin{center}
  \begin{tabular}{|*{4}{c|}} \hline
    erstes Byte & zweites Byte  & drittes Byte  & viertes Byte \\\hline\hline
    \texttt{0x41} & \texttt{0x01} & \texttt{0x02} & \texttt{0x03} \\\hline
  \end{tabular}
\end{center}
soll wie folgt interpretiert werden: 
die Operation mit Nummer \texttt{0x41} soll ausgeführt werden, wobei die
Argumenten dieser Operation sind Register mit Nummer \texttt{0x01}, Register mit
Nummer \texttt{0x02} und die Zahl \texttt{0x03}.



\subsection{RRR}
\label{subsec:RRR}
\index{Instruktionsformat!RRR}
\index{RRR}

\begin{center}
  \begin{tabular}{|*{4}{c|}} \hline
    erstes Byte & zweites Byte  & drittes Byte  & viertes Byte \\\hline\hline
    Befehl      & $R_{1}$       & $R_{2}$       & $R_{3}$      \\\hline
  \end{tabular}
\end{center}

Eine Instruktion im Format RRR besteht aus der Angabe eines Befehls im ersten
Byte, gefolgt von der Angabe dreier Register $R_{1}$, $R_{2}$ und $R_{3}$ in den
jeweiligen folgenden drei Bytes.
Die Register werden als Zahlen angegeben und deren Bedeutung hängt vom
jeweiligen Befehl ab.


\subsection{Zusammenfassung}
\label{subsec:Instr-Formate-Zusammenfassung}
Im folgenden werden die Instruktionsformate tabellarisch zusammengefasst.

\index{Instruktionsformat!Liste}
\begin{center}
  \begin{tabular}{|l||*{4}{c|}}
    \hline
    Format & erstes Byte & zweites Byte  & drittes Byte  & viertes Byte
    \\\hline\hline
    000 & Befehl & \multicolumn{3}{c|}{nicht verwendet}                 \\\hline
    NNN & Befehl & \multicolumn{3}{c|}{numerische Angabe $N$}           \\\hline
    RNN & Befehl & $R_{1}$ & \multicolumn{2}{c|}{numerische Angabe $N$} \\\hline
    RRN & Befehl & $R_{1}$ & $R_{2}$ & numerische Angabe $N$            \\\hline
    RRR & Befehl & $R_{1}$ & $R_{2}$ & $R_{3}$                          \\\hline
  \end{tabular}
\end{center}





\section{Instruktionen}\index{Instruktionen}

Zur besseren Übersicht der verschiedenen UMach-\glspl{Instruktion}, unterteilen
wir den \gls{Instruktionssatz} der UMach virtuellen Maschine in den folgenden
Kategorien:
\index{Instruktionen!Kategorien}

\begin{enumerate}
  \item Kontrollinstruktionen,  die die Maschine in ihrer Gesamtheit
    steuern, wie z.B. den Betriebsmodus umschalten oder Ausschalten.
  \item Arithmetische Instruktionen, wie z.B. Addieren, Subtrahieren etc.
    Alle arithmetische Instruktionen operieren auf Register Inhalte.
  \item Logische Instruktionen, die eine logische Verknüpfung zwischen den Inhalten
    von Registern veranlassen.
  \item Vergleichsinstruktionen: Vergleichen von Registerinhalten.
  \item Sprünge im Programmcode.
  \item Load- und Store-Instruktionen die einzigen, die einen Zugriff auf den
    Speicher ausführen.
  \item Input-Output-Instruktionen operieren auf die I/O-Einheit der UMach VM.
  \item Andere Befehle: diese Kategorie enthält Instruktionen, die in keiner
    anderen Kategorie passen und zukünftige Erweiterungen.
\end{enumerate}

\paragraph{Verteilung des Befehlsraums}
\index{Befehlsraum}
Die oben angegebenen Instruktionskategorien unterteilen den \gls{Befehlsraum} in
8 Bereiche.
Es gibt $256$ Befehle, gemäß $2^{8} = 256$.


Im den folgenden Abschnitten werden die einzelnen Instruktionen beschrieben.
Zu jeder Intruktion wird der \glqq Mnemonic Code\grqq, der Maschinen Code, das
Instruktionsformat und Verwendungsbeispiele angegeben.







\appendix

\printglossary[title=Glossar,toctitle=Glossar,style=altlisthypergroup]
\printindex

\end{document}
