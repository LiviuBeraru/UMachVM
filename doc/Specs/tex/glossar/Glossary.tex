\usepackage[nonumberlist,toc]{glossaries}
\renewcommand*{\glspostdescription}{}% kein automatischer Punkt am Ende
\makeglossaries

% Glossary definitions
% Definition format: \newglossaryentry{label}{key=value...}
% Keys:
% name        - the name which will appear in text
% description - the description of the text, suround it with {}
% plural      - plural form
%
% Using a glossary term: 
% \gls{label}   for singular
% \glspl{label} for plural
% \Gls{label}   for first letter uppercase
% \Glspl{label} for first letter uppercase plural


\newglossaryentry{Adressierungsart}{
name={Adressierungsart},
description={Die Art, wie eine Instruktion die Umach Maschine dazu veranlasst,
             einen Speicherbereich zu adressieren. Siehe auch Abschnitt
             \ref{subsec:Adressierungsarten}.},
plural={Adressierungsarten}
}

\newglossaryentry{Assemblername}{
name={Assemblername},
description={Der Name eines Registers oder eines Befehls, so wie er in einem
             textuellen Programm (ASCII) verwendet wird.
             $R1$, $R2$, $ADD$ sind Assemblernamen von Registern und
             Befehlen.},
plural={Assemblernamen}
}


\newglossaryentry{Byte}{
name=Byte,
description={Eine Reihe oder Gruppe von 8 Bit.},
plural={Bytes}
}

\newglossaryentry{Befehl}{
name=Befehl,
description={Die ersten 8 Bits in einer Instruktion. Operation code.},
plural={Befehle}
}

\newglossaryentry{Befehlsbreite}{
name=Befehlsbreite,
description={Die Länge eines Befehls in Bits. Ein UMach-Befehl ist 8 Bit lang.},
plural={Befehlsbreiten}
}

\newglossaryentry{Befehlsraum}{
name=Befehlsraum,
description={Die Anzahl der möglichen Befehle, abhängig von der
             Befehlsbreite. Beträgt die Befehlsbreite $8$ Bit, so ist der
             Befehlsraum $2^{8} = 256$.},
plural=Befehlsräume
}

\newglossaryentry{Betriebsmodus}{
name={Betriebsmodus},
description={Die Art, wie die UMach Maschine die einzelnen Instruktionen
             abarbeitet.
             Siehe auch \ref{subsec:Betriebsmodi}.},
plural={Betriebsmodi}
}

\newglossaryentry{Instruktion}{
name=Instruktion,
description={Eine Anweisung an die UMach VM etwas zu tun. Eine Instruktion
             besteht aus einem Befehl (Operation Code) und eventuellen
             Argumenten.},
plural={Instructionen}
}

\newglossaryentry{Instruktionsbreite}{
name=Instruktionsbreite,
description={Die Länge einer Instruktion in Bits.
             Eine UMach-Instruktion ist 32 Bit lang.},
plural={Instruktionsbreiten}
}

\newglossaryentry{Instruktionsformat}{
name={Instruktionsformat},
description={Beschreibt die Struktur einer Instruktion auf Byte-Ebene und zwar
             es gibt an, ob ein Byte als eine Registerangabe oder als reine
             numerische Angabe zu interpretieren ist.
             Siehe \ref{sec:Instruktionsformate}.},
plural={Instruktionsformate}
}

\newglossaryentry{Instruktionssatz}{
name=Instruktionssatz,
description={Die Menge aller Instruktionen, die von der UMach Maschine
             ausgeführt werden können.},
plural=Instruktionssätze
}


\newglossaryentry{Maschinenname}{
name={Maschinenname},
description={Der Name eines Registers oder eines Befehls, so wie er im
             Maschinencode erscheint.
             \texttt{0x01}, \texttt{0x02}, \texttt{0x40} sind Maschinennamen
             von Registern und Befehlen.},
plural={Maschinennamen}
}


\newglossaryentry{Register}{
name={Register},
description={Eine sich im Prozessor befindende Speichereinheit. Das Register
            ist dem Programmierer sichbar und kann mit Werten geladen werden.
            Siehe Abschnitt \ref{sec:Register}, Seite \pageref{sec:Register}.},
plural={Register}
}
