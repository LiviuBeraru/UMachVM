
\section{Meist verwendete Befehle}

\begin{frame}{\texttt{SET}}
 \texttt{SET R1 5} oder \texttt{SET R1 label}
 
 Setzt das Register R1 auf den angegebenen Wert. Labels werden durch Adressen ersetzt.
 
\end{frame}

\begin{frame}{Arithmetische Befehle}

\begin{center}
\begin{tabular}{@{\ttfamily\bfseries}ll}
ADD R1 R2 R3 & Addiert $R1 \gets R2 + R3$ \\
SUB R1 R2 R3 & Subtrahiert $R1 \gets R2 - R3$\\
INC R1       & Inkrementiert $R1++$\\
DEC R1       & Dekrementiert $R1--$\\
MUL R1 R2    & Multiplikation \\
DIV R1 R2    & Division durch 0 führt zu Interrupt
\end{tabular}
\end{center}
 
und die entsprechenden Immediate-Varianten
\end{frame}

\begin{frame}{Bedingte Sprünge}
\begin{itemize}
 \item{\texttt{\texttt{CMP R1 R2}}}
        Vergleicht das Register R1 mit R2.
 \item{\texttt{\texttt{BL label}}}
        Springt zum angegebenen Label, falls R1 kleiner R2 ist.
        Weitere Möglichkeiten: \texttt{BLE, BG, BGE, BE}
\end{itemize}


\end{frame}

\begin{frame}{Unbedingte Sprünge}
 \begin{itemize}
  \item \texttt{JMP label}
  \item \texttt{CALL funktion} 
 \end{itemize}
\end{frame}

\begin{frame}{IO-Befehle}
 \begin{itemize}
  \item \texttt{IN R1 R2 ZERO}
  \item \texttt{OUT R1 R2 ZERO}
 \end{itemize}
\end{frame}






\section{Hilfsfunktionen}

\begin{frame}{Hilfsfunktionen}
 \begin{itemize}
  \item inputint
  \item printint
  \item putchar
  \item newline
 \end{itemize}
\end{frame}



\section{Hello World}

\begin{frame}[fragile]{Hello World}
 \begin{lstlisting}
  SET R1 hello
  SET R2 13
  OUT R1 R2 ZERO
  .data
  .string hello "Hello World!"
 \end{lstlisting}
\end{frame}


\section{Fibonacci Zahlen}

\begin{frame}{\insertsection}% name der aktuellen section
 Folge $X_{n} = X_{n-1} + X_{n-2}$
 mit $X_{1} = 1$, und $X_{2} = 2$
 \begin{itemize}
 \item Unterscheidung zwischen $n = 1$, $n = 2$ und $n \geq 3$ notwendig
 \end{itemize}
\end{frame}



\section{Zahl raten}

\begin{frame}{\insertsection}% name der aktuellen section
  \begin{itemize}
   \item Seed erzeugen
   \item Pseudozufallszahl generieren
   \item Formel: $X_{n+1} = (a + b*X_n) \mod m$
   \item Spieler rät die Zahl
   \item Rückmeldung ob die geratene Zahl größer oder kleiner der gesuchten ist
   \item Anzahl der Versuche
  \end{itemize}
\end{frame}


\section{Tic Tac Toe}

\begin{frame}{\insertsection}
 Belegung der Register:
  \begin{itemize}
   \item R1-R9: Spielfelder
   \item R10: Aktueller Spieler
   \item R20: Anzahl der Spielzüge
  \end{itemize}

 Spielzyklus:
  \begin{enumerate}
  \item Eingabe
  \item Ausgabe
  \item Auswertung
  \end{enumerate}


Am Ende des Spiels:
\begin{itemize}
 \item neue Runde oder Beenden
 \item bei neuer Runde aufräumen
\end{itemize}
\end{frame}


