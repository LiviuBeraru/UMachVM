
\section{Zielsetzung}

\begin{frame}{Das Ziel}
 \begin{itemize}
   \item Eine komplette virtuelle Maschine soll entworfen,
         dokumentiert und implementiert werden.
   \item Die Maschine soll praktisch benutzbar sein -- man sollte Programme
         assemblieren und ausführen können.
 \end{itemize}
\end{frame}

\begin{frame}[fragile]{Gewünschter Workflow}
 \begin{enumerate}
  \item Assembler Programm editieren.
\begin{lstlisting}
 loop:   SET R1 137
         CMP R1 ZERO
         BE  finish
         DEC R1
         JMP loop
 finish: EOP
\end{lstlisting}
  \item Das Programm assemblieren.
\begin{lstlisting}
 uasm -o myprog.umx myprog.uasm
\end{lstlisting}
  \item \glqq{}Bytecode\grqq{} ausführen.
\begin{lstlisting}
 umach -v myprog.umx
\end{lstlisting}
 \end{enumerate}

\end{frame}


\section{Was wird geliefert}

\begin{frame}{\insertsection}
 \begin{enumerate}
  \item Spezifikation der Maschine
  \item Spezifikation des Assemblers
  \item Maschine in C99
  \item Assembler in C99
  \item Debugger (integriert und als Qt-Anwendung)
  \item Demos
 \end{enumerate}
\end{frame}



\section{Ähnliche Werke}

\begin{frame}{\insertsection}
 \begin{description}
  \item[JVM]
    Die Java Virtual Machine (virtuelle Stackmaschine).
  \item[MMIX]
    Wurde von Donald Knuth entwickelt.
    
    Wird in \glqq The Art of Computer Programming\grqq{} als hypothetischer
    Rechner benutzt.

    64 Bit, RISC, 256 Befehle, 256 Register, MMIXAL als ASM-Sprache.
    
    Kennt keiner.
    
    Wichtigste Inspirationsquelle.
 \end{description}

\end{frame}




