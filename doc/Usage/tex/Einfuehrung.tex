\section{Einführung}

\subsection{Ein Beispiel}
Gleich am Anfang soll ein Beispiel für die Verwendung der UMach
virtuellen Maschine und des Assemblers gegeben werden.

Wir haben ein UMach-Programm in eine normale Textdatei geschrieben. Das Programm
kann sich über mehrere Dateien erstrecken, aber hier verwenden wir nur eine
Datei. Das Programm sieht wie folgt aus:

\lstinputlisting[caption={Ein einfaches Beispiel}]{progs/prog1.uasm}

Dieses Programm wurde in der Datei \texttt{prog1.uasm} gespeichert (die Endung
der Datei ist eigentlich egal). Wir gehen davon aus, dass der Assembler
\texttt{uasm}, die Programmdatei \texttt{prog1.uasm} und die virtuelle Maschine
\texttt{umach} sich in dem selben Verzeichniss befinden. Sonst muss man die
Befehle entsprechend anpassen.

Das Programm kann wie folgt assembliert werden:
\begin{lstlisting}
 ./uasm -o prog.ux prog1.uasm
\end{lstlisting}

Die Option \texttt{-o} gibt die Ausgabedatei an. Wird diese Option nicht
angegeben, so wird das assemblierte Programm in die Datei \texttt{u.out}
geschrieben. Ergebnis des Assemblers is also die Datei \texttt{prog.ux}. Jetzt
kann man diese Datei \glqq ausführen\grqq:
\begin{lstlisting}
 ./umach prog.ux
\end{lstlisting}
Das Programm beendet sich ohne Ausgabe. Starten wird also das Programm im
Debug-Modus (Option \texttt{-d}):
\begin{lstlisting}
 ./umach -d prog.ux
\end{lstlisting}

Es wird die erste Anweisung angezeigt, der aktuelle Programm-Counter (Inhalt
des Registers \texttt{PC}) und ein Prompt, der auf eine Eingabe von uns wartet.
So könntes es weiter gehen:

\lstinputlisting[nolol]{progs/prog1.output}
