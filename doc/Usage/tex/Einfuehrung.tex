\chapter{Einführung}

\section{Beispiel}
Gleich am Anfang soll ein Beispiel für die Verwendung der UMach
virtuellen Maschine und des Assemblers gegeben werden.

Wir haben ein UMach-Programm in eine normale Textdatei geschrieben. Das Programm
kann sich über mehrere Dateien erstrecken, aber hier verwenden wir nur eine
Datei. Das Programm sieht wie folgt aus:

\begin{lstlisting}
    set r1 3
loop:
    dec r1
    cmp r1 zero
    be  finish
    jmp loop
finish: 
    EOP
\end{lstlisting}

Dieses Programm wurde in der Datei \texttt{text.um} gespeichert (die Endung ist
egal). Wir gehen davon aus, dass der Assembler \texttt{uasm}, die Programmdatei
\texttt{test.um} und die virtuelle Maschine \texttt{umach} sich in dem selben
Verzeichniss befinden. Sonst muss man die Befehle entsprechend anpassen.

Das Programm kann wie folgt assembliert werden:
\begin{lstlisting}
 ./uasm -o prog.ux test.um
\end{lstlisting}
Die Option \texttt{-o} gibt die Ausgabedatei an. Wird diese Option nicht
angegeben, so wird das assemblierte Programm in die Datei \texttt{u.out}
geschrieben. Ergebnis des Assemblers is also die Datei \texttt{prog.ux}. Jetzt
kann man diese Datei \glqq ausführen\grqq:
\begin{lstlisting}
 ./umach prog.ux
\end{lstlisting}
Das Programm beendet sich ohne Ausgabe. Starten wird also das Programm im
Debug-Modus (Option \texttt{-d}):
\begin{lstlisting}
 ./umach -d prog.ux
\end{lstlisting}
Es wird die erste Anweisung angezeigt, der aktuelle Programm-Counter (Inhalt
des Registers \texttt{PC}) und ein Prompt, der auf eine Eingabe von uns wartet.
So könntes es weiter gehen:
\begin{lstlisting}
[256]   SET   R1    3
umdb > show reg r1
R1 = 0x00000000 = 0
umdb > s
[260]   DEC   R1
umdb > show reg r1
R1 = 0x00000003 = 3
umdb > s
[264]   CMP   R1    ZERO  
umdb > s
[268]   BE    2
umdb > s
[272]   JMP   -3
umdb > s
[260]   DEC   R1
umdb > s
[264]   CMP   R1    ZERO  
umdb > show reg r1
R1 = 0x00000001 = 1
umdb > s
[268]   BE    2
umdb > s
[272]   JMP   -3
umdb > s
[260]   DEC   R1
umdb > s
[264]   CMP   R1    ZERO  
umdb > s
[268]   BE    2
umdb > s
[276]   EOP   
umdb > s
umdb > s
The maschine is not running.
umdb > show reg r1 cmpr
R1 = 0x00000000 = 0
CMPR = 0x00000000 = 0
umdb > q
\end{lstlisting}
