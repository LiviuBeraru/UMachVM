\section{Assembler}

\subsection{Eingabe Dateien}

Es können beliebig viele Programmdateien angegeben werden. Sie werden der
Reiche nach abgearbeitet. Man beachte, dass die Instruktion \texttt{EOP} das
Ende des Programms signalisiert. Falls nachher noch weitere Befehle, eventuell
in anderen Datein, angegeben werden, werden diese zwar assembliert, aber nicht
ausgeführt.

Alles außer Labels is case insensitive. Man kann beliebig Leerzeichen
(whitespace) verwenden.

\subsection{Labels}
\index{Labels}

Der UMach-Assembler unterstüzt die Verwendung von sogenannten \glqq
Labels\grqq, oder Sprungmarken. Um ein Label im Programmcode zu definieren
schreibt man auf einer getrennten Zeile den Labelnamen, gefolgt von einem
Doppelpunkt.

Labelname ist höchstens 128 Buchstaben lang.

Labels müssen keine whitespace beinhalten.

Mehrfache Labels, die auf den selben Befehl zeigen, werden unterstüzt. Sie
müssen allergings auf verschiedenen Zeilen deklariert werden, sonst wird der
zweite Label als ein Befehl gelesen.

\subsection{Programmdaten}
\index{Programmdaten}

Daten werden nach der Anweisung \texttt{.data} angegeben.

\subsubsection{Strings}
\index{Strings}

String Daten werden mit der Anweisung \texttt{.string}\index{.string} angegeben.
Nach \texttt{.string} kommt der Name des Strings und dann der eigentlich String
zwischen Anführungszeichen. Siehe das Programm \ref{lst:Datenverwendung} für
ein Beispiel.

Strings werden so assembliert, dass sie immer ein Vielfaches von 4 Bytes
belegen. Braucht der textuelle Inhalt des Strings weniger als 4 Byte, so wird
der String mit Nullbytes gefüllt.


\subsubsection{Ganze Zahlen}
\index{Zahlen}

Ganze Zahlen werden mit der Anweisung \texttt{.int}\index{.int} angegeben. Nach
\texttt{.int} folgt der Name (Label) der Zahl, dann die eigentliche Zahl. Diese
kann in Hexa-, Oktal- oder Dezimalsystem angegeben werden, analog wie in der
C/Java Sprache.

\lstinputlisting[caption={Verwendung der Daten},label={lst:Datenverwendung}]
{progs/example_data.uasm}

Angenommen, der Assembler \texttt{uasm}, die virtuelle Maschine
\texttt{umach} und das Programm \texttt{example\_data.uasm} befinden sich im
aktuellen Verzeichniss, kann das Programm \ref{lst:Datenverwendung}  wie folgt
assembliert und ausgeführt werden:
\begin{lstlisting}
./uasm example_data.uasm
./umach u.out
\end{lstlisting}
Es wird lediglig \glqq Hallo Welt\grqq\ ausgegeben.
