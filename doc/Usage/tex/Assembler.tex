\section{Assembler}

\subsection{Eingabe Dateien}

Es können beliebig viele Assemblercode-Dateien angegeben werden. Sie werden der
Reihe nach assembliert. Man beachte, dass die Instruktion \texttt{EOP} das
Ende des Programms signalisiert und die Virtuelle Maschine anhält, falls dieser
Befehl erreicht wird.

Im Assemblercode wird nicht zwischen Groß- und Kleinschreibung unterschieden,
wobei Namen von Sprungmarken und Variablen eine Ausnahme darstellen.

Siehe den Abschnitt \ref{subsubsec:Funktionen-Beispiel} ab der Seite
\pageref{subsubsec:Funktionen-Beispiel} für ein Beispiel, in dem mehrere
Assemblercode-Dateien für ein Programm verwendet werden.

\subsection{Sprungmarken}

Der UMach-Assembler unterstützt die Verwendung von sogenannten Sprungmarken.
Eine Sprungmarke markiert das Sprungziel für die verschiedenen Sprungbefehle.

Um eine Sprungmarke im Programmcode zu definieren, schreibt man den Namen der
Sprungmarke, gefolgt von einem Doppelpunkt. Namen von Sprungmarken dürfen keine
Whitespaces (Leerzeichen oder Tabulatoren) beinhalten.

Folgen von Sprungmarken, die auf den selben Befehl zeigen, werden unterstützt.
Dabei muss jede Sprungmarke in einer eigenen Zeile definiert sein. Sprungmarken
zeigen immer auf den ersten Befehl der nach der Definition der Sprungmarke
auftaucht. Das Programm \ref{lst:example_labels} zeigt ein Beispiel für die 
Verwendung von (mehrfachen) Sprungmarken.

\lstinputlisting[caption={Beispiel für Sprungmarken},label={lst:example_labels}]
{progs/example_labels.uasm}

In diesem Beispiel zeigen die Sprungmarken \texttt{loop} und \texttt{do} auf den
gleichen Befehl: \texttt{cmp r1 zero}. Analog zeigen die Sprungmarken
\texttt{finish} und \texttt{end} auf den Befehl \texttt{EOP}.

\subsection{Variablen}

Variablen werden nach der Anweisung \texttt{.data} angegeben. Diese Anweisung
muss alleine in einer Zeile stehen.

Die Variablen können auf verschiedenen Assemblercode-Dateien verteilt werden,
sie werden vom Assembler zusammengefügt und ans Ende der assemblierten Datei
geschrieben.

Alle Variablen haben jeweils eine Länge, die ein Vielfaches von 4 Byte darstellt.
Bedarf ein Datenelement weniger als $4 \times n$ Bytes an Speicherplatz, so wird
es trotzdem auf eine durch 4 teilbare Länge mit Nullbytes gefüllt.

\subsubsection{Strings}

String Variablen werden mit der Anweisung \texttt{.string} definiert. Es folgt
der Name der String Variable und abschließend der Initialisierungswert in 
doppelten Anführungszeichen. Siehe das Programm \ref{lst:Datenverwendung} für ein
Beispiel.

String Variablen werden von dem Assembler automatisch mit einem Nullbyte
terminiert.

\subsubsection{Ganze Zahlen}

Integer Variablen werden mit der Anweisung \texttt{.int} definiert. Es folgt
der Name der Integer Variable und abschließend der Initialisierungswert. Dieser
kann im Hexa-, Oktal- oder Dezimalsystem angegeben werden, analog zu der
C oder Java Syntax (\texttt{0x} bzw. \texttt{0} Präfix).

\lstinputlisting[caption={Verwendung von Variablen},label={lst:Datenverwendung}]
{progs/example_data.uasm}

Angenommen, der Assembler \texttt{uasm}, die virtuelle Maschine \texttt{umach}
und das Programm \texttt{example\_data.uasm} befinden sich im aktuellen
Arbeitsverzeichniss, kann das Programm \ref{lst:Datenverwendung}  wie folgt
assembliert und ausgeführt werden:
\begin{lstlisting}
./uasm example_data.uasm
./umach u.out
\end{lstlisting}
Es wird lediglich \glqq Hallo Welt\grqq\ ausgegeben.

\subsection{Funktionen}

Der UMach Assembler unterstützt indirekt die Verwendung von Funktionen.

\subsubsection{Aufbau einer Funktion}

Eine Funktion auf der Assembler-Ebene ist nichts anderes als eine Codesequenz,
die mit einer Sprungmarke versehen ist und deren letzte Instruktion ein
\texttt{RET} ist (RET wie ``Return''). Zu diesem Code wird mit der Anweisung
\texttt{CALL} gesprungen. Der Name der Sprungmarke ist auch Name der Funktion.

Die virtuelle Maschine UMach kennt keine Funktionen, sondern nur einzelne
Maschinenbefehle. Sie kennt also auch keine Argumente, Gültigkeitsbereiche oder
Rückgabewerte. All diese Elemente, welche in höheren Programmiersprachen eine
Funktion (oder Methode) ausmachen, müssen also vom Programmierer selbst
implementiert werden. Jeder Programmierer ist frei in der Entscheidung, wie er
diese Elemente implementiert.

Um diese Entscheidungen zu erleichtern, kann man einige Konventionen befolgen.
Eine solche Konvention wird in den folgenden Punkten gegeben. Allerdings steht
es dem Benutzer frei, sich andere Konventionen und Regeln auszudenken.

\paragraph{Name der Funktion}
Der Name der Funktion wird als Sprungmarke angeben. Eine Funktion beginnt also
wie folgt:
\begin{lstlisting}
funktionsname:
    anweisung
    anweisung
    ...
\end{lstlisting}

\paragraph{Ende der Funktion}

Das Ende einer Funktion wird durch den Befehl \texttt{RET} markiert.
Für mehr Informationen betreffend dieses Befehls siehe die Spezifikation der
UMach Maschine (Abschnitt 3.8.3).

\paragraph{Argumente}

Die Argumente einer Funktion werden in umgekehrter Reihenfolge vor dem Aufruf
auf den Stack abgelegt\footnote{Dies entspricht übrigens gängiger Praxis.}.
Innerhalb der Funktion werden die Argumente mithilfe des Registers \texttt{SP}
(Stack Pointer) und eines Offsets abgelesen.

Möchte man z.B. eine Funktion namens \texttt{max} mit zwei Argumenten aufrufen
und befindet sich das erste Argument in \texttt{R1} und das zweite in \texttt{R2},
kopiert man vor dem Aufruf diese zwei Argumente in umgekehrter Reihenfolge auf
den Stack:
\begin{lstlisting}
PUSH R2
PUSH R1
CALL max
\end{lstlisting}
In der Funktion werden die Adressen der Argumente anhand eines Offsets relativ
zum Register \texttt{SP} berechnet und dann mit dem Befehl \texttt{LW} (Load Word)
geladen. Dabei muss man folgendes beachten:
\begin{enumerate}
 \item Durch den Befehl \texttt{CALL} wird die Rücksprungadresse auf den Stack
       gelegt. Das ergibt einen zusätzlichen Eintrag auf den Stack.
 \item Jeder Stack Eintrag ist 4 Byte groß, unabhängig von dessen Inhalt.
\end{enumerate}
Daraus ergibt sich beim Eintritt in die Funktion das folgende Stack-Layout:
\begin{center}
 \begin{tabular}{ll}
  Adresse     & Inhalt \\\hline
  $\$SP + 0$  & Rücksprungadresse \\
  $\$SP + 4$  & Erstes Argument   \\
  $\$SP + 8$  & Zweites Argument  \\
  \ldots      & \ldots            \\
  $\$SP + 4n$ & $n$-tes Argument
 \end{tabular}
\end{center}
Um an die zwei Argumente der \texttt{max} Funktion zu gelangen, kann man also
folgendes schreiben:
\begin{lstlisting}
ADDI R17 SP 4
LW   R17 R17

ADDI R18 SP 8
LW   R18 R18
\end{lstlisting}
Durch diesen Code stehen also die Argumente (aus \texttt{R1}/\texttt{R2}) nach
Aufruf der Funktion \texttt{max} in \texttt{R17}/\texttt{R18}.

\paragraph{Rückgabewerte}
Es gibt viele Möglichkeiten, einen Wert aus einer Funktion zurückzugeben.
Einige davon wären:
\begin{itemize}
 \item Rückgabewerte in einem vorgeschriebenen oder vereinbarten Register speichern.
 \item Rückgabewerte auf dem Stack speichern, wobei der aufrufende Code vor dem Aufruf
       dafür sorgt, dass entsprechende Stack-Einträge reserviert werden
       (\texttt{SP} dekrementieren).
 \item Rückgabewerte auf dem Stack speichern, wobei die Funktion selber den Stack
       manipuliert.
 \item Rückgabewerte an vorgeschriebener oder vereinbarter Heap-Adresse speichern.
 \item etc.
\end{itemize}

Eine der gängigsten und effizientesten Methoden ist allerdings, den Wert in
einem bestimmten Register abzulegen. Auch wir nutzen diese Methode und verwenden
dafür das Register \texttt{R32}. Wird ein Wert zurückzugeben, kopiert der
\texttt{RET} Befehl diesen in das Register \texttt{R32}.

%%%%%%%%%%%%%%%%%%%%%%%%%%%%%%%%%%%%%%%%%%%%%%%%%%%%%%%%%%%%%%%%%%%%%%%%%%%%%%%%
%\paragraph{Überschreiben der Register}
%Es kann sehr schnell passieren, dass eine Funktion Register überschreibt, die
%von einer anderen Funktion verwendet werden. Verwendet z.B. eine Funktion die
%Register \texttt{R1} bis \texttt{R8} für ihre Berechnungen, und ruft sie
%irgendwann eine zweite Funktion auf, die die selben Register verwendet, so
%werden in der aufrufenden Funktion die Register verändert, ohne dass diese davon
%überhaupt etwas ``merkt''. Ein Beispiel dazu:
%\begin{lstlisting}
%main:
%   SET R1 5
%   SET R2 6
%   INC R1
%   DEC R2
%   CALL function
%   RET
%
%function:
%   CP R3 R1
%   CP R1 R2
%   CP R2 R3
%   RET
%\end{lstlisting}
%
%Hier verändert die Funktion \texttt{function} die Werte der Register \texttt{R1}
%und \texttt{R2}, die auch von der Funktion \texttt{main} verwendet werden.
%Diese Art von unbeabsichtigter Überschreibung der Register führt sehr schnell zu
%schwer zu findenden Fehler. Daher verwenden wir eine Programmiertechnik, die
%dieses Problem vermeiden sollte: wir überlegen uns vor (oder während) der
%Implementierung, welche Register wollen wir benutzen und gleich am Anfang der
%Funktion, pushen wir diese Register auf den Stack. Vor dem \texttt{RET} popen
%wir die Register wieder zurück (in umgekehrter Reihenfolge). Im nächsten
%Beispiel kann man diese Technik sehen.
%%%%%%%%%%%%%%%%%%%%%%%%%%%%%%%%%%%%%%%%%%%%%%%%%%%%%%%%%%%%%%%%%%%%%%%%%%%%%%%%

\paragraph{Beispiel}
Folgend ein Beispiel für den Aufruf und die Implementierung einer Funktion. Es
wird eine einfache \texttt{max} Funktion implementiert, welche die größere aus zwei
Zahlen ermittelt. Die Rückgabe erfolgt im Register \texttt{R32}.

Diese Funktion arbeitet mit den Registern \texttt{R17} und \texttt{R18}, deshalb
werden diese zuerst auf den Stack kopiert. Das ist für dieses Beispiel nicht
zwingend notwendig, da der restliche Code diese Register nicht verwendet, aber
die Technik soll vorgestellt werden. Danach erfolgt eine Berechnung der
Stack-Adressen beider Argumente. Dazu ist es hilfreich, sich das Stack-Layout
nach dem Kopieren der zwei Arbeitsregister zu veranschaulichen:

\begin{center}
 \begin{tabular}{ll}
  Adresse     & Inhalt                \\\hline
  $\$SP + 16$ & Zweites Argument (66) \\
  $\$SP + 12$ & Erstes Argument (55)  \\
  $\$SP + 8$  & Rücksprungadresse     \\
  $\$SP + 4$  & R17                   \\
  $\$SP + 0$  & R18                   \\
 \end{tabular}
\end{center}


\lstinputlisting{progs/example_args.uasm}


\subsubsection{Ein größeres Beispiel}
\label{subsubsec:Funktionen-Beispiel}

Das folgende Beispiel verdeutlicht die Verwendung von Funktionen anhand eines
Programms, das die Länge eines Strings berechnet und die Länge ausgibt. Der
String ist im Programm selbst eingebettet (Datensegment). Das Programm besteht
aus mehreren Dateien, die jeweils eine Funktion implementieren:
\begin{enumerate}
 \item prog2.uasm, enthält das Hauptprogramm (Programm
       \ref{lst:VerwendungFunktionen}).
 \item strlen.uasm, enthält die Funktion \texttt{strlen}, welche die Länge eines
       Strings berechnet (Programm \ref{lst:strlen}).
 \item printint.uasm, enthält die Funktion \texttt{printint}, welche eine
       beliebige ganze Zahl ausgibt (Programm \ref{lst:printint}).
 \item putchar.uasm, enthält die Funktion \texttt{putchar}, welche einen Buchstaben
       ausgibt (Programm \ref{lst:putchar}).
\end{enumerate}


\lstinputlisting[caption={Verwendung der Funktionen},
                 label={lst:VerwendungFunktionen}]
                {progs/prog2.uasm}


\lstinputlisting[caption={Funktion strlen}, label={lst:strlen}]
                {progs/strlen.uasm}


\lstinputlisting[caption={Funktion printint}, label={lst:printint}]
                {progs/printint.uasm}


\lstinputlisting[caption={Funktion putchar}, label={lst:putchar}]
                {progs/putchar.uasm}


Dieses Programm wird wie folgt assembliert:
\begin{lstlisting}
 ./uasm prog2.uasm strlen.uasm printint.uasm putchar.uasm
\end{lstlisting}
Bemerkung: Es werden alle benötigten Dateien angegeben. Die Reihenfolge der
Dateien, die eine Funktion definieren ist beliebig. Lediglich die
\glqq main\grqq-Datei, welche den Startpunkt des Programms beinhaltet, muss an 
der ersten Stelle stehen.

Nach dem Assemblieren wird eine Datei namens \texttt{u.out} erzeugt, die den
\glqq Bytecode\grqq\ für die virtuelle Maschine beinhaltet. Man könnte auch die
Option \texttt{-o} verwenden, um einen alternativen Dateinamen zu spezifizieren.
Um diese Datei auszuführen, gibt man folgenden Befehl ein:
\begin{lstlisting}
 ./umach u.out
\end{lstlisting}
Als Ergebnis erscheint die Fehlermeldung
\begin{lstlisting}
ERROR: Cannot load 268 bytes of program into 512 bytes of
       memory
ERROR: Cannot load program file u.out.
Aborted
\end{lstlisting}
Das bedeutet, die virtuelle Maschine hat nicht genug Speicher um dieses
Programm überhaupt laden zu können. 256 Bytes werden für die Interrupttabelle
reserviert und die Maschine verwendet standardmäßig 512 Bytes für den
Speicher. Man kann in diesem Fall mit der Option \texttt{-m} den Speicher
größer spezifizieren. Als Beispiel Nutzen wir \texttt{-m 600}:
\begin{lstlisting}
 ./umach -m 600 u.out
\end{lstlisting}
Das erhöht den Speicher auf 600 Bytes. Es wird dann \glqq 12\grqq\ ausgegeben.

Bemerkung: Es ist zu empfehlen, beim Start der Maschine die Option \texttt{-v}
anzugeben. Zu viele \texttt{PUSH}-Befehle können bei unzureichenden Speicher
einen Stack Overflow verursachen, was derzeit die Maschine zum Stillstand bringt
-- es sei denn, man programmiert einen Interrupt Handler für den Stack Overflow
Interrupt. Der Stack Overflow wird als Warnung ausgegeben, welche nur mit der
Option \texttt{-v} erscheinen.
