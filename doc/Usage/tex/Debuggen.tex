\section{Debuggen}
\label{sec:debuggen}

Eingebaut in der UMach Maschine ist ein einfacher Debugger, der mit der Option
\glqq{}-d\grqq{} gestartet werden kann. 

\subsection{Debug-Befehle}

Der Debugger unterstützt die folgenden Befehle:

\begin{enumerate}
 \item step, oder s oder leere Eingabe. 
       Die aktuelle Instruktion ausführen und die nächste laden.
 \item run. Das Programm bis zum Ende ausführen. Nach dem Programmende wartet
       der Debugger auf weitere Befehle.
 \item quit, oder q. Debugger beenden.
 \item help, h oder ?. Hilfe anzeigen.
 \item show reg <Registerliste>. Zeigt den Inhalt der spezifizierten Register.
 \item show mem <adresse>. Zeigt Speicherinhalt an der angegebenen Addresse.
 \item show mem <start> <wieviel>. Zeigt <wieviel> Bytes Speicherinhalt ab
       der Adresse <start>.
 \item show ins. Zeigt die aktuelle Instruktion.
\end{enumerate}

