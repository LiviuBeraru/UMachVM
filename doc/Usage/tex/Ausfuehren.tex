\section{Ausführen}

Das \texttt{umach} Programm kann wie folgt ausgeführt werden:
\begin{lstlisting}
  ./umach [optionen] programdatei.umx
\end{lstlisting}

Dabei wir die Datei \texttt{programdatei.umx} ausgeführt. Die Optionen sind
unten erläutert.


\subsection{Optionen}
Bei der Ausführung eines UMach-Programm ist es möglich, Optionen anzugeben, die
die Ausführung in bestimmter Weise beeinflussen. Alle Optionen sind optional und
werden mit Startwerten gesetzt. Es stehen die folgenden Optionen zur Verfügung.
\begin{description}
\item[-m]
Mit \glqq{}-m $z$\grqq{}, wobei \glqq{}z\grqq{} eine ganze Zahl ist, kann der
Arbeitsspeicher der virtuellen Maschine auf $z$ Bytes festgelegt werden.
\grqq{}./umach myProg -m 4096\grqq{} startet das Programm \glqq{}myProg\grqq{}
mit einem Arbeitsspeicher von $4096$ Bytes. Ist die \glqq{}-m\grqq{} Option bei
der Ausführung nicht angegeben, so wird das Programm mit $2048$ Bytes
Arbeitsspeicher gestartet.
\item[-d]
Diese Option veranlasst die Maschine im \glqq{}debug\grqq{}-Modus zu starten.
Hierbeit wird der eingebaute Debugger gestartet, nicht ein externer Debugger.
Siehe den Abschnitt \ref{sec:debuggen} auf der Seite \pageref{sec:debuggen} für
mehr Informationen dazu. Vor der Ausführung eines Befehls hält die Maschine an
und der Benutzer ist aufgefordert Kommandos einzugeben. Mit den Kommandos kann
der Benutzer den Speicher und den Inhalt der Register betrachten, jedoch nicht
modifizieren. Mit dem \glqq{}step\grqq{}-Kommando gibt er die Maschine zum
Abarbeiten des aktuellen Befehls frei.
\item[-s]
Disassembliert das angegebene Programm und zeigt das Ergebnis in der Konsole an.
\item[-x]
Die Darstellung von Zahlenwerten in der Ausgabe des \glqq{}debug\grqq{}-Modus,
oder beim Disassemblieren, wird nach Angabe dieser Option hexadezimal sein.
\item[-v]
Setzt den \glqq{}verbosity level\grqq{} auf \glqq{}warnings\grqq{}. Unter
zweifacher Anwendung der \glqq{}-v\grqq{}-Option, entweder \glqq{}-v -v\grqq{},
oder \glqq{}-vv\grqq{}, wird der \glqq{}verbosity level\grqq{} auf
\glqq{}info\grqq{} gestellt. Ist die Option nicht angegeben, so ist der
\glqq{}verbosity level\grqq{} standardmäßig auf ''error'' gesetzt.


\end{description}

