\section{Ausführen}

Das Programm \texttt{umach} ist die Implementierung der virtuellen Maschine. Es
kann sein, dass man es zuerst kompilieren muss. Für die Kompilierung wird im
Quellenverzeichnis ein Makefile zur Verfügung gestellt. Wechselt man in
das Quellenverzeichnis, so kann man das Programm mit dem folgenden Befehl
kompilieren:
\begin{lstlisting}
make
\end{lstlisting}
Das Programm wurde auf Linux getestet.

Das \texttt{umach} Programm kann wie folgt ausgeführt werden:
\begin{lstlisting}
  ./umach [optionen] programdatei.umx
\end{lstlisting}

Dabei wir die Datei \texttt{programdatei.umx} ausgeführt. Die Optionen sind
unten erläutert.

\subsection{Optionen}
Bei der Ausführung eines UMach-Programms ist es möglich, Optionen anzugeben, die
die Ausführung in bestimmter Weise beeinflussen. Alle Optionen sind optional und
werden mit Standardwerten initialisiert. Es stehen die folgenden Optionen zur Verfügung:
\begin{description}
\item[-m]
Mit \glqq{}-m $z$\grqq{}, wobei \glqq{}z\grqq{} eine positive ganze Zahl ist,
kann der Arbeitsspeicher der virtuellen Maschine auf $z$ Bytes festgelegt
werden. \glqq{}./umach myProg -m 4096\grqq{} startet das Programm
\glqq{}myProg\grqq{} mit einem Arbeitsspeicher von $4096$ Bytes. Ist die
\glqq{}-m\grqq{} Option bei der Ausführung nicht angegeben, so wird das Programm
mit $2048$ Bytes Arbeitsspeicher gestartet.

\item[-d]
Diese Option veranlasst die Maschine im \glqq{}debug\grqq{}-Modus zu starten.
Hierbei wird der eingebaute Debugger gestartet, nicht ein externer.
Siehe den Abschnitt \ref{sec:debuggen} ab der Seite \pageref{sec:debuggen}.

\item[-s]
Disassembliert das angegebene Programm und zeigt das Ergebnis in der Konsole an.
Unter Verwendung der Option -x werden alle nummerischen Werte im
Hexadezimalsystem angezeigt.

\item[-x]
Die Darstellung von Zahlenwerten in der Ausgabe des \glqq{}debug\grqq{}-Modus,
oder beim Disassemblieren, wird nach Angabe dieser Option hexadezimal sein.

\item[-v]
Setzt den \glqq{}verbosity level\grqq{} auf \glqq{}warnings\grqq{} (Warnungen
und Fehler werden ausgegeben). Unter zweifacher Anwendung der
\glqq{}-v\grqq{}-Option, entweder \glqq{}-v -v\grqq{}, oder \glqq{}-vv\grqq{},
wird der \glqq{}verbosity level\grqq{} auf \glqq{}info\grqq{} gestellt
(allgemeine Informationen, Warnungen und Fehler werden ausgegeben). Ist die
Option nicht angegeben, so ist der \glqq{}verbosity level\grqq{} standardmäßig
auf \glqq{}error\grqq{} gesetzt (nur Fehler werden ausgegeben).


\end{description}

