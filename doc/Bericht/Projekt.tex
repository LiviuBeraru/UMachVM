\section{Projektvorstellung}

Das IT-Projekt wurde während des SS 2012 und WS 2012/13 durchgeführt. Ziel
dieses Projektes war die Spezifikation und Implementieung einer virtuellen
Maschine namens UMach. Dazu gehören ein Assembler und ein Debugger für diese
Maschine. Der gewünschte Workflow bei der Fertigstellung des Projektes war der
folgende:

\begin{enumerate}
\item Der Benutzer schreibt eine Assembler-Datei. Diese enthält Anweisungen an
die UMach-Maschine wie die folgenden:
\begin{lstlisting}
SET  R2 0x1A
MULI R2 4
CP   R2 LO
SET  R1 0
SW   R1 R2
\end{lstlisting}
Wir nehmen an, die Datei heißt \glqq{}test.uasm\grqq{}.

\item Diese Datei wird in eine Bytecode-Datei assembliert mit dem folgenden
Befehl:
\begin{lstlisting}
./uasm -o test.umx test.uasm
\end{lstlisting}
Ergebnis ist eine Bytecode-Datei namens \glqq{}test.umx\grqq{}. Diese
Bytecode-Datei enthält UMach-Instruktionen in binärer Form. Die oben angegebenen
Instruktionen werden wie folgt assembliert:
\begin{lstlisting}
0x10 0x02 0x00 0x1A
0x3A 0x02 0x00 0x04
0x11 0x02 0x2B 0x00
0x10 0x01 0x00 0x00
0x15 0x01 0x02 0x00
\end{lstlisting}

\item Die Bytecode-Datei wird von der UMach-Maschine ausgeführt mit dem
folgenden Befehl:
\begin{lstlisting}
./umach test.umx
\end{lstlisting}
Alternativ kann man die Maschine in Debugg-Modus starten mit dem Befehl
\begin{lstlisting}
./umach -d test.umx
\end{lstlisting}
Alternativ steht ein getrennter Qt-Debugger zur Verfügung.
\end{enumerate}

Diesen Workflow betrachten wir als implementiert. Alle diese Schritte sind
möglich und liefern die gewünschten Ergebnisse. Das Projekt enthält auch eine
Reihe von fertiggeschriebenen Assembler-Programmen, die gleich getestet werden
können.


\subsection{Teilaufgaben}

Im Rahmen dieses Projektes wurden die folgenden Teilaufgaben übernommen und
gelöst.

\begin{enumerate}
\item Konzeptioneller Entwurf der virtuellen Maschine.
\item Spezifikation der virtuellen Maschine als PDF Dokument (Liviu Beraru).
\item Implementieung der Maschine in C99 für Linux (Liviu Beraru).
\item Assembler für die Maschine in C99 für Linux (Werner Linne).
\item Qt-Debugger (Simon Beer).
\item Demonstrationsprogramme in der UMach Assemblersprache \texttt{uasm} 
      (Willi Fink).
\item Anweisungen zur Verwendung der UMach virtuellen Maschine, als PDF
Dokument.
\end{enumerate}

Die späteren Abschnitte gehen auf die Lösungswege der einzelnen Teilaufgabe
ein. Jeder Abschnitt wurde vom zuständigen Gruppenmiglied geschrieben.


\subsection{Organisation}

Die Kommunikation unter den Gruppenmiglieder lief über Email, persönlichen
Gespräche, Telefonaten und TODO-Dateien. Für die Verwaltung aller \LaTeX{} und C
Dateien wurde das Versionsverwaltungssystem GIT verwendet. Für die zentralle
Verwaltung aller Dateien wurde ein GIT-Account auf Github eingerichtet und dort
ein Projekt angelegt. Das Projekt ist öffentlich zugänglich, Schreiberechte aber
haben nur die Gruppenmiglieder. Das Projekt kann unter der Adresse

\url{https://github.com/Malkavian/UMachVM}

nachgeschlagen werden. Dort können alle Dateien heruntergeladen werden.
