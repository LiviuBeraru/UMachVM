\section{Demo-Programme}
\begin{flushright}
Willi Fink
\end{flushright}

Um die Funktionalität der Virtuellen Maschine testen und demonstrieren zu
können, mussten im Rahmen des Projekts Demo-Programme erstellt werden. 
\subsection{Meist verwendete Befehle}
Die dabei am häufigsten verwendeten Befehle werden nun nähergebracht.
\subsubsection{Wertzuweisung}
 \texttt{SET R1 5} oder \texttt{SET R1 label}
 
 Setzt das Register R1 auf den angegebenen Wert. Labels werden durch Adressen
ersetzt.
 

\subsubsection{Arithmetische Befehle}
\begin{center}
\begin{tabular}{@{\ttfamily}ll}
ADD R1 R2 R3 & Addiert $R1 \gets R2 + R3$ \\
SUB R1 R2 R3 & Subtrahiert $R1 \gets R2 - R3$\\
INC R1 & Inkrementiert $R1++$\\
DEC R1 & Dekrementiert $R1--$\\
MUL R1 R2 & Multiplikation \\
DIV R1 R2 & Division durch 0 führt zu Interrupt
\end{tabular}
\end{center}
\subsubsection{Bedingte Sprünge}
\begin{itemize}
 \item{\texttt{\texttt{CMP R1 R2}}}\\
 Vergleicht das Register R1 mit R2.
 \item{\texttt{\texttt{BL label}}}\\
 Springt zum angegebenen Label, falls R1 kleiner R2 ist.
 Weitere Möglichkeiten: \texttt{BLE, BG, BGE, BE}
\end{itemize}

\subsubsection{Unbedingte Sprünge}
 \begin{itemize}
 \item \texttt{JMP label}\\
 Sprung zu einem Label.
 \item \texttt{CALL funktion} \\
 Funktionsaufruf ähnelt dem JMP Befehl, mit dem Unterschied, dass nach der
Ausführung der Funktion ins Hauptprogramm zurückgesprungen und der nächste
Befehl ausgeführt wird.
 \end{itemize}
\subsubsection{IO-Befehle}
 \begin{itemize}
 \item \texttt{IN R1 R2 ZERO}\\
 Liest \texttt{R2} viele Bytes vom Port \texttt{ZERO}(Konsole) und schreibt sie
an die Adresse \texttt{R1}.
 \item \texttt{OUT R1 R2 ZERO}\\
 Schreibt \texttt{R2} viele Bytes aus dem Speicher, beginnend mit dem Byte an
der Adresse \texttt{R1}, an den Port \texttt{ZERO} raus.
 \end{itemize}





\subsection{Hilfsfunktionen}
Durch wiederkehrende Muster innerhalb der Demoprogramme, wurde schon früh die 
Notwendigkeit von Hilfsfunktionen deutlich. Folgende Hilfsfunktionen verrichten
Aufgaben, die häufig bewältigt werden müssen:
 \begin{itemize}
 \item inputint
 \item printint
 \item putchar
 \item newline
 \end{itemize}
 Die Funktionen arbeiten nach dem Prinzip, dass der Callee die über den Stack 
 überreichten Argumente hinter sich aufräumt.

\subsubsection{inputint}
Diese Hilfsfunktion liest 10 Zeichen aus der Eingabe der Konsole, wandelt diese
in ein Integer um und pusht anschließend das Ergebnis auf den Stack. Dieses
Ergebnis kann dann im Hauptprogramm mit einem PULL-Befehl in ein beliebiges
Register gespeichert werden. Es werden nur 10 Zeichen gelesen, weil die größte
Zahl, die ein Register darstellen kann, zehnstellig ist.

\subsubsection{printint}
Die über den Stack übergebene Zahl wird durch fortlaufende Divisionen und das
Aufaddieren des ASCII-Wertes der Zahl 0 nach und nach in einen String
umgewandelt. Jedes Zeichen wird auf den Stack gepusht. Sobald die Konvertierung
fertig ist wird die ''putchar''-Funktion entsprechend oft ausgeführt, um die
Zeichen an die Konsole herauszuschreiben.

\subsubsection{putchar}
Das über den Stack übergebene Zeichen im ASCII-Format wird auf die Konsole 
herausgeschrieben.

\subsubsection{newline}
Diese Funktion schreibt das ASCII-Zeichen, das einen Zeilenumbruch symbolisiert,
auf die Konsole heraus.


\subsection{Die Demo-Programme}

\subsubsection{helloWorld.uasm}
Schreibt ein ''Hello!'' an die Konsole raus.

\subsubsection{echo.uasm}
Ein Testprogramm das die folgende Zeile ''Echo program. End with q'' an die 
Konsole schreibt und durch die Eingabe des Zeichens 'q' beendet wirden kann.

\subsubsection{99\_bottles.uasm}
Dieses Programm schreibt den Songtext des Lieds ''99 Bottles of Beer'' an die 
Konsole heraus. ''99 Bottles of Beer'' wurde in über 1500 verschiedenen
Programmiersprachen als Ausgabe umgesetzt. Dieser Songtext und weitere
Informationen sind auf der Webseite \url{www.99-bottles-of-beer.net} zu
finden.

\subsubsection{ggT.uasm}
Der Benutzer gibt zwei Zahlen an, von diesen wird anschließend der größte 
gemeinsame Teiler mit Hilfe des euklidischen Algorithmus berechnet und
ausgegeben.
\subsubsection{fibonacci.uasm}
Dieses Programm schreibt $n$ Zahlen der Fibonacci-Folge, 
$X_{n} = X_{n-1} + X_{n-2}$ mit $X_{1} = 1$ und $X_{2} = 2$, an die Konsole
heraus. Die Zahl $n$ wird vom Benutzer durch die Hilfsfunktion ''inputint''
angegeben. Es ist eine Unterscheidung zwischen $n = 1$, $n = 2$ und $n \geq 3$
notwendig, da erst ab $n \geq 3$ alle Vorgänger für die Folge feststehen, zuvor
muss man die Vorgänger aus der Definition entnehmen. Für den Fall $n \geq 3$
kommt eine Schleife zum Einsatz, die die Folge entsprechend der eingegebenen
Zahl $n$ berechnet. Am Ende jedes Durchlaufs des Schleifenrumpfs wird die
berechnete Fibonacci Zahl ausgegeben.


\subsubsection{zahl\_raten.uasm}
Hierbei handelt es sich um ein Spiel, bei dem es geht eine ''zufällige'' Zahl
zu erraten. Die Vorgehensweise des Programms ist wie folgt:
 \begin{enumerate}
 \item Größtmögliche Zahl abfragen\\
 Der Benutzer wird zunächst gebeten die größtmögliche Zahl $m$ einzugeben.
 \item Seed erzeugen\\
 Der Benutzer wird gebeten einen Seed einzugeben. Der Seed wird in Form eines 
 Strings angegeben. Der String wird mit der ''inputint''-Funktion gelesen und in
eine Zahl $X_0$ umgewandelt.
 \item Pseudozufallszahl durch die Formel 
 ''$X_{n+1} = (a + b\cdot X_n) \mod m$'' generieren\\
 Diese Formel wird n mal aufgerufen. Hierbei sind $n$, $a$ und $b$ im Code 
 festgelegt. Um alle Zahlen erreichen zu können, sind $a$ und $b$ als Primzahlen
gewählt.
 \item Spieler rät die Zahl\\
 Der Spieler rät durch eingeben die Zahl. Dabei bekommt er jedes mal die 
 Rückmeldung, ob die geratene Zahl größer, kleiner oder gleich der gesuchten
ist. Die Anzahl der Versuche wird mitgezählt und bei jeder Rückmeldung mit
ausgegeben. Hat der Spieler die gesuchte Zahl erraten, so ist das Spiel vorbei.
 \end{enumerate}

\subsubsection{tictactoe.uasm}
Tic Tac Toe ist ein Spiel für zwei Personen. Hierbei geht es darum, abwechselnd 
eins der neun im Quadrat angelegten Felder mit seiner Markierung zu versehen. Es
gewinnt der Spieler, der drei Markierungen in einer Reihe, senkrecht, waagrecht,
oder diagonal, angeordnet hat.

Für die Realisierung dieses Spiels werden die Spielinformationen in den
Registern 
gespeichert:
 \begin{itemize}
 \item R1-R9: Spielfelder\\
 Für den Wertebereich eins Spielfeldregisters gilt: 0(von niemand belegt), 
 1(von Spieler 1 belegt) und 2(von Spieler 2 belegt)
 \item R10: Aktueller Spieler
 Für den Wertebereich eins Spielfeldregisters gilt: 1(Spieler 1 ist an der 
 Reihe), 2(Spieler 2 ist an der Reihe)
 \item R20: Anzahl der Spielzüge\\
 Die Anzahl der Spielzüge wird festgehalten, um während der Auswertung ein 
 Unentschieden festzustellen zu können.
 \end{itemize}
Der Spielzyklus besteht aus drei wesentlichen Abschnitten:
 \begin{enumerate}
 \item Eingabe\\
 Der Spieler, der an der Reihe ist wird gebeten ein Feld anzugeben, in das er 
 seine Markierung setzen will. Dafür gibt er eine Zahl zwischen 1 und 9 an. Nach
der Eingabe wird der Inhalt des Registers R10 in ein Register der
Spielfelder(R1-R9) kopiert. Nach der Eingabe endet der Zug des Spielers.
 \item Ausgabe
 \begin{lstlisting}[xleftmargin=6cm]
  1|2|3
  -+-+-
  4|5|6
  -+-+-
  7|8|9
 \end{lstlisting}
So sieht die Ausgabe der Spielfelder zu Beginn des Spiels aus. Hierbei wird
für jedes Spielfeldregister abgefragt ob es den Wert 0 hat, oder nicht. Ist in
einem Feld die 0, so wird eine Zahl herausgeschrieben um zu symbolisieren, dass
dieses Feld frei ist und während der Eingabe mit der entsprechenden Zahl
ausgewählt werden kann, anderenfalls wird der Inhalt auf den Stack gepusht und
es wird eine Funktion aufgerufen. Diese Funktion schreibt auf die Konsole ein
'X' heraus falls das gepushte eine '1' ist. Für eine '2' schreibt es ein 'O'.
 \item Auswertung\\
 Es gibt 8 Reihen die komplett mit der Markierung eines Spielers belegt werden 
 können, damit ein Spieler gewinnt. Dem entsprechend gibt es 8 Abfragen die
erfolgen müssen um zu prüfen ob jemand gewonnen hat. Für eine Abfrage werden die
drei Register aus einer Reihe mit einander multipliziert. Es gibt nur drei
relevante Äquivalenzklassen der Ergebnisse, die dabei berücksichtigt werden
müssen. '0','2','4': Entweder eines der Felder ist nicht belegt, oder in einer
Reihe treten sowohl '1'-en als auch '2'-en auf. '1': alle Register der Reihe
enthalten eine '1', somit gewinnt Spieler 1. '8': alle Register der Reihe
enthalten eine '2', somit gewinnt der Spieler 2. Steht ein Gewinner fest, so
tritt sofort das Ende des Spiels ein. Falls kein Gewinner feststeht wird der
Rundenzähler, das Register R20, um eins erhöht. Beinhaltet er die Zahl '9', so
sind '9' Spielzüge vergangen, ohne dass ein Spieler gewonnen hat. Folglich sind
alle Felder belegt und es tritt ein Unentschieden ein und somit das Ende des
Spiels. Hat 
niemand gewonnen und es steht auch kein Unentschieden fest, so wird das 
Spielerregister entweder von '1' auf '2' oder von '2' auf '1' gesetzt und der
Spielzyklus beginnt von vorn.
 \end{enumerate}

Am Ende des Spiels:
\begin{enumerate}
 \item Ausgabe des Ergebnisses\\
 Es wird eine Nachricht ausgegeben, die entweder besagt welcher Spieler
gewonnen 
 hat, oder ob ein Unentschieden feststeht. Es folgt die Frage, ob eine neue
Runde gespielt werden Soll. Ist die Eingabe ein 'q' so beendet sich das
Programm, anderenfalls startet das Spiel neu.
 \item bei neuer Runde aufräumen\\
 Vor dem Neustarten müssen alle Spielfeldregister geleert werden und das 
 Spielerregister wird wieder auf '1' gesetzt.
\end{enumerate}











