\section{Demo-Programme}
\begin{flushright}
Willi Fink
\end{flushright}

Um die Funktionalität der Virtuellen Maschine testen und demonstrieren zu
können, wurden im Rahmen des Projekts Demo-Programme erstellt. 
\subsection{Meist verwendete Befehle}
\subsubsection{Wertzuweisung}
 \texttt{SET R1 5} oder \texttt{SET R1 label}
 
 Setzt das Register R1 auf den angegebenen Wert. Labels werden durch Adressen
ersetzt.
 

\subsubsection{Arithmetische Befehle}
\begin{center}
\begin{tabular}{@{\ttfamily}ll}
ADD R1 R2 R3 & Addiert $R1 \gets R2 + R3$ \\
SUB R1 R2 R3 & Subtrahiert $R1 \gets R2 - R3$\\
INC R1 & Inkrementiert $R1++$\\
DEC R1 & Dekrementiert $R1--$\\
MUL R1 R2 & Multiplikation \\
DIV R1 R2 & Division durch 0 führt zu Interrupt
\end{tabular}
\end{center}
\subsubsection{Bedingte Sprünge}
\begin{itemize}
 \item{\texttt{\texttt{CMP R1 R2}}}
 Vergleicht das Register R1 mit R2.
 \item{\texttt{\texttt{BL label}}}
 Springt zum angegebenen Label, falls R1 kleiner R2 ist.
 Weitere Möglichkeiten: \texttt{BLE, BG, BGE, BE}
\end{itemize}

\subsubsection{Unbedingte Sprünge}
 \begin{itemize}
 \item \texttt{JMP label}
 Sprung zu einem Label.
 \item \texttt{CALL funktion} 
 Funktionsaufruf ähnelt dem JMP Befehl, mit dem Unterschied, dass nach der
Ausführung der Funktion ins Hauptprogramm zurückgesprungen und der nächste
Befehl ausgeführt wird.
 \end{itemize}
\subsubsection{IO-Befehle}
 \begin{itemize}
 \item \texttt{IN R1 R2 ZERO}
 Liest \texttt{R2} viele Bytes vom Port \texttt{ZERO}(Konsole) und schreibt sie
an die Adresse \texttt{R1}.
 \item \texttt{OUT R1 R2 ZERO}
 Schreibt \texttt{R2} viele Bytes aus dem Speicher, beginnend mit dem Byte an
der Adresse \texttt{R1}, an den Port \texttt{ZERO} raus.
 \end{itemize}





\subsection{Hilfsfunktionen}
 \begin{itemize}
 \item inputint
 \item printint
 \item putchar
 \item newline
 \end{itemize}



\subsection{Fibonacci Zahlen}

 Folge $X_{n} = X_{n-1} + X_{n-2}$
 mit $X_{1} = 1$, und $X_{2} = 2$
 \begin{itemize}
 \item Unterscheidung zwischen $n = 1$, $n = 2$ und $n \geq 3$ notwendig
 \end{itemize}


\subsection{Zahl raten}
 \begin{itemize}
 \item Seed erzeugen
 \item Pseudozufallszahl generieren
 \item Formel: $X_{n+1} = (a + b\cdot X_n) \mod m$
 \item Spieler rät die Zahl
 \item Rückmeldung ob die geratene Zahl größer oder kleiner der gesuchten ist
 \item Anzahl der Versuche
 \end{itemize}

\subsection{Tic Tac Toe}
 Belegung der Register:
 \begin{itemize}
 \item R1-R9: Spielfelder
 \item R10: Aktueller Spieler
 \item R20: Anzahl der Spielzüge
 \end{itemize}
 Spielzyklus:
 \begin{enumerate}
 \item Eingabe
 \item Ausgabe
 \item Auswertung
 \end{enumerate}

Am Ende des Spiels:
\begin{itemize}
 \item neue Runde oder Beenden
 \item bei neuer Runde aufräumen
\end{itemize}











