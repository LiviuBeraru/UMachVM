\section{Assembler}
\begin{flushright}
Werner Linne
\end{flushright}

\subsection{Zielsetzung}

\paragraph{Aufgabe des Assemblers}

Der \emph{uasm} Assembler übersetzt Quelltext in UMach Bytecode und erstellt
Debuginformationen.

\paragraph{Erwünschte Eigenschaften}

Die aus meiner Sicht wichtigsten Eigenschaften des Assemblers sind:
\begin{itemize}
\item \textbf{Effektive Syntax:} Die Syntax des UMach Assemblers soll verständlich
und komfortabel sein. Dadurch wird das Codieren eines UMach Programms erleichtert.

\item \textbf{Performance:} Der Assembler soll auch größere Mengen an Quelltext
mit wenig CPU-Zeit und wenig Arbeitsspeicherbedarf übersetzten können.

\item \textbf{Aussagekräftige Fehlermeldungen:} Bei syntaktischen Fehlern im
Quelltext soll der Benutzer möglichts genau über Art und Position des gefundenen
Fehlers informiert werden.

\item \textbf{Nützliche Debuginformationen:} Der Assembler generiert bei der
Übersetzung des Quelltexts (optional) Debuginformationen. Art und Format dieser
Informationen wurden maßgeschneidert auf die Bedürfnisse des UMach Debuggers
festgelegt.
\end{itemize}

\subsection{Bedienung und Syntax}

\paragraph{Bedienung}

Der Assembler wird über die Befehlszeile bedient. Die Aufrufsyntax ist folgendermaßen
definiert:
\begin{quote}\texttt{uasm [-o outfile] [-g] [-w] file(s)}\end{quote}
Die Bedeutung der einzelnen Elemente wird im folgenden genauer erklärt:

Das \emph{-o} Flag ist optional und benötigt, falls es gesetzt wurde, genau ein
Argument (\emph{outfile}). Das Argument \emph{outfile} steht für den Namen der
Datei, in welche der vom Assembler erzeugte UMach Bytecode geschrieben wird. Ist 
das \emph{-o} Flag nicht gesetzt wird die Datei namens \emph{u.out} verwendet.

Das \emph{-g} Flag steuert die Generierung von Debuginformationen. Ist das Flag
gesetzt werden Debuginformationen generiert, ansonsten nicht.

Das \emph{-w} Flag veranlasst den Assembler im Fehlerfall erst dann zu Terminieren,
nachdem der Benutzer die Eingabetaste gedrückt hat. Dieses Flag wird hauptsächlich
von dem Debugger genutzt.

Das Argument \emph{file(s)} steht für ein oder mehrere Dateien und wir zwingend
benötigt. Der Assembler übersetzt alle in \emph{file(s)} genannten Quelltextdateien
in den UMach Bytecode.

\paragraph{Syntax}

Die Syntax von UMach Quelltextdateien ist relativ einfach gestaltet. Wir
unterscheiden zwischen den Abschnitten \emph{Code Section} und \emph{Data Section}.
Jede Quelltextdatei beginnt implizit mit der \emph{Code Section}. Die
\emph{Code Section} beinhaltet Befehle, Sprungmarken, Kommentare und Leerzeilen.
Die \emph{Data Section} ist optional und wird durch die Textzeile ``\emph{.data}''
eröffnet. Die \emph{Data Section} beinhaltet Definitionen von Variablen,
Kommentare sowie Leerzeilen.

Zur Veranschaulichung folgt ein fiktives Beispiel einer UMach Quelltextdatei:

\begin{lstlisting}[numbers=left, numberstyle=\tiny]
SET R1 hello
myloop:
    CALL println #this function is implemented somewhere else
    DEC R2
    CMP R2 ZERO
BNE myloop
#lines containing only a comment or nil are ignored
SET R1 9001
EOP

.data #begin of data definitions
.string hello  "Hello World!"
.int    answer 42
.int    drink  0xCAFE
\end{lstlisting}

In diesem Beispiel bilden die Zeilen 1 bis 10 die \emph{Code Section} und die
\emph{Data Section} wird in Zeile 11 eingeleitet und beinhaltet Zeile 12 bis 14.

Zu beachten sind folgende syntaktischen Regeln:
\begin{itemize}
    \item foo
    \item bar
    \item baz
\end{itemize}
