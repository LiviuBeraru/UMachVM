\section{Qt Debugger}
\begin{flushright}
Simon Beer
\end{flushright}

\subsection{Einleitung}

\paragraph{Zielsetzung}
Ziel der Entwicklung war ursprünglich nur die Bereitstellung eines graphischen Debuggers. Dabei sollen folgende Funktionalitäten mindestens unterstützt werden:

\begin{itemize}
	\item Setzen von Haltepunkten an denen die Maschine die Ausführung unterbrechen soll und auf eine Benutzereingabe wartet.
	\item Einzelschritt nach einem Haltepunkt durch die Instruktionsfolge
	\item Anzeigen der entsprechenden Codestelle bei Auftreten eines Haltepunktes oder bei Einzelschritt
	\item Inspizieren und Modifizieren der Register, Daten und des Maschinenzustandes.
\end{itemize}

Eine weitere Anforderung besteht darin, dass die graphische Oberfläche und der Maschinenkern als eigenständige Prozesse laufen sollen und die Kommunikation und Steuerung somit per Inter-Prozess-Kommunikation realisiert ist. Dies hat insofern den Vorteil, das der Maschinenkern nicht als Teil der GUI realisiert werden muss. Ändert sich z.B. etwas an der internen Maschinenkernimplementierung muss keine Änderung oder ein Neu-Kompilierung der graphischen Oberfläche vorgenommen werden.

Ein sekundäres Ziel besteht darin, den graphischen Debugger plattform-unabhängig zu gestalten, oder zumindest eine unproblematische Umsetzung für andere gängige Betriebssysteme (Windows, MacOS) abseits der verwendeten Entwicklungsplattform Linux zu ermöglichen.

\paragraph{Grundlegende Designentscheidungen}
Im Verlauf der Planung zeigte sich jedoch, dass alle für die Umsetzung der Zielstellung an den graphischen Debugger notwendigen Elemente im Ansatz schon zu einer rudimentären Entwicklungsumgebung geführt haben. Vor allem die Tatsache, dass an Haltepunkten die entsprechende Zeile im Quelltext angezeigt werden sollte. Die dafür in \emph{Qt} verwendeten graphischen Oberflächenelemente erlauben jedoch auch die Edition des angezeigten Text. Dieser Umstand hatte zur Folge, dass der Debugger als eine minimalistische Entwicklungsumgebung  implementiert wurde, welche hauptsächlich Funktionalitäten zum Debuggen der Programme bereitstellt. 

Für die Verwendung von \emph{Qt} sprach die damit zu erreichende Plattform-Unabhängigkeit. Nicht nur in Anbetracht der graphischen Elemente sondern auch im Bereich der Inter-Prozess-Kommunikation. Hier stellt \emph{Qt} eigene Klassen bereit, die auf allen Systemen gleichermaßen benutzt werden können, ohne dass hier abhängig Betriebssystem spezifische Kenntnisse notwendig sind.

\subsection{Debugging}

Es gibt verschiedene Möglichkeiten die Steuerung der Maschine beim Debuggen zu realisieren. Daher soll eine kleine Übersicht über mögliche Verfahren gegeben werden und eine Begründung für die Wahl des in der Realisierung des Debuggers verwendeten Verfahrens. 

\paragraph{Haltepunkte}
Es gibt insgesamt zwei nennenswerte Verfahren um Haltepunkte zu realisieren. Eine Möglichkeit besteht darin die entsprechende Instruktion, an der ein Halt der Maschine stattfinden soll, durch einen speziellen Interrupt zu ersetzen. Tritt dieser ein, wird dem Debugger dies durch die Maschine signalisiert. Diese Möglichkeit ist aber aufgrund verschiedener Aspekte nicht umgesetzt worden. Vor allem hätten dadurch Änderungen in der Maschinenspezifikation vorgenommen worden müssen, welche aber nicht erwünscht waren.

Daher kam letztendlich eine weitere Methode zur Anwendung, dem Abgleich der aktuellen Instruktions-Adresse. Hier wird die aktuelle Adresse mit einer Liste von Adressen verglichen an denen ein Halten der Maschine erwünscht ist. Dies kann entweder durch die Maschine selbst geschehen, oder direkt durch den Debugger. Letzteres wurde letztendlich auch zur Umsetzung realisiert. Dabei wird nach nach jeder Instruktions-Ausführung die aktuelle Adresse vom Debugger mit Hilfe der Inter-Prozess-Kommunikation aus der Maschine ausgelesen und abgeglichen.

\paragraph{Adresstabelle}
Als Haltepunkte können vom Benutzer entweder Zeilennummern oder Sprunglabels angegeben werden. Um diese den entsprechenden Instruktions-Adressen zuordnen zu können wird vom Assembler beim Assemblieren eine entsprechende Datei mit einer Tabelle erzeugt, welche beim Starten des Debugging-Vorganges durch den Debugger eingelesen wird. Da der Abgleich zwischen Zeilennummer und Sprunglabel erst beim Start abgeglichen werden kann, und in der GUI keine zusätzliche Prüfung vorhanden ist, können vom Benutzer ungültige Zeilen oder Labels angegeben werden. Diese werden beim Debugging-Vorgang ignoriert.

\paragraph{Einzelschritt}
Standardmäßig läuft die Maschine, wenn sie durch den Debugger kontrolliert wird, in einer Art Einzelschrittmodus, um dem Debugger die Möglichkeit zu geben an bestimmten definierten Zeitpunkten die Maschine zu manipulieren um z.B. die aktuelle Instruktions-Adresse auslesen zu können.

In der Regel bleibt dies jedoch vom Benutzer unbemerkt, da nur auf direkten Wunsch des Benutzers dieser automatische Zyklus zwischen Debugger und Maschine unterbrochen und auf Eingaben vom Benutzer gewartet wird. Dies ist in der Regel nach Haltepunkten oder einem daruf vom Benutzer explizit angestoßenen Einzelschritt der Fall. Auf die explizite Implementierung diese Zyklus wird im späteren Fauluf im Kapitel Inter-Prozess-Kommunikation im Detail eingegangen.

\subsection{Inter-Prozess-Kommunikation}

Die Steuerung der Maschine und der Datenaustausch zwischen Maschine und Debugger wurde wie schon erwähnt durch Inter-Prozess-Kommunikation realisiert. Im diesem Kapitel soll im Detail darauf eingegangen werden, welche Mittel dazu verwendet worden sind, um dies zu realisieren. Weiter soll auch der Ablauf der Kommunikation im Detail geschildert wird.

\paragraph{QSystemSemaphore}
Bei der \emph{QSystemSemaphore} handelt es sich um eine von der \emph{Qt}-Bibliothek bereitgestellte Semaphore. Dabei handelt es sich um einen Ressourcenzähler, der prozess-übergreifend eingesetzt werden kann. Dieser kann mit einem beliebigen Startwert initialisiert werden. Weiterhin werden zwei Funktionen bereit gestellt mit denen Ressourcen angefordert  oder freigegeben werden können.

Die Besonderheit liegt hier darin, dass wenn nicht mehr die Angeforderte Anzahl von Ressource verfügbar sind, d.h. der Ressourcenzähler einen stand von null oder weniger erreichen würde, der anfordernde Prozess, solange blockiert wird, bis ein weiterer Prozess ausreichend zusätzliche Ressourcen freigibt.  Wird immer nur jeweils eine Ressource freigegeben oder Angefordert, so kann das Semaphore als ein Mutex verwendet werden, um die Abläufe von zwei Prozessen miteinander zu synchronisieren.

Angelegt werden \emph{QSystemSemaphore}  mit einem Konstruktor, der gleichzeitig eine eindeutige ID als String erwartet, sodass diese von verschiedenen Prozessen aus identifiziert und drauf zugegriffen werden kann. Ist eine \emph{QSystemSemaphore} bereits angelegt, so wird durch jeden weiteren Konstruktor-Aufruf mit Angabe der gleichen ID nur eine Referenz auf die bestehende Semaphore zurückgegeben.
 
\paragraph{QSharedMemory}
Beim \emph{QSharedMemory} handelt es sich um einen eigenständigen Speicher der den Zugriff aus verschiedenen Prozessen aus erlaubt, da in der Regel Prozesse nicht auf den anderer zugreifen können. Genau wie die \emph{QSystemSemaphore} wird dieser beim ersten Konstruktor-Aufruf und der vergabe einer eindeutigen ID erzeugt. Daraufhin kann mit einem Aufruf von \emph{create()} der eigentliche Speicher unter Angabe einer Größe reserviert alloziert werden. Jeder weitere Konstruktor-Aufruf unter Angabe der gleichen ID liefert eine Referenz auf den bestehenden \emph{QSharedMemory} zurück.

Um den so erzeugten Speicher in einem Prozess nutzbar machen, muss er mit dem Funktionsaufruf von attach() an den Prozess "angehängt" werden. Wird der Speicher nicht mehr benötigt, so kann er mir einem \emph{detach()} wieder vom Prozess gelöst werden. Hierbei ist jedoch sicherzustellen, dass der Speicher erst komplett von allen Prozessen abgehängt wird, wenn er nicht mehr benötigt wird, da das Abhängen vom letzten Prozess automatisch den Destruktor aufruft und somit somit alle hinterlegten Daten verloren gehen.

Ein weiterer Aspekt der beim Umgang mit \emph{QSharedMemory} zu beachten ist, dass wenn der Speicher nicht mehr benötigt wird, darauf geachtet werden muss, dass er von allen Prozessen abgehängt werden muss, damit die Freigabe sichergestellt wird. Dies hängt damit zusammen, dass unter Linux der Speicher dass Prozessende aller Zugreifenden Prozesse überlebt und als Speicherleiche bestehen bleibt. Besonders im Fehlerfall ist darauf zu achten.

Um den geordneten zugriff zu gewährleisten und gleichzeitige Zugriffsversuche zu unterbinden, stellt \emph{QSharedMemory} auch ein integriertes Mutex zur verfügung um den Speicher dagegen abzusichern. 

\paragraph{Übersicht}
Zur Realisierung der Inter-Prozess-Kommunikation und zur Steuerung der Maschine durch den Debugger werden die oben beschriebenen \emph{Qt}-Klassen \emph{QSystemSemaphore} und \emph{QSharedMemory} verwendet. Hierbei dienen die Semaphoren dazu die Maschine durch den Debugger Steuern zu können um so den Ablauf kontrollieren zu können. Der \emph{QSharedMemory} dient dazu den Datenaustausch zwischen Debugger und Maschine zu realisieren. So werden Register- und Speicherinhalte durch die Maschine über den \emph{QSharedMemory} dem Debugger zur Verfügung Gestellt, damit dieser diese Anzeigen oder auch Manipulieren kann. Auch werden über den gemeinsamen Speicher Daten zur Steuerung ausgetauscht, z.B. ob ein Abbruch durch den Benutzer vorliegt.

Realisiert ist dies durch ein \emph{Struct} welche in einer eigenen Header-Datei deklariert ist und sowohl vom Debugger als auch vom Kern eingebunden wird. Dies hat den Vorteil, dass die Größe eines \emph{Structs} fest definiert ist und die benötigte Speichergröße beim anlegen des \emph{QSharedMemory} schon bekannt ist. Auch erlaubt dies einen einfachen Zugriff auf die Daten, da der \emph{QSharedMemory} bei zugriff nur einen void pointer zurückliefert. Dieser muss nur durch einen \emph{Cast} in einen \emph{Pointer} auf das entsprechende \emph{Struct} umgewandelt werden und der Zugriff ist möglich.


\paragraph{Kontrollzyklus}
Der Kontrollzyklus wurde mit vier \emph{QSystemSemaphore} realisiert, die in diesem Fall in vereinfachter Verwendung als Mutex verwendet werden. Dies wird wie schon erläutert Dadurch erreicht, dass nur eine Ressource zur Verfügung gestellt wird. Dies liegt auch darin begründet, dass von der Verwendeten \emph{Qt}-Bibliothek kein prozess-übergreifendes Mutex zur Verfügung gestellt wird, aber dennoch dafür nur \emph{Qt}-Klassen verwendet werden sollten um genannte Zielsetzungen wie Plattform-Unabhängigkeit nicht zu verletzen.

Betrachtet man den einen Zyklus des Maschinenkerns, so besteht dieser aus zwei Hauptsächlichen Abläufen. Einem \emph{Fetch}, gefolgt von einem \emph{Execute}. Beide sollen nur auf expliziten Befehl des Debuggers ausgeführt werden. Dafür werden zwei der Semaphoren verwendet. Die Semaphoren werden mit null verfügbaren Ressourcen initialisiert. Bevor der Kern nun entweder das \emph{Fetch} oder \emph{Execute} ausführt, versucht er von der jeweiligen Semaphore eine Ressource anzufordern. Da in beiden Fällen noch keine Ressource vorhanden sind, müssen diese erst durch den Debugger frei gegeben werden. Somit wird der Kern solange blockiert, bis dies geschieht. Somit ist es dem Debugger möglich, den genauen Zeitpunkt zu bestimmen, wann das \emph{Fetch} oder \emph{Execute} ausgeführt werden darf.

Die zwei weiteren Semaphoren dienen wiederum dem Kern die Möglichkeit dem Debugger zu signalisieren, wann der \emph{Fetch}- oder \emph{Execute}-Vorgang beendet worden sind. In diesem Fall geschieht das Gegenteil, und der Debugger fordert von den Semaphoren eine Ressource an. Ist der Kern mit seinen Aufgaben fertig, so gibt er diese Ressource frei, und der Debugger weiß, dass der Vorgang abgeschlossen ist, und kann weiter arbeiten.

Weitere Aufgabe des Kerns ist es, vor einem \emph{Fetch} oder \emph{Execute} eventuell von dem Debugger manipulierte Daten aus dem \emph{QSharedMemory} zu kopieren, und danach die vom Kern veränderte Daten im \emph{QSharedMemory} zu hinterlegen. Der Debugger hat so die Möglichkeiten die Daten und Zustände der Maschine zu manipulieren, ohne direkten Zugriff darauf zu haben. Weiterhin erhält er so wichtige Informationen wie die aktuelle Instruktions-Adresse, anhand deren er z.B. entscheiden kann ob ein Haltepunkt vorliegt. In diesem Fall kann er z.B. erst auf eine Benutzereingabe warten, biss er anhand der Semaphore dem Kern die Freigabe zur weiteren Abarbeitung des geladenen Programmes gibt.

Dies geschieht zwischen \emph{Fetch} und vor der nächsten Instruktions-Abarbeitung mittels \emph{Execute}. Dadurch ist der als nächstes auszuführende Befehl anhand der Instruktions-Adresse dem Debugger bekannt, und anhand den vom Assembler erzeugten Zusatzinformationen auch die zugehörige Quelldatei und die Zeilennummer. So kann die entsprechende Codestelle vor deren Ausführung im Debugger angezeigt werden. Wird nun vom Benutzer ein Einzelschritt ausgeführt, so wird ihm nach dessen Ausführung die nächste auszuführende Instruktion angezeigt und der kann das Ergebnis der vorherigen Instruktions-Abarbeitung begutachten.  

\subsection{Qt}

Wie schon häufiger erläutert wurde für die Umsetzung des graphischen Debuggers auf die \emph{Qt}-Bibliothek zurückgegriffen. Es sollen nun weiter Gründe für die Entscheidung abseits des schon genannten Vorteils der Plattform-Unabhängigkeit erläutert und auf weitere Eigenheiten von \emph{Qt} eingegangen werden. Im Detail soll davon das \emph{Signals \& Slots} Prinzip von \emph{Qt} erläutert werden, welches die Kommunikation zwischen den einzelnen Objekten realisiert.

\paragraph{Entscheidungsgründe für Qt}
Weiterer Grund für die Entscheidung für \emph{Qt} waren die vorhandenen Erfahrungswerte in der Entwicklung mit \emph{Qt}. Dies ist ein nicht zu unterschätzender Faktor, der deutliche Zeitersparnis bei der Entwicklung der Oberfläche mit sich gebracht hat. Dies ist auf zwei wesentliche Aspekte zurückzuführen. Einmal ist dies der Wegfall von einem gewissen Lernprozess, der immer Notwendig ist, wenn sich in neue Konzepte eingearbeitet werden muss, wie z.B. einer Bibliothek zur Oberflächen-Programmierung. Weiterhin wird auch im Verlauf der Entwicklung Aufwand eingespart, da schon bekannte Problemlösungen angewendet und die Aneignung neuer entfällt.

Ein zusätzlicher Vorteil von \emph{Qt} war in diesem Fall auch, dass \emph{Qt} eine Bibliothek für die Programmiersprache \emph{C++} ist. Dies hatte gerade den Vorteil, dass der Maschinenkern in \emph{C} geschrieben ist. Da \emph{C++} nativ das einbinden von \emph{C}-Quelltext erlaubt, konnte die "Main"-Funktion der Maschine unproblematisch um \emph{C++} Anteile zur Inter-Prozess-Kommunikation erweitert werden.

Auch für die Verwendung von \emph{Qt} sprach, dass \emph{Qt} nicht nur eine Bibliothek zur reinen Oberflächenprogrammierung ist, sondern auch wie schon oben beschrieben z.B. Klassen für die Inter-Prozess-Kommunikation mitbringt. Somit hat sich letztlich die Entscheidung für \emph{Qt} als sinnvoll erwiesen.

\paragraph{Signals \& Slots Prinzip}
Beim \emph{Signals \& Slots} Prinzip handelt es sich um ein Konzept, um \emph{Callback}-Funktionen zu verdecken, erweitern und deren Verwendung zu vereinfachen um Signale zwischen oder innerhalb von Objekten zu Verschicken. In \emph{Qt} dienen sie hauptsächlich in der Oberflächenprogrammierung dazu, Oberflächenereignisse weiter zu leiten und die mit Funktionalität zu hinterlegen.

Das \emph{Signals \& Prinzips }umfasst fünf Komponenten:
\begin{itemize}
\item \emph{Signals}, die zwischen den Objekten versendet werden.
\item \emph{Emit}, um ein Signal auszulösen.
\item \emph{Slot}, um Signale aufzunehmen und Funktionalität bereitzustellen.
\item \emph{Connect}, um das von einem Objekt gesendete Signal einem Slot zuzuweisen.
\item \emph{Dissconect}, um eine Signal-Slot Verknüpfung wieder aufzuheben.
\end{itemize}

Anhand eines Beispieles soll die Funktionsweise dieses Prinzip näher erläutert werden.

\begin{lstlisting}[numbers=left, numberstyle=\tiny]
class  Debugger
{
signals:
  void requestOpenFileInTab (IFile *file);
private:
  void showCodeLine(int instructionAddress) {
    ...
    if (!asmFiles[i]->isOpen()) {
      emit requestOpenFileInTab (asmFiles [i]);
    ...
  }
};

class CodeEditor
{
public slots:
  void openFileInTab (IFile *file) {
  ...
  }
};	

class UMachGui
{
  UMachGui() {  	
    mDebugger = new Debugger;
    mCodeEditor = new CodeEditor;
    connect (m_debugger, SIGNAL(requestOpenFileInTab(IFile)),
             m_codeEditor, SLOT(openFileInTab(IFile)));  
  }
}
\end{lstlisting}

In diesem Beispiel wird mit Hilfe eines \emph{connects()} das Signal \emph{requestOpenFileInTab(IFile)} des Objekts \emph{mDebugger} vom Typ Debugger mit dem Slot \emph{openFileInTab(IFile)} des Objektes \emph{mCodeEditor} vom Typ \emph{CodeEditor} verbunden. Somit werden zukünftig alle Signale diesen Typs die vom Objekt \emph{mDebugger} mit einem \emph{Emit} emittiert werden an den verbundenen Slot des Objektes \emph{mCodeEditor} weitergeleitet.

\paragraph{Qt Meta-Object-Compiler}
Das \emph{Signals \& Slots }Beispiel zeigt auch, dass \emph{Qt}-Schlüsselwörter einführt, welche von einem standardisierten \emph{C++}-Compiler nicht verstanden werden. Um dieses Problem zu Lösen, gibt es einen sogenannten \emph{Meta-Object-Compiler}. Dieser wird vor dem eigentlichen Compilieren ausgeführt und wandelt den um \emph{Qt}-Schlüsselwörter erweiterten Code in reinen \emph{C++}-Code um.

\subsection{GUI}

Im letzten Kapitel soll eine kleine Übersicht über die letztendliche Oberfläche und der von der Oberfläche verwendeten Projektdateien gegeben werden.

\paragraph{Bestandteile}
Die Software besteht aus einem Hauptfenster, welches einen Editor enthält und mehrere Dateien in Reiter geöffnet halten kann. Weiterhin steht eine Liste bereit, welche alle zu einem Projekt gehörigen Dateien auflistet. Alle dort aufgelisteten Dateien können im Editor geöffnet werden. Eine Toolbar gibt schnellen Zugriff auf häufig verwendete Funktionen, wie "Nächster Haltepunk", etc.
Als extra Fenster gibt es eine Tabelle zum selektiven anzeigen bestimmter Registerinhalte und ein Fenster zum setzten von Haltepunkten anhand Zeilennummern oder Label. Auch ein Optionsfenster ist vorhanden, in dem momentan nur die Speichergröße der Maschine eingestellt werden kann. Eine Tabelle, welche die Daten der Variablen anzeigt und deren Manipulation erlaubt, ist im Hauptfenster untergebracht.

\paragraph{Projektdatei}
Die Notwendigkeit einer Projektdatei hat sich insofern gezeigt, da bei vorhandener Editierbarkit des Quelltextes in der Entwicklungsumgebung auch ein \emph{Build-Prozess} vor dem Ausführen notwendig wurde. Somit erfüllt diese momentan Hauptsächlich die Aufgabe einer \emph{Make}-Datei, in der alle notwendigen Quelldateien hinterlegt sind, da eine solche vom Assembler nicht direkt angeboten wird. In der Entwicklungsumgebung können neue oder bestehende Quelldateien einem Projekt hinzugefügt oder entfernt werden. Projektdateien können gespeichert, geladen oder neu angelegt werden.  Aufbau der Projektdatei ist wie folgt: Diese besteht aus zweckspezifischen abschnitten, die mit einem Marker eingeleitet werden, welcher mit einem „.“ beginnt, gefolgt von einem Schlüsselwort. Dies erlaubt zu einem späteren Zeitpunkt weitere Abschnitte einzuführen, in denen zusätzliche Informationen hinterlegt werden können, wie zum Beispiel gesetzte Haltepunkt, sodass diese beim laden eines Projektes wieder automatisch zur Verfügung stehen.

