
\section{Meist verwendete Befehle}

\begin{frame}{\texttt{SET}}
 \texttt{SET R1 5} oder \texttt{SET R1 label}
 
 Tut das und jenes. Label wird durch Adresse ersetzt.
 
 Beispiel:
\end{frame}


\begin{frame}{Sprungbefehle}
 \begin{itemize}
  \item \texttt{JMP label}
  \item \texttt{BL label}
        springt zum angegebenen Label, falls R1 kleiner R2 ist
        Weitere Möglichkeiten: \texttt{BLE, BG, BGE, BE}
 \end{itemize}
\end{frame}

\begin{frame}{Vergleiche}
 \texttt{CMP R1 R2}
 
 Immer or einer bedingten Verzweigung.
\end{frame}

\begin{frame}{IO-Befehle}
 \begin{itemize}
  \item \texttt{IN R1 R2 ZERO}
  \item \texttt{OUT R1 R2 ZERO}
 \end{itemize}
\end{frame}


\begin{frame}{Arithmetische Befehle}
 \begin{description}
  \item [\texttt{ADD}] Addiert $a+b = 0$
  \item [\texttt{SUB}]
  \item [\texttt{INC}]
  \item [\texttt{DEC}] Berechnet $\pi^{2} = \frac{a^i}{e}$
  \item [\texttt{MUL}]
  \item [\texttt{DIV}] Muss man auf die Null aufpassen, denn die Null
        wurde erst im 9ten Jahrhundert erfunden und alte Programme nicht
        mehr kompilieren.
 \end{description}
 
und die entsprechenden Immediate-Varianten
\end{frame}


\begin{frame}{Funktionen}  
 \texttt{CALL funktion} 
\end{frame}



\section{Hilfsfunktionen}

\begin{frame}{Hilfsfunktionen}
 \begin{itemize}
  \item inputint
  \item printint
  \item putchar
  \item newline
 \end{itemize}
\end{frame}



\section{Hello World}

\begin{frame}[fragile]{Hello World}
 \begin{lstlisting}
  SET R1 hello
  SET R2 13
  OUT R1 R2 ZERO
  .data
  .string hello "Hello World!"
 \end{lstlisting}
\end{frame}


\section{Fibonacci Zahlen}

\begin{frame}{\insertsection}% name der aktuellen section
 Folge $X_{n} = X_{n-1} + X_{n-2}$
 mit $X_{1} = 1$, und $X_{2} = 2$
\end{frame}



\section{Zahl raten}

\begin{frame}{\insertsection}% name der aktuellen section
 Erzeugt eine Pseudozufallszahl aus dem eingegebenen Seed.
 Gibt Rückmeldung ob die geratene Zahl kleiner oder größer 
 als die gesuchte ist.
 
 Zählt die Anzahl der Versuche.
\end{frame}


\section{Tic Tac Toe}

\begin{frame}{\insertsection}
 Belegung der Register:
  \begin{itemize}
   \item R1-R9: Spielfelder
   \item R10: Aktueller Spieler
   \item R20: Anzahl der Spielzüge
  \end{itemize}

 Spielzyklus:
  \begin{enumerate}
  \item Eingabe
  \item Ausgabe
  \item Auswertung
  \end{enumerate}


Am ende des Spiels:
\begin{itemize}
 \item neue Runde oder Beenden
 \item bei neuer Runde aufräumen
\end{itemize}
\end{frame}


