%%%%%%%%%%%%%%%%%%%%
%%  (C) by init0  %%
%%%%%%%%%%%%%%%%%%%%

\section{Zielsetzung}

\subsection{Aufgabe des Assemblers}

\begin{frame}{\insertsubsection}
    Der Assembler übersetzt Quelltext in UMach Bytecode
    und erstellt Debuginformationen.
\end{frame}


\subsection{Erwünschte Eigenschaften}

\begin{frame}{\insertsubsection}
    \begin{itemize}
        \item ``Angenehme'' Syntax
        \item Performance
        \item Aussagekräftige Fehlermeldungen
        \item Nützliche Debuginformationen
    \end{itemize}
\end{frame}

\section{Bedienung \& Syntax}

\subsection{Bedienung}

\begin{frame}{\insertsubsection}
    Der Assembler ``\texttt{uasm}'' wird über eine Shell aufgerufen. \\~\\
    Aufrufsyntax: \texttt{uasm [-o outfile] [-g] [-w] file(s)}
\end{frame}

\subsection{Assembler Syntax Beispiel}

\begin{frame}[fragile]{\insertsubsection}
\begin{verbatim}
SET R1 hello
myloop:  #useful comment
    CALL println
    DEC R2
    CMP R2 ZERO
BNE myloop
#lines containing only a comment or nil are ignored
SET R1 9001
EOP
.data  #begin of data definitions
.string hello  "Hello World!"
.int    answer 42
.int    drink  0xCAFE
\end{verbatim}
\end{frame}

\subsection{Assembler Syntax Regeln}

\begin{frame}[fragile]{\insertsubsection}
    \begin{itemize}
        \item Beliebig viele \verb#`\t'# und \verb*#` '# vor und nach Tokens
        \item Nur Symbolbezeichner sind case-sensitiv
        \item Symbolbezeichner sind keine gültigen Zahlwerte
        \item Symbolbezeichner bestehen aus genau einem Wort
    \end{itemize}
\end{frame}

\begin{frame}[fragile]{\insertsubsection}
    \begin{itemize}
        \item Definition einer Sprungmarke endet mit \verb#`:'#
        \item Sprungmarken stehen in einer eigenen Zeile
        \item Für alle Dateien gilt der gleiche Namensraum
        \item Ab \verb|`#'| beginnt ein Kommentar bis einschl. \verb#`\n'#
    \end{itemize}
\end{frame}

\section{Implementierung}

\begin{frame}[fragile]{\insertsection}
    \begin{itemize}
        \item 2-pass Assembler, geschrieben in C99
        \item Abhängigkeiten: \texttt{glibc} und \texttt{glib}
        \item Funktioniert mit 32- und 64-Bit Compiler
    \end{itemize}
\end{frame}

\subsection{Toolchain}

\begin{frame}{\insertsection}{\insertsubsection}
    Ausschließlich FOSS Komponenten:
    \begin{itemize}
        \item GNU/Linux
        \item GCC und clang
        \item GNU make und gdb
        \item glib
        \item Git
        \item \LaTeX{}
        \item Valgrind
    \end{itemize}
\end{frame}

\subsection{Datenstrukturen}

\begin{frame}{\insertsection}{\insertsubsection}
    Drei Hashtabellen für
    \begin{enumerate}
        \item Befehle \textit{(statisch)}
        \item Register \textit{(statisch)}
        \item Symbole \textit{(dynamisch)}
    \end{enumerate}
    Eine verkettete Liste für den Inhalt von Daten
\end{frame}

\subsection{Commands}

\begin{frame}[fragile]{\insertsubsection}
\begin{lstlisting}
typedef enum {
    CMDFMT_NUL, CMDFMT_NNN, CMDFMT_R00,
    CMDFMT_RNN, CMDFMT_RR0, CMDFMT_RRN,
    CMDFMT_RRR
} cmdformat_t;

typedef struct {
    uint8_t     opcode;
    char       *opname;
    cmdformat_t format;
    char        has_label;
} command_t;
\end{lstlisting}
\end{frame}

\subsection{Register}

\begin{frame}[fragile]{\insertsubsection}
\begin{lstlisting}
typedef struct {
    uint8_t regcode;
    char   *regname;
} register_t;
\end{lstlisting}
\end{frame}

\subsection{Symbole}

\begin{frame}[fragile]{\insertsubsection}
\begin{lstlisting}
typedef enum {
    SYMTYPE_JUMP,
    SYMTYPE_INTDAT,
    SYMTYPE_STRDAT
} symbol_type_t;

typedef struct {
    char         *sym_name;
    symbol_type_t sym_type;
    uint32_t      sym_addr;
} symbol_t;
\end{lstlisting}
\end{frame}

\subsection{Variablen}

\begin{frame}[fragile]{\insertsubsection}
\begin{lstlisting}
typedef struct {
    char *label;
    char *value;
} string_data_t;

typedef struct {
    char   *label;
    int32_t value;
} int_data_t;
\end{lstlisting}
\end{frame}

\begin{frame}[fragile]{\insertsubsection}
\begin{lstlisting}
typedef enum {
    DATATYPE_INT, DATATYPE_STRING
} data_type_t;

typedef struct {
    data_type_t type;
    union {
        string_data_t string_data;
        int_data_t    int_data;
    };
} data_t;
\end{lstlisting}
\end{frame}

\subsection{Assembler Pass 1}

\begin{frame}{\insertsection}{\insertsubsection}
    ``Predict''
    \begin{itemize}
        \item Findet Sprungmarken und berechnet deren Adresse
        \item Berechnet die Codegröße
        \item Speichert Werte von Daten
        \item Berechnet Adressen von Daten
    \end{itemize}
\end{frame}

\subsection{Assembler Pass 2}

\begin{frame}{\insertsection}{\insertsubsection}
    ``Execute''
    \begin{itemize}
        \item Generiert UMach Bytecode
        \item Generiert Debuginformationen
        \item Speichert Werte von Daten in das Outputfile
    \end{itemize}
\end{frame}

\subsection{Alternativen}

\begin{frame}{\insertsection}{\insertsubsection}
    Der Parser muss nicht unbedingt selbst geschrieben werden,
    Tools wie z.B. \textit{GNU Bison} können aus einer Grammatik einen
    Parser generieren.\\~\\
    Vorteile:
    \begin{itemize}
        \item Flexibel
        \item Weniger Aufwand bei komplexer Syntax
    \end{itemize}
    Nachteile:
    \begin{itemize}
        \item Hoher Lernaufwand
        \item Geringere Performance
    \end{itemize}
\end{frame}

\section{Performance}

\begin{frame}[fragile]{\insertsection}
    $Durchsatz \approx 1.4 \times 10^6 \frac{Zeilen}{Sekunde}$
    (AMD Athlon II X2 250, 3 GHz) \\~\\
    Speicherbedarf wächst linear mit der Anzahl der Symbolen \\~\\
    Auflösung von Symbolen meist in $\mathcal{O}(1)$ \\~\\
    Keine linearen Suchen; Programmlaufzeit in $\mathcal{O}(n)$
\end{frame}

\section{Fehlerdiagnose}

\begin{frame}{\insertsection}
    Der \texttt{uasm} Assembler informiert den Benutzer u.a. über folgende
    Fehler im Quelltext:
    \begin{itemize}
        \item Unbekannte Befehle
        \item Ungültige Argumente zu einem Befehl
        \item Unbekannte Symbole (Sprungmarken und Variablen)
        \item Unbekannte Register
        \item Ungültige Deklaration einer Variable
        \item Re-Definition einer Sprungmarke
        \item etc\ldots{}
    \end{itemize}
\end{frame}

\begin{frame}[fragile]{\insertsection}
    Zusätzlich zur Art des Fehlers wird Name \& Zeilennummer der Quelltextdatei
    in welcher der Fehler gefunden wurde ausgegeben.\\~\\
    Beispiele:
\begin{verbatim}
echo.uasm, line 1: No such command: <SQRT>
echo.uasm, line 2: Command <CMP> expects RR0: REG,REG
echo.uasm, line 3: Unset label <getinput>
echo.uasm, line 4: Not a register: <R77>
echo.uasm, line 6: Label <get_input> already exists
echo.uasm, line 8: No content for <myint> provided
\end{verbatim}
\end{frame}

\section{Debuginformationen}

\begin{frame}{\insertsection}
    Wird das generieren von Debuginformationen per ``\texttt{\$uasm~-g}~\ldots''
    aktiviert, werden folgende Dateien erstellt:
    \begin{itemize}
        \item \texttt{u.out.fmap}
        \item \texttt{u.out.sym}
        \item \texttt{u.out.debug}
    \end{itemize}
\end{frame}

\subsection{File Map}

\begin{frame}[fragile]{\insertsubsection}
    Die Textdatei \texttt{u.out.fmap} enthält $n$ 1:1 Relationen (File-ID,~File-Name).
    \\~\\
    Beispiel:
\begin{verbatim}
0 tictactoe.uasm
1 func/inputint.uasm
2 func/newline.uasm
3 func/printint.uasm
4 func/putchar.uasm
\end{verbatim}
\end{frame}

\subsection{Debug File}

\begin{frame}[fragile]{\insertsubsection}
    Die Binärdatei \texttt{u.out.debug} enthält $n$ 32Bit-Datentripel
    (File-ID,~Line-No,~Address).
    \\~\\
    Beispiel:
\begin{verbatim}
00000000:  00 00 00 00  00 00 00 05  00 00 01 00
0000000c:  00 00 00 00  00 00 00 08  00 00 01 04
00000018:  00 00 00 00  00 00 00 09  00 00 01 08
00000024:  00 00 00 00  00 00 00 0d  00 00 01 0c
00000030:  00 00 00 00  00 00 00 0e  00 00 01 10
0000003c:  00 00 00 00  00 00 00 0f  00 00 01 14
00000048:  00 00 00 00  00 00 00 11  00 00 01 18
00000054:  00 00 00 00  00 00 00 15  00 00 01 1c
00000060:  00 00 00 00  00 00 00 16  00 00 01 20
0000006c:  00 00 00 00  00 00 00 17  00 00 01 24
\end{verbatim}
\end{frame}

\subsection{Symbol File}

\begin{frame}[fragile]{\insertsubsection}
    Die Textdatei \texttt{u.out.sym} enthält $n$ Datentripel (Address,~Symbol-Type,~Symbol-Name).
    \\~\\
    Beispiel:
\begin{verbatim}
000005e0 jmp start_inputint
000006b4 jmp printint_convert
0000050c jmp p1Won
000004d0 jmp draw
0000070c jmp putchar
00000784 str promptdraw
00000794 str promptend
00000540 jmp p2Won
000005ec jmp inputint_nextnbr
\end{verbatim}
\end{frame}
